% ===== CHAPTER 6 =====
\chapter{氢原子}
\label{chap:6}
    
\section{单粒子中心力问题}
\label{sec:6.1 The One-Particle Central Force Problem}
    在研究氢原子之前,我们先来看看单个粒子在中心力作用下运动这一更为普遍的问题。本节的结果适用于任何中心力问题。例如氢原子(第 \ref{sec:6.5 The Hydrogen Atom} 节)和各向同性的三维谐振子(第 \ref{sec:6.3 Reduction of the Two-Particle Problem to Two One-Particle Problems} 节)。

    \textbf{中心力}(central force)是由球面对称的势能函数导出的,这意味着中心力只是粒子与原点距离的函数$V = V\left(r\right)$。力和势能的关系由式(\ref{eq:5.31})给出:
    \begin{equation}
        F - \nabla V\left(x,y,z\right) = -\mathbf{i}\left(\partial V/\partial x\right) - \mathbf{j}\left(\partial V/\partial y\right) - \mathbf{k}\left(\partial V/\partial z\right)
        \label{eq:6.1}
    \end{equation}
    (\ref{eq:6.1})中的偏导数可以用链式法则求得[式(\ref{eq:5.53})- \ref{eq:5.55}]。在这个例子中,由于$V$只是$r$的函数,所以我们有$\left(\partial V/\partial\theta\right)_{r,\phi} = 0$和$\left(\partial V/\partial\phi\right)_{r,\theta} = 0$。因此,
    \begin{equation}
        \left(\frac{\partial V}{\partial x}\right)_{y,z} = \frac{\mathrm{d}V}{\mathrm{d}r}\left(\frac{\partial r}{\partial x}\right)_{y,z} = \frac{x}{r}\frac{\mathrm{d}V}{\mathrm{d}r}
        \label{eq:6.2}
    \end{equation}
    \begin{equation}
        \left(\frac{\partial V}{\partial y}\right)_{x,z} = \frac{y}{r}\frac{\mathrm{d}V}{\mathrm{d}r}, \quad \left(\frac{\partial V}{\partial z}\right)_{x,y} = \frac{z}{r}\frac{\mathrm{d}V}{\mathrm{d}r}
        \label{eq:6.3}
    \end{equation}
    其中我们用到了式(\ref{eq:5.57})和(\ref{eq:5.58})。则方程(\ref{eq:6.1})可以写成
    \begin{equation}
        \mathbf{F} = -\frac{1}{r}\frac{\mathrm{d}V}{\mathrm{d}r}\left(x\mathbf{i}+y\mathbf{j}+z\mathbf{k}\right) = -\frac{\mathrm{d}V\left(r\right)}{\mathrm{d}r}\frac{\mathbf{r}}{r}
        \label{eq:6.4}
    \end{equation}
    其中$\mathbf{r}$的计算使用了式(\ref{eq:5.33})。式(\ref{eq:6.4})中的$\mathbf{r}/r$是径向上的单位矢量。因此中心力是径向的。

    现在,让我们来看看受中心力作用的单粒子量子力学。其哈密顿算符为
    \begin{equation}
        \hat{H} = \hat{T} + \hat{V} = -\left(\hbar^2/2m\right)\nabla^2 + V\left(r\right)
        \label{eq:6.5}
    \end{equation}
    其中$\nabla^2 \equiv \partial^2/\partial x^2 + \partial^2/\partial y^2 + \partial^2/\partial z^2$[式(\ref{eq:3.46})]。由于$V$是球对称的,我们将在球坐标系中进行计算。因此,我们希望将拉普拉斯算符转换为球坐标的形式。我们已经有了$\partial/\partial x$、$\partial/\partial y$和$\partial /\partial z$的表达式[式(\ref{eq:5.62})-(\ref{eq:5.65})],通过将这些算符平方并相加,我们就得到了拉普拉斯算符。该计算留作课后习题。结果为(问题6.4):
    \begin{equation}
        \nabla^2 = \frac{\partial^2}{\partial r^2} + \frac{2}{r}\frac{\partial}{\partial r} + \frac{1}{r^2}\frac{\partial^2}{\partial\theta^2} + \frac{1}{r^2}\cot\theta\frac{\partial}{\partial\theta} + \frac{1}{r^2\sin^2\theta}\frac{\partial^2}{\partial\phi^2}
        \label{eq:6.6}
    \end{equation}

    回过头来看式(\ref{eq:5.68}),其中给出了单粒子轨道角动量大小平方的算符$\hat{L}^2$,我们可以看到
    \begin{equation}
        \nabla^2 = \frac{\partial^2}{\partial r^2} + \frac{2}{r}\frac{\partial}{\partial r} - \frac{1}{r^2\hbar^2}\hat{L}^2
        \label{eq:6.7}
    \end{equation}
    哈密顿算符(\ref{eq:6.5})变为
    \begin{equation}
        \hat{H} = -\frac{\hbar^2}{2m}\left(\frac{\partial^2}{\partial r^2} + \frac{2}{r}\frac{\partial}{\partial r}\right) + \frac{1}{2mr^2}\hat{L}^2 + V\left(r\right)
        \label{eq:6.8}
    \end{equation}

    在经典力学中,受中心力作用的粒子其角动量是守恒的(第\ref{sec:5.3 Angular Momentum of a One-Particle System}节)。在量子力学中,我们可能会问,我们是否可以拥有能量和角动量都有确定值的状态?为了使$\hat{H}$的本征函数组也是$\hat{L}^2$的本征函数,对易子$\left[\hat{H},\hat{L}^2\right]$必须为零。我们有
    \begin{equation*}
        \left[\hat{H},\hat{L}^2\right] = \left[\hat{T},\hat{L}^2\right] + \left[\hat{V},\hat{L}^2\right]
    \end{equation*}
    \begin{equation*}
        \left[\hat{T}, \hat{L}^2\right] = \left[-\frac{\hbar^2}{2m}\left(\frac{\partial ^2}{\partial r^2} + \frac{2}{r}\frac{\partial}{\partial r}\right) + \frac{1}{2mr^2}\hat{L}^2, \hat{L}^2\right]
    \end{equation*}
    \begin{equation}
        \left[\hat{T},\hat{L}^2\right] = -\frac{\hbar^2}{2m}\left[\frac{\partial^2}{\partial r^2} + \frac{2}{r}\frac{\partial}{\partial r}, \hat{L}^2\right] + \frac{1}{2m}\left[\frac{1}{r^2}\hat{L}^2, \hat{L}^2\right]
        \label{eq:6.9}
    \end{equation}
    回忆$\hat{L}^2$只是$\theta$和$\phi$的函数,而不是$r$的函数[式(\ref{eq:5.68})]。因此它与任何包含$r$的算符可对易。[为了证明这一结论,我们必须使用(\ref{eq:5.47})类似的关系,但将$x$和$z$分别换成$r$和$\theta$。]因此,式(\ref{eq:6.9})中第一项的对易子为零。此外,由于任何算符都与它自己对易,式(\ref{eq:6.9})第二项的对易子也为零。因此,$\left[\hat{T},\hat{L}^2\right] = 0$。同样地,由于$\hat{L}^2$不包含$r$,$V$只是$r$的函数,我们有$\left[\hat{V},\hat{L}^2\right] = 0$。因此,
    \begin{equation}
        \left[\hat{H},\hat{L}^2\right] = 0, \quad \text{若} V = V\left(r\right)
        \label{eq:6.10}
    \end{equation}
    当势能函数与$\theta$和$\phi$都无关时,$\hat{H}$和$\hat{L}^2$可对易。

    现在,我们来考虑算符$\hat{L}_z = -\mathrm{i}\hbar\partial/\partial\phi$[式(\ref{eq:5.67})]。由于$\hat{L}_z$不包含$r$,因此它与$\hat{L}^2$可对易[式(\ref{eq:5.50})],因此$\hat{H}$(\ref{eq:6.8})和$\hat{L}_z$也可对易。我们有
    \begin{equation}
        \left[\hat{H}, \hat{L}_z\right] = 0, \quad \text{若} V = V\left(r\right)
        \label{eq:6.11}
    \end{equation}

    因此,我们现在有了中心力问题中,一组同时是$\hat{H}$、$\hat{L}^2$和$\hat{L}_z$的本征函数的函数。令$\psi$表示这些共同本征函数:
    \begin{equation}
        \boxed{
            \hat{H}\psi = E\psi
        }
        \label{eq:6.12}
    \end{equation}
    \begin{equation}
        \boxed{
            \hat{L}^2\psi = l\left(l+1\right)\hbar^2\psi, \quad l = 0, 1, 2, \ldots
        }
        \label{eq:6.13}
    \end{equation}
    \begin{equation}
        \boxed{
            \hat{L}_z\psi = m\hbar\psi, \quad m = -l, -l+1, \ldots, l-1, l
        }
        \label{eq:6.14}
    \end{equation}
    其中,我们用到了式(\ref{eq:5.104})和式(\ref{eq:5.105})。

    根据式(\ref{eq:6.8})和(\ref{eq:6.13}),对薛定谔方程(\ref{eq:6.12}),有
    \begin{equation*}
        -\frac{\hbar^2}{2m}\left(\frac{\partial^2}{\partial r^2} + \frac{2}{r}\frac{\partial}{\partial r}\right) + \frac{1}{2mr^2}\hat{L}^2\psi + V\left(r\right)\psi = E\psi
    \end{equation*}
    \begin{equation}
        -\frac{\hbar^2}{2m}\left(\frac{\partial^2}{\partial r^2} + \frac{2}{r}\frac{\partial}{\partial r}\right) + \frac{l\left(l+1\right)\hbar^2}{2mr^2}\psi + V\left(r\right)\psi = E\psi
        \label{eq:6.15}
    \end{equation}

    $\hat{L}^2$的本征函数是球谐函数$Y_l^m\left(\theta, \phi\right)$,由于$\hat{L}^2$不包含$r$,我们可以将任意一个关于$r$的函数与球谐函数相乘,新函数依然是$\hat{L}^2$和$\hat{L}_z$的本征函数。因此,
    \begin{equation}
        \boxed{
            \psi = R\left(r\right)Y_l^m\left(\theta, \phi\right)
        }
        \label{eq:6.16}
    \end{equation}
    将(\ref{eq:6.16})代入(\ref{eq:6.15}),然后两边同除以$Y_l^m$,得到关于未知函数$R\left(r\right)$的微分方程:
    \begin{equation}
        -\frac{\hbar^2}{2m}\left(R^{\prime\prime}+ \frac{2}{r}R^{\prime}\right) + \frac{l\left(l+1\right)\hbar^2}{2mr^2}R + V\left(r\right)R = ER\left(r\right)
        \label{eq:6.17}
    \end{equation}

    我们已经证明:\textit{对于任意势能函数$V\left(r\right)$是球对称的单粒子问题,定态波函数为$\psi = R\left(r\right)Y_l^m\left(\theta, \phi\right)$,其中$R\left(r\right)$满足微分方程(\ref{eq:6.17})。}通过指定式(\ref{eq:6.17})中势能函数$V\left(r\right)$的形式,我们可以针对某个特定问题解决它。

\section{非相互作用粒子和变量分离}
\label{sec:6.2 Noninteracting Particles and Separation of Variables}
    到目前为止,我们只解决了单粒子量子力学问题。氢原子是一个双粒子系统,作为处理氢原子的第一步,我们首先考虑一个更简单的情况,即两个不相互作用的粒子。

    假设一个系统由两个无相互作用的粒子1和2组成。令$q_1$和$q_2$分别表示粒子1的$\left(x_1,y_1,z_1\right)$坐标和粒子2的$\left(x_2,y_2,z_2\right)$坐标。由于粒子之间没有相互作用力,系统的经典机械能就是两个粒子的能量之和:$E = E_1+E_2 = T_1+V_1 + T_2 + V_2$,经典哈密顿量是两个粒子哈密顿量之和:$H = H_1 + H_2$。因此,系统的哈密顿算符为
    \begin{equation*}
        \hat{H} = \hat{H}_1 + \hat{H}_2
    \end{equation*}
    其中$\hat{H}_1$只和坐标$q_1$有关,动量算符$\hat{p}_1$对应于$q_1$。系统的薛定谔方程为
    \begin{equation}
        \left(\hat{H}_1 + \hat{H}_2\right)\psi\left(q_1,q_2\right) = E\psi\left(q_1,q_2\right)
        \label{eq:6.18}
    \end{equation}
    我们尝试用分离变量法来求解(\ref{eq:6.18}),令
    \begin{equation}
        \psi\left(q_1,q_2\right) = G_1\left(q_1\right)G_2\left(q_2\right)
        \label{eq:6.19}
    \end{equation}
    我们有
    \begin{equation}
        \hat{H}_1G_1\left(q_1\right)G_2\left(q_2\right) + \hat{H}_2G_1\left(q_1\right)G_2\left(q_2\right) = EG_1\left(q_1\right)G_2\left(q_2\right)
        \label{eq:6.20}
    \end{equation}
    由于$\hat{H}_1$只包含粒子1的位置和动量算符,我们有$\hat{H}_1\left[G_1\left(q_1\right)G_2\left(q_2\right)\right] = G_2\left(q_2\right)\hat{H}_1G_1\left(q_1\right)$,因此,就$\hat{H}_1$而言,$G_2\left(q_2\right)$是常数。使用这个方程以及类似的关于$\hat{H}_2$的方程,我们发现(\ref{eq:6.20})变为
    \begin{equation}
        G_2\left(q_2\right)\hat{H}_1G_1\left(q_1\right) + G_1\left(q_1\right)\hat{H}_2G_2\left(q_2\right) = EG_1\left(q_1\right)G_2\left(q_2\right)
        \label{eq:6.21}
    \end{equation}
    \begin{equation}
        \frac{\hat{H}_1G_1\left(q_1\right)}{G_1\left(q_1\right)} + \frac{\hat{H}_2G_2\left(q_2\right)}{G_2\left(q_2\right)} = E
        \label{eq:6.22}
    \end{equation}
    现在,通过与式 (\ref{eq:3.65}) 相同的论证,我们得出结论:在 (\ref{eq:6.22}) 中,左边的每项必须是常数。令$E_1$和$E_2$分别表示这些常数,我们有
    \begin{equation*}
        \frac{\hat{H}_1G_1\left(q_1\right)}{G_1\left(q_1\right)} = E_1, \quad \frac{\hat{H}_2G_2\left(q_2\right)}{G_2\left(q_2\right)} = E_2
    \end{equation*}
    \begin{equation}
        E = E_1 + E_2
        \label{eq:6.23}
    \end{equation}
    因此,当系统由两个无相互作用的粒子组成时,通过求解下式,我们可以将双粒子问题简化为两个单粒子问题:
    \begin{equation}
        \hat{H}_1G_1\left(q_1\right) = E_1G_1\left(q_1\right), \quad \hat{H}_2G_2\left(q_2\right) = E_2G_2\left(q_2\right)
        \label{eq:6.24}
    \end{equation}
    它们是每个粒子的独立薛定谔方程。

    将该结果推广到$n$个无相互作用的粒子,我们有
    \begin{equation*}
        \hat{H} = \hat{H}_1 + \hat{H}_2 + \ldots + \hat{H}_n
    \end{equation*}
    \begin{equation}
        \boxed{
                \psi\left(q_1,q_2,\ldots,q_n\right) = G_1\left(q_1\right)G_2\left(q_2\right)\ldots G_n\left(q_n\right)
        }
        \label{eq:6.25}
    \end{equation}
    \begin{equation}
        \boxed{
            E = E_1 + E_2 + \ldots + E_n
        }
        \label{eq:6.26}
    \end{equation}
    \begin{equation}
        \boxed{
            \hat{H}_iG_i = E_iG_i, \quad i = 1, 2, \ldots, n
        }
        \label{eq:6.27}
    \end{equation}
    \textit{对于一个无相互作用粒子系统,能量是每个粒子各自能量的总和,波函数是每个粒子波函数的乘积。利用哈密顿算符$\hat{H}_i$,求解粒子$i$的薛定谔方程,即可求得粒子$i$的波函数$G_i$。}

    这些结果也适用于单个粒子,其哈密顿算符是每个坐标的独立项之和:
    \begin{equation*}
        \hat{H} = \hat{H}_x\left(\hat{x},\hat{p}_x\right) + \hat{H}_y\left(\hat{y},\hat{p}_y\right) + \hat{H}_z\left(\hat{z},\hat{p}_z\right)
    \end{equation*}
    在这种情况下,我们得出结论:波函数和能量分别为
    \begin{equation*}
        \psi\left(x,y,z\right) = F\left(x\right)G\left(y\right)K\left(z\right), E = E_x + E_y + E_z
    \end{equation*}
    \begin{equation*}
        \hat{H}_xF\left(x\right) = E_xF\left(x\right), \quad \hat{H}_yG\left(y\right) = E_yG\left(y\right), \quad \hat{H}_zK\left(z\right) = E_zK\left(z\right)
    \end{equation*}
    例如三维盒子中的质点(第 \ref{sec:3.5 The Particle in a Three-Dimensional Box} 节)、三维自由质点(问题 3.42)和三维谐振子(问题 4.20)。
\section{将双粒子问题简化为两个单粒子问题}
\label{sec:6.3 Reduction of the Two-Particle Problem to Two One-Particle Problems}
    氢原子包含两个粒子——质子和电子。对于一个双粒子系统(粒子1的坐标为$\left(x_1,y_1,z_1\right)$,粒子2的坐标为$\left(x_2,y_2,z_2\right)$),粒子间相互作用的势能通常只是粒子相对坐标$x_2-x_1$、$y_2-y_1$和$z_2-z_1$的函数。在这种情况中,双粒子问题可以简化为两个单粒子问题,正如我们现在所要证明的。

    先考虑经典力学处理两个相互作用粒子的情况,令其质量分别为$m_1$和$m_2$。我们用从直角坐标系原点出发的半径矢量$\mathbf{r}_1$和$\mathbf{r}_2$来指定它们的位置(图\ref{fig:6.1})。粒子1和2的坐标分别为$\left(x_1,y_1,z_1\right)$和$\left(x_2,y_2,z_2\right)$。我们从粒子1到粒子2画出矢量$\mathbf{r} = \mathbf{r}_2 - \mathbf{r}_1$,并用$x$、$y$和$z$表示其分量:
    \begin{figure}[h!]
        \centering
        \includegraphics[width=0.4\textwidth]{Figures/6.1.png}
        \caption{质心位于$C$的双粒子系统}
        \label{fig:6.1}
    \end{figure}
    \begin{equation}
        \boxed{
            x = x_2 - x_1, \quad y = y_2 - y_1, \quad z = z_2 - z_1
        }
        \label{eq:6.28}
    \end{equation}
    $x$、$y$和$z$称为\textbf{相对坐标}或\textbf{内坐标}(relative or internal coordinates)。

    我们现在从原点指向系统的质心$C$画出矢量$\mathbf{R}$,其分量用$X$、$Y$和$Z$表示:
    \begin{equation}
        \mathbf{R} = \mathbf{i}X + \mathbf{j}Y + \mathbf{k}Z
        \label{eq:6.29}
    \end{equation}
    根据双粒子系统质心的定义,有
    \begin{equation}
        X = \frac{m_1x_1 + m_2x_2}{m_1 + m_2}, \quad Y = \frac{m_1y_1 + m_2y_2}{m_1 + m_2}, \quad Z = \frac{m_1z_1 + m_2z_2}{m_1 + m_2}
        \label{eq:6.30}
    \end{equation}
    这三个方程与矢量方程等价:
    \begin{equation}
        \mathbf{R} = \frac{m_1\mathbf{r}_1 + m_2\mathbf{r}_2}{m_1 + m_2}
        \label{eq:6.31}
    \end{equation}
    我们也有
    \begin{equation}
        \mathbf{r} = \mathbf{r}_2 - \mathbf{r}_1
        \label{eq:6.32}
    \end{equation}
    我们将式(\ref{eq:6.31})和(\ref{eq:6.32})联立为关于$\mathbf{r}_1$和$\mathbf{r}_2$的方程组,并求解得到
    \begin{equation}
        \mathbf{r}_1 = \mathbf{R} - \frac{m_2}{m_1 + m_2}\mathbf{r}, \quad \mathbf{r}_2 = \mathbf{R} + \frac{m_1}{m_1 + m_2}\mathbf{r}
        \label{eq:6.33}
    \end{equation}
    方程(\ref{eq:6.31})和(\ref{eq:6.32})代表了一种将$x_1,y_1,z_1$和$x_2,y_2,z_2$坐标转换为$X,Y,Z$和$x,y,z$坐标的变换。考虑一下在这种变化下,系统的哈密顿量会发生什么变化。用字母顶上的点表示对时间的导数。粒子1的速度为[式(\ref{eq:5.34})]$\mathbf{v}_1 = \mathrm{d}\mathbf{r}_1/\mathrm{d}t = \dot{\mathbf{r}}_1$。系统的动能为两个粒子的动能之和:
    \begin{equation}
        T = \frac{1}{2}m_1\left|\dot{\mathbf{r}_1}\right|^2 + \frac{1}{2}m_2\left|\dot{\mathbf{r}_2}\right|^2
        \label{eq:6.34}
    \end{equation}
    将(\ref{eq:6.33})的时间导数代入(\ref{eq:6.34}),我们得到
    \begin{equation*}
        \begin{aligned}
            T = & \frac{1}{2}m_1\left(\dot{\mathbf{R}} - \frac{m_2}{m_1+m_2}\dot{\mathbf{r}}\right)\cdot \left(\dot{\mathbf{R}} - \frac{m_2}{m_1+m_2}\dot{\mathbf{r}}\right) \\
            & + \frac{1}{2}m_2\left(\dot{\mathbf{R}} + \frac{m_1}{m_1+m_2}\dot{\mathbf{r}}\right)\cdot \left(\dot{\mathbf{R}} + \frac{m_1}{m_1+m_2}\dot{\mathbf{r}}\right) \\
        \end{aligned}
    \end{equation*}
    其中,我们用到了$\left|\mathbf{A}\right|^2 = \mathbf{A}\cdot \mathbf{A}$[式(\ref{eq:5.24})]。使用矢量数量积的分配律,化简后,我们得到
    \begin{equation}
        T = \frac{1}{2}\left(m_1 + m_2\right)\left|\dot{\mathbf{R}}\right|^2 + \frac{1}{2}\left(\frac{m_1m_2}{m_1+m_2}\right)\left|\dot{\mathbf{r}}\right|^2
        \label{eq:6.35}
    \end{equation}
    令$M$为系统的总质量:
    \begin{equation}
        M \equiv m_1 + m_2
        \label{eq:6.36}
    \end{equation}
    我们定义双粒子系统的\textbf{折合质量}(reduced mass)为
    \begin{equation}
        \boxed{
            \mu \equiv \frac{m_1m_2}{m_1+m_2}
        }
        \label{eq:6.37}
    \end{equation}
    那么
    \begin{equation}
        T = \frac{1}{2}M\left|\dot{\mathbf{R}}\right|^2 + \frac{1}{2}\mu\left|\dot{\mathbf{r}}\right|^2
        \label{eq:6.38}
    \end{equation}

    (\ref{eq:6.38})的第一项是质量为$M$的整个系统平动产生的动能。\textbf{平动}(translational motion)是指每个质点都经历相同位移的运动。$\frac{1}{2}M\left|\dot{\mathbf{R}}\right|^2$是位于质心,具有质量$M$的假想粒子的动能。(\ref{eq:6.38})的第二项是两个粒子内运动(相对运动)产生的动能。内运动有两种形式。两个粒子间的距离$r$可以变化(振动),以及矢量$\mathbf{r}$的方向可以发生变化(转动)。注意:$\left|\dot{\mathbf{r}}\right| = \left|\mathrm{d}\mathbf{r}/\mathrm{d}t\right| \neq \mathrm{d}\left|\mathbf{r}\right|/\mathrm{d}t$。

    与六个原始坐标$x_1,y_1,z_1,x_2,y_2,z_2$相对应,我们有六个线性动量:
    \begin{equation}
        p_{x_1} = m_1\dot{x}_1, \quad \ldots, \quad p_{z_2} = m_2\dot{z}_2
        \label{eq:6.39}
    \end{equation}
    将(\ref{eq:6.34})和(\ref{eq:6.38})相比较,我们将新坐标$X,Y,Z,x,y,z$的动量定义为
    \begin{equation*}
        p_X \equiv M\dot{X}, \quad p_Y \equiv M\dot{Y}, \quad p_Z \equiv M\dot{Z}
    \end{equation*}
    \begin{equation*}
        p_x \equiv \mu\dot{x}, \quad p_y \equiv \mu\dot{y}, \quad p_z \equiv \mu\dot{z}
    \end{equation*}
    我们定义两个新动量矢量:
    \begin{equation*}
        \mathbf{p}_M = \mathbf{i}M\dot{X} + \mathbf{j}M\dot{Y} + \mathbf{k}M\dot{Z},
        \quad \text{和} \quad \mathbf{p}_{\mu} = \mathbf{i}\mu\dot{x} + \mathbf{j}\mu\dot{y} + \mathbf{k}\mu\dot{z}
    \end{equation*}
    将这些动量代入式(\ref{eq:6.38}),我们得到
    \begin{equation}
        T = \frac{\left|\mathbf{p}_M\right|^2}{2M} + \frac{\left|\mathbf{p}_{\mu}\right|^2}{2\mu}
        \label{eq:6.40}
    \end{equation}

    现在,我们来考虑势能。我们限制$V$只是两个粒子的相对坐标$x$、$y$和$z$的函数:
    \begin{equation}
        V = V\left(x,y,z\right)
        \label{eq:6.41}
    \end{equation}
    (\ref{eq:6.41})的一个例子是相互作用遵守库仑定律的两个带电粒子[见式(\ref{eq:3.53})]。根据对$V$的这一限制,哈密顿函数为
    \begin{equation}
        H = \frac{p_M^2}{2M} + \left[\frac{p_{\mu}^2}{2\mu} + V\left(x,y,z\right)\right]
        \label{eq:6.42}
    \end{equation}

    现在,假设我们有一个由质量为 $M$ 的粒子和质量为 $\mu$ 的粒子组成的系统,前者不受力,后者受势能函数 $V\left(x,y,z\right)$ 的作用。再假设这些粒子之间没有相互作用。如果$\left(X,Y,Z\right)$表示质量为$M$的粒子的坐标,而$\left(x,y,z\right)$表示质量为$\mu$的粒子的坐标,那么这个假设系统的哈密顿量是什么?显然,它与式(\ref{eq:6.42})完全相同。

    哈密顿量(\ref{eq:6.42})可以看作是两个无相互作用假想粒子的哈密顿量之和:质量为$M$的粒子其哈密顿量为$p_M^2/2M$,质量为$\mu$的粒子其哈密顿量为$p_{\mu}^2/2\mu + V\left(x,y,z\right)$。因此,第\ref{sec:6.2 Noninteracting Particles and Separation of Variables}的结果表明:系统的量子力学能量是两个假想粒子的能量之和[式(\ref{eq:6.23})]:$E = E_M + E_{\mu}$。由式(\ref{eq:6.24})和(\ref{eq:6.42})可知,平动能$E_M$可以通过求解薛定谔方程$\left(\hat{p}_M^2/2M\right)\psi_M = E_M\psi_M$得到。这是质量为$M$的自由粒子的薛定谔方程,因此其可能的本征值全是非负数[式(\ref{neq:2.31})]:$E_m \ge 0$。由式(\ref{eq:6.42})和(\ref{eq:6.24})可知,内运动能$E_{\mu}$可以通过求解以下薛定谔方程得到:
    \begin{equation}
        \left[\frac{\hat{p}_{\mu}^2}{2\mu} + V\left(x,y,z\right)\right]\psi_{\mu}\left(x,y,z\right) = E_{\mu}\psi_{\mu}\left(x,y,z\right)
        \label{eq:6.43}
    \end{equation}

    这样,我们就把两个粒子根据仅取决于相对坐标 $x$、$y$、$z$ 的势能函数 $V\left(x,y,z\right)$ 进行相互作用的问题分离成了两个独立的单粒子问题:(1)质量为$M$的整体系统平动,它只是在系统能量的基础上增加了一个恒非负的能量$E_M$;(2)通过求解质量为 $\mu$ 的假想粒子的薛定谔方程 (\ref{eq:6.43})来处理相对运动或内部运动,该粒子的坐标为相对坐标 $x$、$y$、$z$,并在势能 $V\left(x,y,z\right)$ 的作用下运动。

    例如,对于包含一个电子(e)和一个质子(p)的氢原子,原子的总能量为$E = E_M + E_{\mu}$,其中的$E_M$是质量为$M = m_e+m_p$的整个原子在空间中的平动能,其中的$E_{\mu}$是通过将$\mu = m_em_p/(m_e+m_p)$代入(\ref{eq:6.43})得到的,$V$ 是电子和质子相互作用的库仑定律势能;见第 \ref{sec:6.5 The Hydrogen Atom} 节。















\section{双粒子的刚性转子模型}
\label{sec:6.4 The Two-Particle Rigid Rotor}

\section{氢原子}
\label{sec:6.5 The Hydrogen Atom}

\section{束缚态氢原子波函数}
\label{sec:6.6 The Bound-State Hydrogen-Atom Wave Functions}

\section{类氢轨道}
\label{sec:6.7 Hydrogenlike orbitals}

\section{塞曼效应}
\label{sec:6.8 Zeeman effect}

\section{径向薛定谔方程的数值解法}
\label{sec:6.9 Numerical Solution of the Radial Schrödinger Equation}

\section*{总结}

\section*{习题}
