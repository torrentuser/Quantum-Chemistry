% ===== CHAPTER 13 =====
\chapter{双原子分子的电子结构}
\label{chap:13}
\section{Born-Oppenheimer近似}
\label{sec:13.1 The Born-Oppenheimer Approximation}

\section{双原子分子核的运动}
\label{sec:13.2 Nuclear Motion in Diatomic Molecules}

\section{原子单位制}
\label{sec:13.3 Atomic Units}

\section{氢分子离子}
\label{sec:13.4 The Hydrogen Molecular Ion}

\section{H$_2^+$离子基态的近似解}
\label{sec:13.5 Approximate Treatments of the H$_2^+$ Ground Electronic State}

\section{H$_2^+$离子激发态的分子轨道}
\label{sec:13.6 Molecular Orbitals for H$_2^+$ Excited States}

\section{同核双原子分子的分子轨道构型}
\label{sec:13.7 MO Configurations for Homonuclear Diatomic Molecules}

\section{双原子分子的分子光谱项}
\label{sec:13.8 Electronic Terms of Diatomic Molecules}

\section{氢分子}
\label{sec:13.9 The Hydrogen Molecule}

\section{价键理论求解H$_2$}
\label{sec:13.10 The Valence-Bond Treatment of H$_2$}

\section{分子轨道理论和价键理论的比较}
\label{sec:13.11 Comparison of the MO and VB Theories}

\section{同核双原子分子的分子轨道理论和价键理论波函数}
\label{sec:13.12 MO and VB Wave Functions for Homonuclear Diatomic Molecules}

\section{H$_2$分子的激发态}
\label{sec:13.13 Excited States of H$_2$}

\section{双原子分子的自洽场方法波函数}
\label{sec:13.14 SCF Wave Functions for Diatomic Molecules}

\section{分子轨道理论处理异核双原子分子}
\label{sec:13.15 MO Treatment of Heteronuclear Diatomic Molecules}

\section{价键理论处理异核双原子分子}
\label{sec:13.16 VB Treatment of Heteronuclear Diatomic Molecules}

\section{价电子近似}
\label{sec:13.17 The Valence-Electron Approximation}