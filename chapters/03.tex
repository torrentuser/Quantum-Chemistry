% ===== CHAPTER 3 =====
\chapter{算符}
\section{算符}
	现在,我们以一种比以前更一般的方式来发展量子力学理论。首先,我们将单粒子一维定态薛定谔方程 (\ref{eq:1.19 time-independent Schrödinger Equation}) 写成以下形式
	\begin{equation}
		\left[-\frac{\hbar^2}{2m}\frac{\mathrm{d}^2\psi}{\mathrm{d}x^2}+V\left(x\right)\right]\psi\left(x\right) = E\psi\left(x\right)
		\label{eq:3.1}
	\end{equation}
	(\ref{eq:3.1}) 中括号内的实体是一个\textit{算符}。方程(\ref{eq:3.1})表明我们有一个能量算符,它对波函数起作用后,波函数又回来了,不过是乘以一个允许的能量值。因此,接下来我们讨论算符。\\
	\indent \textbf{算符}是一种将给定函数转换为另一个函数的规则。例如,让 $\hat{D}$ 算符来表示对关于$x$的函数求导。我们用一个字母上的小尖号来表示算符。给定一个可导函数$f\left(x\right)$,则将$\hat{D}$作用到$f\left(x\right)$上即可表示为$\hat{D}f\left(x\right) = f^{\prime}\left(x\right)$。例如,$\hat{D}\left(x^2+3\mathrm{e}^{2x}\right) = 2x+6\mathrm{e}^{2x}$。如果$\hat{3}$是将函数乘以 3 的算符,那么$\hat{3}\left(x^2+3\mathrm{e}^x\right) = 3x^2+9\mathrm{e}^x$。如果 tan 是取函数正切值的算符,那么将 tan 应用于函数$x^2+1$,我们得到了$\tan\left(x^2+1\right)$。如果算符$\hat{A}$将函数$f\left(x\right)$转换为另一个函数$g\left(x\right)$,我们将其写作$\hat{A}f\left(x\right) = g\left(x\right)$。\\
	\indent 我们将两个算符$\hat{A}$和$\hat{B}$的\textbf{和}与\textbf{差}定义为
	\begin{equation}
		\boxed{
			\left(\hat{A}+\hat{B}\right)f\left(x\right) \equiv \hat{A}f\left(x\right)+\hat{B}f\left(x\right)
		}
		\label{eq:3.2 definition of operators' sum and differenct}
	\end{equation}
	\begin{equation*}
		\left(\hat{A}-\hat{B}\right)f\left(x\right) \equiv \hat{A}f\left(x\right)-\hat{B}f\left(x\right)
	\end{equation*}
	例如,若$\hat{D} \equiv \mathrm{d}/\mathrm{d}x$,那么
	\begin{equation*}
		\left(\hat{D}+\hat{3}\right)\left(x^3-5\right) = \hat{D}\left(x^3-5\right)+ \hat{3}\left(x^3-5\right) = 3x^2+\left(3x^3-15\right) = 3x^3+3x^2-15
	\end{equation*}
	\indent 一个算符也可以涉及多个变量。例如,算符$\partial^2/\partial x^2+\partial^2/\partial y^2$具有如下性质:
	\begin{equation*}
		\left(\partial^2/\partial x^2+\partial^2/\partial y^2\right)g\left(x,y\right)= \partial^2 g/\partial x^2 + \partial^2 g/\partial y^2
	\end{equation*}
	\indent 两个算符的\textbf{积}定义为
	\begin{equation}
		\boxed{
			\hat{A}\hat{B}f\left(x\right) = \hat{A}\left[\hat{B}f\left(x\right)\right]
		}
		\label{eq:3.3 definition of operators' product}
	\end{equation}
	换句话说,我们先将乘积右边的算符作用到函数$f\left(x\right)$上,然后将左边的算符作用到得到的函数上。例如,$\hat{3}\hat{D}f\left(x\right) = \hat{3}\left[\hat{D}f\left(x\right)\right] = \hat{3}f^{\prime}\left(x\right) = 3f^{\prime}\left(x\right)$\\
	\indent 算符$\hat{A}\hat{B}$和$\hat{B}\hat{A}$可能不会有同样的效果。例如,考虑算符$\mathrm{d}/\mathrm{d}x$和$\hat{x}$($\hat{x}$表示将函数乘以$x$):
	\begin{equation}
		\hat{D}\hat{x}f\left(x\right)=\frac{\mathrm{d}}{\mathrm{d}x}\left[xf\left(x\right)\right] = f\left(x\right) + f^{\prime}\left(x\right) = \left(\hat{1}+\hat{x}\hat{D}\right)f\left(x\right)
		\label{eq:3.4}
	\end{equation}
	\begin{equation*}
		\hat{x}\hat{D}f\left(x\right) = \hat{x}\left[\frac{\mathrm{d}}{\mathrm{d}x}f\left(x\right)\right] = xf^{\prime}\left(x\right)
	\end{equation*}
	在这个例子中,$\hat{A}\hat{B}$和$\hat{B}\hat{A}$是不同的算符。\\
	\indent 我们可以建立如下的\textbf{算符代数}。若对任意的函数$f$,有$\hat{A}f = \hat{B}f$,则我们说算符$\hat{A}$和算符$\hat{B}$是\textbf{相等的}。相等的算符作用到给定的函数上时,会有相同的结果。例如,式(\ref{eq:3.4})表明
	\begin{equation}
		\hat{D}\hat{x} = 1 + \hat{x}\hat{D}
		\label{eq:3.5}
	\end{equation}
	算符$\hat{1}$(乘以1)称作\textbf{单位算符}。算符$\hat{0}$(乘以0)称作\textbf{空算符}。我们通常省略简单常数算符顶上的小尖号。我们还可以将算符从算符方程的一边转移到另一边(问题 3.7 )。因此,式(\ref{eq:3.5})与$\hat{D}\hat{x}-\hat{x}\hat{D}-1 = 0$等价,其中省略了空算符和单位算符上的小尖号。\\
	\indent 算符运算遵循结合律:
	\begin{equation}
		\hat{A}\left(\hat{B}\hat{C}\right) = \left(\hat{A}\hat{B}\right)\hat{C}
		\label{eq:3.6 law of multiplication for operators}
	\end{equation}
	式(\ref{eq:3.6 law of multiplication for operators})的证明见问题3.10。例如,令$\hat{A} = \mathrm{d}/\mathrm{d}x$,$\hat{B} = \hat{x}$,而$\hat{C} = 3$。使用式(\ref{eq:3.5}),我们有
	\begin{equation*}
		\begin{aligned}
			\left(\hat{A}\hat{B}\right) = \hat{D}\hat{x} = 1+ \hat{x}\hat{D}, \quad & \left[\left(\hat{A}\hat{B}\right)\hat{C}\right]f = \left(1+\hat{x}\hat{D}\right)3f = 3f+3xf^{\prime}\\
			\left(\hat{B}\hat{C}\right) = 3\hat{x},  \qquad \quad \qquad & \left[\hat{A}\left(\hat{B}\hat{C}\right)\right]f = \hat{D}\left(3xf\right) = 3f+3xf^{\prime}
		\end{aligned}
	\end{equation*}
	\indent 算符代数与普通代数的一个主要区别是:数的运算遵守乘法的交换律,但算符不一定。例如:若$a$和$b$都是数,那么有$ab=ba$;但是$\hat{A}\hat{B}$和$\hat{B}\hat{A}$不一定相等。我们定义算符$\hat{A}$和$\hat{B}$的\textbf{交换子}$\left[\hat{A},\hat{B}\right]$为$\hat{A}\hat{B}-\hat{B}\hat{A}$:
	\begin{equation}
		\boxed{
			\left[\hat{A},\hat{B}\right] \equiv \hat{A}\hat{B} - \hat{B}\hat{A}
		}
		\label{eq:3.7 definition of commutator for two operators}
	\end{equation}
	如果有$\hat{A}\hat{B} = \hat{B}\hat{A}$,那么$\left[\hat{A},\hat{B}\right] = 0$,则我们说算符$\hat{A}$和$\hat{B}$是\textbf{可对易的}。如果$\hat{A}\hat{B} \neq \hat{B}\hat{A}$,那么$\hat{A}$和$\hat{B}$是不可对易的。注意$\left[\hat{A}, \hat{B}\right]f = \hat{A}\hat{B}f - \hat{B}\hat{A}f$。由于我们作用算符 3 和 $\mathrm{d}/\mathrm{d}x$ 的顺序没有区别,因此我们有
	\begin{equation*}
		\left[\hat{3},\frac{\mathrm{d}}{\mathrm{d}x}\right] = \hat{3}\frac{\mathrm{d}}{\mathrm{d}x} - \frac{\mathrm{d}}{\mathrm{d}x}\hat{3} = 0
	\end{equation*}
	\indent 从式(\ref{eq:3.5}),我们有
	\begin{equation}
		\left[\frac{\mathrm{d}}{\mathrm{d}x}, \hat{x}\right] = \hat{D}\hat{x}-\hat{x}\hat{D} = 1
		\label{eq:3.8}
	\end{equation}
	那么算符$\mathrm{d}/\mathrm{d}x$和$\hat{x}$是不可对易的。\\
	\begin{examplebox}
		\textbf{例题:}\\
		求$\left[z^3,\mathrm{d}/\mathrm{d}z\right]$。\\
		\\
		\indent 为了求出$\left[z^3,\mathrm{d}/\mathrm{d}z\right]$,我们需要将其作用到任意一个函数$g\left(x\right)$上。由式(\ref{eq:3.7 definition of commutator for two operators})交换子的定义及算符差和乘积的定义,我们有
		\begin{equation*}
			\begin{aligned}
				\left[z^3,\mathrm{d}/\mathrm{d}z\right]g = \left[z^3\left(\mathrm{d}/\mathrm{d}z\right) - \left(\mathrm{d}/\mathrm{d}z\right)z^3\right]g & = z^3\left(\mathrm{d}/\mathrm{d}z\right)g - \left(\mathrm{d}/\mathrm{d}z\right)\left(z^3g\right) \\
				& = z^3g^{\prime} - 3z^2g-z^3g^{\prime} = -3z^2g
			\end{aligned}
		\end{equation*}
		删去任意的函数$g$,我们有算符方程$\left[z^3,\mathrm{d}/\mathrm{d}z\right] = -3z^2$。\\
		\\
		\textbf{练习:}\\
		求$\left[\mathrm{d}/\mathrm{d}x, 5x^2+3x+4\right]$。(\textit{答案:}$10x+3$)
	\end{examplebox}
	\indent 算符的\textbf{平方}定义为算符和它本身的乘积:$\hat{B}^2 = \hat{B}\hat{B}$。我们来求求导算符的平方:
	\begin{equation*}
		\begin{aligned}
			\hat{D}^2f\left(x\right) & = \hat{D}\left(\hat{D}f\right) = \hat{D}f^{\prime} = f^{\prime \prime} \\
			\hat{D}^2& = \mathrm{d}^2/\mathrm{d}x^2
		\end{aligned}
	\end{equation*}
	再比如:取函数复共轭的算符的平方等于单位算符,因为取两次复共轭可以得到原始函数。算符$\hat{B}^n \: \left(n = 1,2,3\cdots\right)$定义为连续应用$n$次算符$\hat{B}$。\\
	\indent 事实表明,量子力学中出现的算符都是线性的。当且仅当算符$\hat{A}$具有以下两个性质时,它是\textbf{线性算符}:
	\begin{equation}
		\boxed{
			\hat{A}\left[f\left(x\right)+g\left(x\right)\right] = \hat{A}f\left(x\right)+\hat{A}g\left(x\right)
		}
		\label{eq:3.9}
	\end{equation}
	\begin{equation}
		\boxed{
			\hat{A}\left[cf\left(x\right)\right] = c\hat{A}f\left(x\right)
		}
		\label{eq:3.10}
	\end{equation}
	其中$f$和$g$是任意函数,$c$是任意常数(可以不必是实数)。线性算符的例子有$\hat{x}^2$、$\mathrm{d}/\mathrm{d}x$和$\mathrm{d}^2/\mathrm{d}x^2$。非线性算符有$\cos$和$\left(\quad\right)^2$,其中$\left(\quad\right)^2$是对作用的函数求平方。\\
	\begin{examplebox}
		\textbf{例题:}\\
		$\mathrm{d}/\mathrm{d}x$是线性算符吗?$\sqrt{\quad}$是线性算符吗?\\
		\\
		\indent 我们有
		\begin{equation*}
			\left(\mathrm{d}/\mathrm{d}x\right) \left[f\left(x\right)+g\left(x\right)\right] = \mathrm{d}f\mathrm{d}x+\mathrm{d}g+\mathrm{d}x=\left(\mathrm{d}/\mathrm{d}x\right)f\left(x\right)+\left(\mathrm{d}/\mathrm{d}x\right)g\left(x\right)
		\end{equation*}
		\begin{equation*}
			\left(\mathrm{d}/\mathrm{d}x\right)\left[cf\left(x\right)\right] = c \mathrm{d}f\left(x\right) /\mathrm{d}x
		\end{equation*}
		$\mathrm{d}/\mathrm{d}x$符合式(\ref{eq:3.9})和(\ref{eq:3.10}),是线性算符。然而,
		\begin{equation*}
			\sqrt{f\left(x\right)+g\left(x\right)} \neq \sqrt{f\left(x\right)} +\sqrt{g\left(x\right)}
		\end{equation*}
		所以$\sqrt{\quad}$不是线性算符。\\
		\\
		\textbf{练习:}算符$x^2\times$(乘以$x^2$)是线性算符吗?(\textit{答案:}是)
	\end{examplebox}
	\indent 线性算符运算中有用的算符有
	\begin{equation}
		\boxed{
			\left(\hat{A}+\hat{B}\right)\hat{C} = \hat{A}\hat{C}+\hat{B}\hat{C}
		}
		\label{eq:3.11}
	\end{equation}
	\begin{equation}
		\boxed{
			\hat{A}\left(\hat{B}+\hat{C}\right) = \hat{A}\hat{B}+\hat{A}\hat{C}
		}
		\label{eq:3.12}
	\end{equation}
	\begin{examplebox}
		\textbf{例题:}证明线性算符满足分配律(\ref{eq:3.11})。\\
		\\
		\indent 开始证明的好方法是先写下给出的内容和要证明的内容。已知$\hat{A}$、$\hat{B}$和$\hat{C}$是线性算符,求证$\left(\hat{A}+\hat{B}\right)\hat{C} = \hat{A}\hat{C}+\hat{B}\hat{C}$。\\
		为了证明算符$\left(\hat{A}+\hat{B}\right)\hat{C}$等于算符$\hat{A}\hat{C}+\hat{B}\hat{C}$,我们必须证明这两个算符作用到任意函数$f$上的结果是相等的。即
		\begin{equation*}
			\left[\left(\hat{A}+\hat{B}\right)\hat{C}\right]f = \left(\hat{A}\hat{C}+\hat{B}\hat{C}\right)f
		\end{equation*}
		我们从左边$\left[\left(\hat{A}+\hat{B}\right)\hat{C}\right]f$开始。这个表达式包含了算符$\hat{A}$与$\hat{B}$的乘积及算符$\hat{C}$。将算符乘积的定义(\ref{eq:3.3 definition of operators' product})中的$\hat{A}$用$\hat{A}+\hat{B}$替代,$\hat{B}$用$\hat{C}$替代,得出$\left[\left(\hat{A}+\hat{B}\right)\hat{C}\right]f = \left(\hat{A}+\hat{B}\right)\left(\hat{C}f\right)$。将$\hat{C}f$整体视为一个函数,使用式(\ref{eq:3.2 definition of operators' sum and differenct})关于两个算符$\hat{A}$与$\hat{B}$的和$\hat{A}+\hat{B}$的定义,得出$\left(\hat{A}+\hat{B}\right)\left(\hat{C}f\right) = \hat{A}\left(\hat{C}f\right)+\hat{B}\left(\hat{C}f\right)$。因此,
		\begin{equation*}
			\left[\left(\hat{A}+\hat{B}\right)\hat{C}\right]f = \left(\hat{A}+\hat{B}\right)\left(\hat{C}f\right) = \hat{A}\left(\hat{C}f\right)+\hat{B}\left(\hat{C}f\right)
		\end{equation*}
		使用式(\ref{eq:3.3 definition of operators' product})给出的算符积的定义,有$\hat{A}\left(\hat{C}f\right) = \hat{a}\hat{C}f$及$\hat{B}\left(\hat{C}f\right) = \hat{B}\hat{C}f$。则
		\begin{equation}
			\left[\left(\hat{A}+\hat{B}\right)\hat{C}\right]f = \hat{A}\hat{C}f+\hat{B}\hat{C}f
			\label{eq:3.13}
		\end{equation}
		将算符和的定义(\ref{eq:3.2 definition of operators' sum and differenct})中的$\hat{A}$用$\hat{A}\hat{C}$替代,$\hat{B}$用$\hat{B}\hat{C}$替代,得出$\left(\hat{A}\hat{C}+\hat{B}\hat{C}\right)f = \hat{A}\hat{C}f+\hat{B}\hat{C}f$,则式(\ref{eq:3.13})变成了
		\begin{equation*}
			\left[\left(\hat{A}+\hat{B}\right) \hat{C}\right]f = \left(\hat{A}\hat{C}+\hat{B}\hat{C}\right)f
		\end{equation*}
		这就是我们要证明的。因此,$\left(\hat{A}+\hat{B}\right)\hat{C} = \hat{A}\hat{C}+\hat{B}\hat{C}$。\\
		注意我们没有用到$\hat{A}$、$\hat{B}$和$\hat{C}$是线性算符这一条件,事实上,式(\ref{eq:3.11})对任意算符都成立。但式(\ref{eq:3.12})只在$\hat{A}$是线性算符时成立。(问题 3.17)
	\end{examplebox}
	\begin{examplebox}
		\textbf{例题:}求算符$\mathrm{d}/\mathrm{d}x+\hat{x}$的平方。\\
		\\
		为了求$\left(\mathrm{d}/\mathrm{d}x+\hat{x}\right)^2$,我们将其作用到任意一个函数$f\left(x\right)$上。令$\hat{D} \equiv \mathrm{d}/\mathrm{d}x$,我们有
		\begin{equation*}
			\begin{aligned}
				\left(\hat{D}+\hat{x}\right)^2f\left(x\right) & = \left(\hat{D}+\hat{x}\right)\left[\left(\hat{D}+\hat{x}\right)f\right] = \left(\hat{D}+\hat{x}\right)\left(f^{\prime}+xf\right) \\
				& = f^{\prime \prime}+f+xf^{\prime}+xf^{\prime}+x^2f=\left(\hat{D}^2+2\hat{x}\hat{D}+\hat{x}^2+1\right)f\left(x\right)
			\end{aligned}
		\end{equation*}
		\begin{equation*}
			\left(\hat{D}+\hat{x}\right)^2 = \hat{D}^2+2\hat{x}\hat{D}+\hat{x}^2+1
		\end{equation*}
		让我们只使用算符方程重复这一计算:
		\begin{equation*}
			\begin{aligned}
				\left(\hat{D}+\hat{x}\right)^2 & = \left(\hat{D}+\hat{x}\right)\left(\hat{D}+\hat{x}\right) = \hat{D}\left(\hat{D}+\hat{x}\right)+\hat{x}\left(\hat{D}+\hat{x}\right) \\
				& = \hat{D}^2+\hat{D}\hat{x}+\hat{x}\hat{D}+\hat{x}^2 = \hat{D}^2+\hat{x}\hat{D}+1+\hat{x}\hat{D}+\hat{x} \\
				& = \hat{D}^2+2\hat{x}\hat{D}+\hat{x}^2+1
			\end{aligned}
		\end{equation*}
		其中用到了式(\ref{eq:3.11})、(\ref{eq:3.12})和(\ref{eq:3.5}),并省略了 “乘以 $x$ ”算符上的小尖号。\textit{在彻底掌握算符之前,最安全的做法是在进行算符操作时,始终保持算符对任意函数 $f$ 进行操作,然后在最后删除 $f$。}
		
		
		
	\end{examplebox}
	
\section{本征函数和本征值}
An operator $\hat{A}$ satisfies:
\begin{equation}
	\hat{A} \psi = a \psi
\end{equation}
where $a$ is the eigenvalue.

\section{算符和量子力学}
Key operators in quantum mechanics:
\begin{align}
	\text{Position:} & \quad \hat{x} = x \\
	\text{Momentum:} & \quad \hat{p}_x = -i\hbar \pdv{x} \\
	\text{Hamiltonian:} & \quad \hat{H} = -\frac{\hbar^2}{2m} \nabla^2 + V(\mathbf{r})
\end{align}

\section{三维多粒子的薛定谔方程}

\section{三维盒子中的粒子}

\section{简并}

\section{平均值}

\section{波函数的约束条件}

\section*{总结}

\section*{习题}