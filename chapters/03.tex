% ===== CHAPTER 3 =====
\chapter{算符}
\section{算符}
	现在,我们以一种比以前更一般的方式来发展量子力学理论。首先,我们将单粒子一维定态薛定谔方程 (\ref{eq:1.19 time-independent Schrödinger Equation}) 写成以下形式
	\begin{equation}
		\left[-\frac{\hbar^2}{2m}\frac{\mathrm{d}^2\psi}{\mathrm{d}x^2}+V\left(x\right)\right]\psi\left(x\right) = E\psi\left(x\right)
		\label{eq:3.1}
	\end{equation}
	(\ref{eq:3.1}) 中括号内的实体是一个\textit{算符}(operator)。方程(\ref{eq:3.1})表明我们有一个能量算符,它对波函数起作用后,波函数又回来了,不过是乘以一个允许的能量值。因此,接下来我们讨论算符。\\
	\indent \textbf{算符}是一种将给定函数转换为另一个函数的规则。例如,让 $\hat{D}$ 算符来表示对关于$x$的函数求导。我们用一个字母上的小尖号来表示算符。给定一个可导函数$f\left(x\right)$,则将$\hat{D}$作用到$f\left(x\right)$上即可表示为$\hat{D}f\left(x\right) = f^{\prime}\left(x\right)$。例如,$\hat{D}\left(x^2+3\mathrm{e}^{2x}\right) = 2x+6\mathrm{e}^{2x}$。如果$\hat{3}$是将函数乘以 3 的算符,那么$\hat{3}\left(x^2+3\mathrm{e}^x\right) = 3x^2+9\mathrm{e}^x$。如果 tan 是取函数正切值的算符,那么将 tan 应用于函数$x^2+1$,我们得到了$\tan\left(x^2+1\right)$。如果算符$\hat{A}$将函数$f\left(x\right)$转换为另一个函数$g\left(x\right)$,我们将其写作$\hat{A}f\left(x\right) = g\left(x\right)$。\\
	\indent 我们将两个算符$\hat{A}$和$\hat{B}$的\textbf{和}与\textbf{差}定义为
	\begin{equation}
		\boxed{
			\left(\hat{A}+\hat{B}\right)f\left(x\right) \equiv \hat{A}f\left(x\right)+\hat{B}f\left(x\right)
		}
		\label{eq:3.2 definition of operators' sum and differenct}
	\end{equation}
	\begin{equation*}
		\left(\hat{A}-\hat{B}\right)f\left(x\right) \equiv \hat{A}f\left(x\right)-\hat{B}f\left(x\right)
	\end{equation*}
	例如,若$\hat{D} \equiv \mathrm{d}/\mathrm{d}x$,那么
	\begin{equation*}
		\left(\hat{D}+\hat{3}\right)\left(x^3-5\right) = \hat{D}\left(x^3-5\right)+ \hat{3}\left(x^3-5\right) = 3x^2+\left(3x^3-15\right) = 3x^3+3x^2-15
	\end{equation*}
	\indent 一个算符也可以涉及多个变量。例如,算符$\partial^2/\partial x^2+\partial^2/\partial y^2$具有如下性质:
	\begin{equation*}
		\left(\partial^2/\partial x^2+\partial^2/\partial y^2\right)g\left(x,y\right)= \partial^2 g/\partial x^2 + \partial^2 g/\partial y^2
	\end{equation*}
	\indent 两个算符的\textbf{积}定义为
	\begin{equation}
		\boxed{
			\hat{A}\hat{B}f\left(x\right) = \hat{A}\left[\hat{B}f\left(x\right)\right]
		}
		\label{eq:3.3 definition of operators' product}
	\end{equation}
	换句话说,我们先将乘积右边的算符作用到函数$f\left(x\right)$上,然后将左边的算符作用到得到的函数上。例如,$\hat{3}\hat{D}f\left(x\right) = \hat{3}\left[\hat{D}f\left(x\right)\right] = \hat{3}f^{\prime}\left(x\right) = 3f^{\prime}\left(x\right)$\\
	\indent 算符$\hat{A}\hat{B}$和$\hat{B}\hat{A}$可能不会有同样的效果。例如,考虑算符$\mathrm{d}/\mathrm{d}x$和$\hat{x}$($\hat{x}$表示将函数乘以$x$):
	\begin{equation}
		\hat{D}\hat{x}f\left(x\right)=\frac{\mathrm{d}}{\mathrm{d}x}\left[xf\left(x\right)\right] = f\left(x\right) + f^{\prime}\left(x\right) = \left(\hat{1}+\hat{x}\hat{D}\right)f\left(x\right)
		\label{eq:3.4}
	\end{equation}
	\begin{equation*}
		\hat{x}\hat{D}f\left(x\right) = \hat{x}\left[\frac{\mathrm{d}}{\mathrm{d}x}f\left(x\right)\right] = xf^{\prime}\left(x\right)
	\end{equation*}
	在这个例子中,$\hat{A}\hat{B}$和$\hat{B}\hat{A}$是不同的算符。\\
	\indent 我们可以建立如下的\textbf{算符代数}。若对任意的函数$f$,有$\hat{A}f = \hat{B}f$,则我们说算符$\hat{A}$和算符$\hat{B}$是\textbf{相等的}。相等的算符作用到给定的函数上时,会有相同的结果。例如,式(\ref{eq:3.4})表明
	\begin{equation}
		\hat{D}\hat{x} = 1 + \hat{x}\hat{D}
		\label{eq:3.5}
	\end{equation}
	算符$\hat{1}$(乘以1)称作\textbf{单位算符}。算符$\hat{0}$(乘以0)称作\textbf{空算符}。我们通常省略简单常数算符顶上的小尖号。我们还可以将算符从算符方程的一边转移到另一边(问题 3.7 )。因此,式(\ref{eq:3.5})与$\hat{D}\hat{x}-\hat{x}\hat{D}-1 = 0$等价,其中省略了空算符和单位算符上的小尖号。\\
	\indent 算符运算遵循结合律:
	\begin{equation}
		\hat{A}\left(\hat{B}\hat{C}\right) = \left(\hat{A}\hat{B}\right)\hat{C}
		\label{eq:3.6 law of multiplication for operators}
	\end{equation}
	式(\ref{eq:3.6 law of multiplication for operators})的证明见问题3.10。例如,令$\hat{A} = \mathrm{d}/\mathrm{d}x$,$\hat{B} = \hat{x}$,而$\hat{C} = 3$。使用式(\ref{eq:3.5}),我们有
	\begin{equation*}
		\begin{aligned}
			\left(\hat{A}\hat{B}\right) = \hat{D}\hat{x} = 1+ \hat{x}\hat{D}, \quad & \left[\left(\hat{A}\hat{B}\right)\hat{C}\right]f = \left(1+\hat{x}\hat{D}\right)3f = 3f+3xf^{\prime}\\
			\left(\hat{B}\hat{C}\right) = 3\hat{x},  \qquad \quad \qquad & \left[\hat{A}\left(\hat{B}\hat{C}\right)\right]f = \hat{D}\left(3xf\right) = 3f+3xf^{\prime}
		\end{aligned}
	\end{equation*}
	\indent 算符代数与普通代数的一个主要区别是:数的运算遵守乘法的交换律,但算符不一定。例如:若$a$和$b$都是数,那么有$ab=ba$;但是$\hat{A}\hat{B}$和$\hat{B}\hat{A}$不一定相等。我们定义算符$\hat{A}$和$\hat{B}$的\textbf{交换子}(commutator)$\left[\hat{A},\hat{B}\right]$为$\hat{A}\hat{B}-\hat{B}\hat{A}$:
	\begin{equation}
		\boxed{
			\left[\hat{A},\hat{B}\right] \equiv \hat{A}\hat{B} - \hat{B}\hat{A}
		}
		\label{eq:3.7 definition of commutator for two operators}
	\end{equation}
	如果有$\hat{A}\hat{B} = \hat{B}\hat{A}$,那么$\left[\hat{A},\hat{B}\right] = 0$,则我们说算符$\hat{A}$和$\hat{B}$是\textbf{可对易的}。如果$\hat{A}\hat{B} \neq \hat{B}\hat{A}$,那么$\hat{A}$和$\hat{B}$是不可对易的。注意$\left[\hat{A}, \hat{B}\right]f = \hat{A}\hat{B}f - \hat{B}\hat{A}f$。由于我们作用算符 3 和 $\mathrm{d}/\mathrm{d}x$ 的顺序没有区别,因此我们有
	\begin{equation*}
		\left[\hat{3},\frac{\mathrm{d}}{\mathrm{d}x}\right] = \hat{3}\frac{\mathrm{d}}{\mathrm{d}x} - \frac{\mathrm{d}}{\mathrm{d}x}\hat{3} = 0
	\end{equation*}
	\indent 从式(\ref{eq:3.5}),我们有
	\begin{equation}
		\left[\frac{\mathrm{d}}{\mathrm{d}x}, \hat{x}\right] = \hat{D}\hat{x}-\hat{x}\hat{D} = 1
		\label{eq:3.8}
	\end{equation}
	那么算符$\mathrm{d}/\mathrm{d}x$和$\hat{x}$是不可对易的。\\
	\begin{examplebox}
		\textbf{例题:}\\
		求$\left[z^3,\mathrm{d}/\mathrm{d}z\right]$。\\
		\\
		\indent 为了求出$\left[z^3,\mathrm{d}/\mathrm{d}z\right]$,我们需要将其作用到任意一个函数$g\left(x\right)$上。由式(\ref{eq:3.7 definition of commutator for two operators})交换子的定义及算符差和乘积的定义,我们有
		\begin{equation*}
			\begin{aligned}
				\left[z^3,\mathrm{d}/\mathrm{d}z\right]g = \left[z^3\left(\mathrm{d}/\mathrm{d}z\right) - \left(\mathrm{d}/\mathrm{d}z\right)z^3\right]g & = z^3\left(\mathrm{d}/\mathrm{d}z\right)g - \left(\mathrm{d}/\mathrm{d}z\right)\left(z^3g\right) \\
				& = z^3g^{\prime} - 3z^2g-z^3g^{\prime} = -3z^2g
			\end{aligned}
		\end{equation*}
		删去任意的函数$g$,我们有算符方程$\left[z^3,\mathrm{d}/\mathrm{d}z\right] = -3z^2$。\\
		\\
		\textbf{练习:}\\
		求$\left[\mathrm{d}/\mathrm{d}x, 5x^2+3x+4\right]$。(\textit{答案:}$10x+3$)
	\end{examplebox}
	\indent 算符的\textbf{平方}定义为算符和它本身的乘积:$\hat{B}^2 = \hat{B}\hat{B}$。我们来求求导算符的平方:
	\begin{equation*}
		\begin{aligned}
			\hat{D}^2f\left(x\right) & = \hat{D}\left(\hat{D}f\right) = \hat{D}f^{\prime} = f^{\prime \prime} \\
			\hat{D}^2& = \mathrm{d}^2/\mathrm{d}x^2
		\end{aligned}
	\end{equation*}
	再比如:取函数复共轭的算符的平方等于单位算符,因为取两次复共轭可以得到原始函数。算符$\hat{B}^n \: \left(n = 1,2,3\cdots\right)$定义为连续应用$n$次算符$\hat{B}$。\\
	\indent 事实表明,量子力学中出现的算符都是线性的。当且仅当算符$\hat{A}$具有以下两个性质时,它是\textbf{线性算符}:
	\begin{equation}
		\boxed{
			\hat{A}\left[f\left(x\right)+g\left(x\right)\right] = \hat{A}f\left(x\right)+\hat{A}g\left(x\right)
		}
		\label{eq:3.9}
	\end{equation}
	\begin{equation}
		\boxed{
			\hat{A}\left[cf\left(x\right)\right] = c\hat{A}f\left(x\right)
		}
		\label{eq:3.10}
	\end{equation}
	其中$f$和$g$是任意函数,$c$是任意常数(可以不必是实数)。线性算符的例子有$\hat{x}^2$、$\mathrm{d}/\mathrm{d}x$和$\mathrm{d}^2/\mathrm{d}x^2$。非线性算符有$\cos$和$\left(\quad\right)^2$,其中$\left(\quad\right)^2$是对作用的函数求平方。\\
	\begin{examplebox}
		\textbf{例题:}\\
		$\mathrm{d}/\mathrm{d}x$是线性算符吗?$\sqrt{\quad}$是线性算符吗?\\
		\\
		\indent 我们有
		\begin{equation*}
			\left(\mathrm{d}/\mathrm{d}x\right) \left[f\left(x\right)+g\left(x\right)\right] = \mathrm{d}f\mathrm{d}x+\mathrm{d}g+\mathrm{d}x=\left(\mathrm{d}/\mathrm{d}x\right)f\left(x\right)+\left(\mathrm{d}/\mathrm{d}x\right)g\left(x\right)
		\end{equation*}
		\begin{equation*}
			\left(\mathrm{d}/\mathrm{d}x\right)\left[cf\left(x\right)\right] = c \mathrm{d}f\left(x\right) /\mathrm{d}x
		\end{equation*}
		$\mathrm{d}/\mathrm{d}x$符合式(\ref{eq:3.9})和(\ref{eq:3.10}),是线性算符。然而,
		\begin{equation*}
			\sqrt{f\left(x\right)+g\left(x\right)} \neq \sqrt{f\left(x\right)} +\sqrt{g\left(x\right)}
		\end{equation*}
		所以$\sqrt{\quad}$不是线性算符。\\
		\\
		\textbf{练习:}算符$x^2\times$(乘以$x^2$)是线性算符吗?(\textit{答案:}是)
	\end{examplebox}
	\indent 线性算符运算中有用的算符有
	\begin{equation}
		\boxed{
			\left(\hat{A}+\hat{B}\right)\hat{C} = \hat{A}\hat{C}+\hat{B}\hat{C}
		}
		\label{eq:3.11}
	\end{equation}
	\begin{equation}
		\boxed{
			\hat{A}\left(\hat{B}+\hat{C}\right) = \hat{A}\hat{B}+\hat{A}\hat{C}
		}
		\label{eq:3.12}
	\end{equation}
	\begin{examplebox}
		\textbf{例题:}证明线性算符满足分配律(\ref{eq:3.11})。\\
		\\
		\indent 开始证明的好方法是先写下给出的内容和要证明的内容。已知$\hat{A}$、$\hat{B}$和$\hat{C}$是线性算符,求证$\left(\hat{A}+\hat{B}\right)\hat{C} = \hat{A}\hat{C}+\hat{B}\hat{C}$。\\
		为了证明算符$\left(\hat{A}+\hat{B}\right)\hat{C}$等于算符$\hat{A}\hat{C}+\hat{B}\hat{C}$,我们必须证明这两个算符作用到任意函数$f$上的结果是相等的。即
		\begin{equation*}
			\left[\left(\hat{A}+\hat{B}\right)\hat{C}\right]f = \left(\hat{A}\hat{C}+\hat{B}\hat{C}\right)f
		\end{equation*}
		我们从左边$\left[\left(\hat{A}+\hat{B}\right)\hat{C}\right]f$开始。这个表达式包含了算符$\hat{A}$与$\hat{B}$的乘积及算符$\hat{C}$。将算符乘积的定义(\ref{eq:3.3 definition of operators' product})中的$\hat{A}$用$\hat{A}+\hat{B}$替代,$\hat{B}$用$\hat{C}$替代,得出$\left[\left(\hat{A}+\hat{B}\right)\hat{C}\right]f = \left(\hat{A}+\hat{B}\right)\left(\hat{C}f\right)$。将$\hat{C}f$整体视为一个函数,使用式(\ref{eq:3.2 definition of operators' sum and differenct})关于两个算符$\hat{A}$与$\hat{B}$的和$\hat{A}+\hat{B}$的定义,得出$\left(\hat{A}+\hat{B}\right)\left(\hat{C}f\right) = \hat{A}\left(\hat{C}f\right)+\hat{B}\left(\hat{C}f\right)$。因此,
		\begin{equation*}
			\left[\left(\hat{A}+\hat{B}\right)\hat{C}\right]f = \left(\hat{A}+\hat{B}\right)\left(\hat{C}f\right) = \hat{A}\left(\hat{C}f\right)+\hat{B}\left(\hat{C}f\right)
		\end{equation*}
		使用式(\ref{eq:3.3 definition of operators' product})给出的算符积的定义,有$\hat{A}\left(\hat{C}f\right) = \hat{a}\hat{C}f$及$\hat{B}\left(\hat{C}f\right) = \hat{B}\hat{C}f$。则
		\begin{equation}
			\left[\left(\hat{A}+\hat{B}\right)\hat{C}\right]f = \hat{A}\hat{C}f+\hat{B}\hat{C}f
			\label{eq:3.13}
		\end{equation}
		将算符和的定义(\ref{eq:3.2 definition of operators' sum and differenct})中的$\hat{A}$用$\hat{A}\hat{C}$替代,$\hat{B}$用$\hat{B}\hat{C}$替代,得出$\left(\hat{A}\hat{C}+\hat{B}\hat{C}\right)f = \hat{A}\hat{C}f+\hat{B}\hat{C}f$,则式(\ref{eq:3.13})变成了
		\begin{equation*}
			\left[\left(\hat{A}+\hat{B}\right) \hat{C}\right]f = \left(\hat{A}\hat{C}+\hat{B}\hat{C}\right)f
		\end{equation*}
		这就是我们要证明的。因此,$\left(\hat{A}+\hat{B}\right)\hat{C} = \hat{A}\hat{C}+\hat{B}\hat{C}$。\\
		注意我们没有用到$\hat{A}$、$\hat{B}$和$\hat{C}$是线性算符这一条件,事实上,式(\ref{eq:3.11})对任意算符都成立。但式(\ref{eq:3.12})只在$\hat{A}$是线性算符时成立。(问题 3.17)
	\end{examplebox}
	\begin{examplebox}
		\textbf{例题:}求算符$\mathrm{d}/\mathrm{d}x+\hat{x}$的平方。\\
		\\
		为了求$\left(\mathrm{d}/\mathrm{d}x+\hat{x}\right)^2$,我们将其作用到任意一个函数$f\left(x\right)$上。令$\hat{D} \equiv \mathrm{d}/\mathrm{d}x$,我们有
		\begin{equation*}
			\begin{aligned}
				\left(\hat{D}+\hat{x}\right)^2f\left(x\right) & = \left(\hat{D}+\hat{x}\right)\left[\left(\hat{D}+\hat{x}\right)f\right] = \left(\hat{D}+\hat{x}\right)\left(f^{\prime}+xf\right) \\
				& = f^{\prime \prime}+f+xf^{\prime}+xf^{\prime}+x^2f=\left(\hat{D}^2+2\hat{x}\hat{D}+\hat{x}^2+1\right)f\left(x\right)
			\end{aligned}
		\end{equation*}
		\begin{equation*}
			\left(\hat{D}+\hat{x}\right)^2 = \hat{D}^2+2\hat{x}\hat{D}+\hat{x}^2+1
		\end{equation*}
		让我们只使用算符方程重复这一计算:
		\begin{equation*}
			\begin{aligned}
				\left(\hat{D}+\hat{x}\right)^2 & = \left(\hat{D}+\hat{x}\right)\left(\hat{D}+\hat{x}\right) = \hat{D}\left(\hat{D}+\hat{x}\right)+\hat{x}\left(\hat{D}+\hat{x}\right) \\
				& = \hat{D}^2+\hat{D}\hat{x}+\hat{x}\hat{D}+\hat{x}^2 = \hat{D}^2+\hat{x}\hat{D}+1+\hat{x}\hat{D}+\hat{x} \\
				& = \hat{D}^2+2\hat{x}\hat{D}+\hat{x}^2+1
			\end{aligned}
		\end{equation*}
		其中用到了式(\ref{eq:3.11})、(\ref{eq:3.12})和(\ref{eq:3.5}),并省略了 “乘以 $x$ ”算符上的小尖号。\textit{在彻底掌握算符之前,最安全的做法是在进行算符操作时,始终保持算符对任意函数 $f$ 进行操作,然后在最后删除 $f$。}\\
		\\
		\textbf{练习:}\\
		求$\left(\mathrm{d}^2/\mathrm{d}x^2\right)^2$。(\textit{答案:}$\mathrm{d}^4/\mathrm{d}x^4+2x\mathrm{d}^2/\mathrm{d}x^2+2\mathrm{d}/\mathrm{d}x+x^2$)
	\end{examplebox}
	
\section{本征函数和本征值}
	假设某线性算符$\hat{A}$作用在某个函数$f\left(x\right)$上的结果是对$f\left(x\right)$乘上一个常数$k$。我们说$f\left(x\right)$是$\hat{A}$的一个\textbf{本征函数}(eigenfunction),$k$是\textbf{本征值}(eigenvalue)。作为定义的一部分,我们将要求本征函数$f\left(x\right)$不等于零。我们的意思是说,尽管 $f\left(x\right)$ 可能在不同点上消失,但它并非处处为零。我们有
	\begin{equation}
		\boxed{
			\hat{A}f\left(x\right) = kf\left(x\right)
		}
		\label{eq:3.14 definition of eigenfunctions and eigenvalues}
	\end{equation}
	例如,函数$\mathrm{e}^{2x}$是算符$\mathrm{d}/\mathrm{d}x$的一个本征函数,其本征值为2:
	\begin{equation*}
		\left(\mathrm{d}/\mathrm{d}x\right)\mathrm{e}^{2x} = 2 \mathrm{e}^{2x}
	\end{equation*}
	然而$\sin 2x$不是$\mathrm{d}/\mathrm{d}x$的本征函数,因为$\left(\mathrm{d}/\mathrm{d}x\right)\left(\sin 2x\right) = 2 \cos 2x$,不等于一个常数乘以$\sin 2x$。
	\begin{examplebox}
		\textbf{例题:}如果函数$f\left(x\right)$是线性算符$\hat{A}$的一个本征函数,$c$是任意常数。求证:$cf\left(x\right)$也是算符$\hat{A}$的本征函数,并与$f\left(x\right)$有相同的本征值。\\
		\\
		了解如何进行证明的一个好方法是按以下步骤进行:\\
		1. 写下给定的信息,并将这些信息从文字转化为等式。\\
		2. 用一个或多个方程的形式写下要证明的内容。\\
		3. (a) 处理步骤 1 中的给定方程,将其转化为步骤 2 中的所需方程。 (b) 或者,从我们要证明的方程的一边开始,使用步骤 1 中的给定方程来处理这一边,直到它转化为要证明的方程的另一边。\\
		\\
		我们有三个条件:$f\left(x\right)$是算符$\hat{A}$的本征函数;$\hat{A}$是线性算符;$c$是常数。将这些表述转化为方程 [见方程(\ref{eq:3.14 definition of eigenfunctions and eigenvalues})、(\ref{eq:3.9})和(\ref{eq:3.10})],我们有
		\begin{equation}
			\hat{A} f = kf
			\label{eq:3.15}
		\end{equation}
		\begin{equation}
			\hat{A}\left(f+g\right) = \hat{A}f+\hat{A}g,\quad \text{及}\: \hat{A}\left(bf\right) = b\hat{A}f
			\label{eq:3.16}
		\end{equation}
		\begin{equation*}
			c = a \: \left(\text{常数}\right)
		\end{equation*}
		其中$k$和$b$是常数,$f$和$g$是函数。\\
		我们想要证明$cf\left(x\right)$也是算符$\hat{A}$的本征函数且有相同的本征值,将其写成方程,即
		\begin{equation*}
			\hat{A}\left(cf\right) = k\left(cf\right)
		\end{equation*}
		使用策略(3)的(b),我们从待证明方程的左侧$\hat{A}\left(cf\right)$出发,尝试证明它等于$k\left(cf\right)$。由线性算符定义(\ref{eq:3.16})的第二个方程,我们有$\hat{A}\left(cf\right) = c\hat{A}f$。使用本征方程(\ref{eq:3.15}),有$c\hat{A}f = ckf$。因此,
		\begin{equation*}
			\hat{A}\left(cf\right) = c\hat{A}f = ckf= k\left(cf\right)
		\end{equation*}
		得证。
	\end{examplebox}
	\begin{examplebox}
		\textbf{例题:}(a) 求算符$\mathrm{d}/\mathrm{d}x$的本征函数及对应的本征值;(b)如果我们假设定解条件为本征函数在$x \to \pm \infty$时为有限值,求对应的本征值。\\
		\\
		(a)将算符$\hat{A} = \mathrm{d}/\mathrm{d}x$带入方程(\ref{eq:3.14 definition of eigenfunctions and eigenvalues}),有
		\begin{equation}
			\frac{\mathrm{d}f\left(x\right)}{\mathrm{d}x} = kf\left(x\right)
			\label{eq:3.17}
		\end{equation}
		\begin{equation*}
			\frac{1}{f}\mathrm{d}f = k \mathrm{d}x
		\end{equation*}
		求积分,有
		\begin{equation*}
			\begin{aligned}
				\ln f & = kx + \text{常数} \\
				f & = \mathrm{e}^{\text{常数}}\mathrm{e}^{kx} \\
			\end{aligned}
		\end{equation*}
		\begin{equation}
			f  = c\mathrm{e}^{kx}
			\label{eq:3.18}
		\end{equation}
		算符$\mathrm{d}/\mathrm{d}x$的本征函数由(\ref{eq:3.18}),本征值为$k$,它可以是任何数字,而 (\ref{eq:3.17}) 仍然成立。本征函数包含一个任意的常数 $c$。这对每个线性算符的特征函数都是正确的,正如前面的例子所证明的。(\ref{eq:3.18}) 中 $k$ 的每个不同值都会产生不同的特征函数。然而,$k$ 值相同而 $c$ 值不同的特征函数并不是相互独立的。\\
		(b)由于$k$可以是复数,我们可以写作$k = a+\mathrm{i}b$,其中$a$和$b$是实数。我们有$f\left(x\right) = c\mathrm{e}^{ax}\mathrm{e}^{\mathrm{i}bx}$。若$a>0$,则当$x$趋于无穷时因子$\mathrm{e}^{ax}$也趋于无穷。若$a<0$,则$x \to - \infty$时$\mathrm{e}^{ax} \to \infty$。因此,这两个定解条件要求$a=0$,所以本征值$k=\mathrm{i}b$,其中$b$是实数。
	\end{examplebox}
	\indent 在第3.1节的第一个例题中,我们知道对任意的函数$g$,有$\left[z^3,\mathrm{d}/\mathrm{d}z\right]g\left(z\right) = -3z^2g\left(z\right)$,所以$\left[z^3,\mathrm{d}/\mathrm{d}z\right] = -3z^2$。相反地,本征方程$\hat{A}f\left(x\right) = kf\left(x\right)$[式(\ref{eq:3.14 definition of eigenfunctions and eigenvalues})]却不一定对任意函数$f\left(x\right)$成立,从这个方程中我们并不能得出$\hat{A}=k$。因此,结论$\left(\mathrm{d}/\mathrm{d}x\right)\mathrm{e}^{2x} = 2\mathrm{e}^{2x}$并不意味着$\mathrm{d}/\mathrm{d}x$与乘以2相等。
	
\section{算符和量子力学}
	我们现在考虑算符和量子力学之间的关系。将式 (\ref{eq:3.1}) 与 (\ref{eq:3.14 definition of eigenfunctions and eigenvalues}) 比较,我们可以发现薛定谔方程是一个本征值问题。能量 $E$ 的值就是本征值。本征函数就是定态薛定谔方程$\psi$。所需的本征函数和本征值的算符是$-\left(\hbar^2/2m\right)\mathrm{d}^2/\mathrm{d}x^2+V\left(x\right)$。这个算符称为系统的\textbf{哈密顿算符}(Hamiltonian operator)。\\
	\indent 威廉·罗文·哈密顿爵士(Sir William Rowan Hamilton)(1805-1865)设计了牛顿运动方程的另一种形式,其中涉及一个函数 $H$,即系统的\textbf{哈密顿函数}(Hamiltonian function)。对于势能仅是坐标函数的系统,总能量随时间保持不变;也就是说,$E$ 是守恒的。我们将只讨论这种保守系统。对于保守系统,经典力学哈密顿函数可以简单地用坐标和共轭动量来表示总能量。对于笛卡尔坐标 $x$、$y$、$z$,\textbf{共轭动量}是线性动量在 $x$、$y$、$z$ 方向上的分量$p_x$、$p_y$、$p_z$:
	\begin{equation}
		\boxed{
			p_x \equiv mv_x, \quad p_y \equiv mv_y, \quad p_z \equiv mv_z
		}
		\label{eq:3.19}
	\end{equation}
	其中$v_x$、$v_y$、$v_z$分别是粒子的速度矢量在$x$、$y$、$z$方向上的分量。\\
	\indent 让我们找出质量为 $m$ 的粒子在一维中运动并受到势能 $V\left(x\right)$ 作用的经典力学哈密顿函数。哈密顿函数等于系统的能量,而能量包括动能和势能。然而,我们熟悉的动能形式$\frac{1}{2}mv_x^2$并不适用,因为我们必须将哈密顿表示为坐标和动量的函数,而不是速度的函数。由于$v_x = p_x/m$,我们想要的动能表示形式为$p_x^2/2m$。那么哈密顿函数为
	\begin{equation}
		H = \frac{p_x^2}{2m}+V\left(x\right)
		\label{eq:3.20}
	\end{equation}
	\indent 定态薛定谔方程(\ref{eq:3.1})表明:与哈密顿函数(\ref{eq:3.20})相对应,我们有一个量子力学算符
	\begin{equation*}
		-\frac{\hbar^2}{2m}\frac{\mathrm{d}^2}{\mathrm{d}x^2}+V\left(x\right)
	\end{equation*}
	其本征值为系统能量的允许值。经典力学中的物理量与量子力学中的算符之间的这种对应关系是普遍的。量子力学的一个基本假设是,\textit{每一个物理量(例如能量、$x$ 坐标、动量)都有一个对应的量子力学算符。}我们进一步假设,将 $B$ 的经典力学表达式写成笛卡尔坐标和相应动量的函数,然后进行如下替换,就能找到与属性 $B$ 相对应的算符。每个笛卡尔坐标 $q$ 都用与该坐标相乘的算符来代替:
	\begin{equation*}
		\hat{q} = q \times
	\end{equation*}
	线性动量 $p_q$ 的每个笛卡尔分量都由以下算符代替:
	\begin{equation*}
		\hat{p}_q = \frac{\hbar}{\mathrm{i}}\frac{\partial}{\partial q} = -\mathrm{i}\hbar\frac{\partial}{\partial q}
	\end{equation*}	
	其中$\mathrm{i}$是虚数单位,$\partial / \partial q$是$p$对坐标 $q$ 的偏导数算符。\\
	\indent 考虑几个例子。与 $x$ 坐标相对应的算符是 $x$ 的乘法运算:
	\begin{equation}
		\boxed{
			\hat{x} = x \times
		}
		\label{eq:3.21}
	\end{equation}
	同样地,
	\begin{equation}
		\boxed{
			\hat{y} = y \times, \quad \hat{z} = z \times
		}
		\label{eq:3.22}
	\end{equation}
	线性动量各分量的算符为
	\begin{equation}
		\boxed{
			\hat{p}_x = \frac{\hbar}{\mathrm{i}}\frac{\partial}{\partial x}, \quad \hat{p}_y = \frac{\hbar}{\mathrm{i}}\frac{\partial}{\partial y}, \quad \hat{p}_z = \frac{\hbar}{\mathrm{i}}\frac{\partial}{\partial z}
		}
		\label{eq:3.23}
	\end{equation}
	与$p_x^2$相对应的算符为
	\begin{equation}
		\hat{p}_x^2 = \left(\frac{\hbar}{\mathrm{i}} \frac{\partial}{\partial x}\right)^2 = \frac{\hbar}{\mathrm{i}} \frac{\partial}{\partial x}\frac{\hbar}{\mathrm{i}} \frac{\partial}{\partial x} = -\hbar^2 \frac{\partial^2}{\partial x^2}
		\label{eq:3.24}
	\end{equation}
	$p_y$与$p_z$的表示类似。\\
	\indent 一维的势能算符和动能算符是什么?设一个系统的势能函数为$V\left(x\right) = ax^2$,其中$a$是常数。将$x$用$x \times$替代,我们可以看到,势能算符只是乘以 $ax^2$。即,$\hat{V}\left(x\right) = ax^2 \times$。一般来说,对于任何势能函数
	\begin{equation}
		\boxed{
			\hat{V}\left(x\right) = V\left(x\right) \times 
		}
		\label{eq:3.25 definition of potential energy operator}
	\end{equation}
	经典力学中动能$T$的表达式(\ref{eq:3.20})为
	\begin{equation}
		\boxed{
			T = p_x^2 / 2m
		}
		\label{eq:3.26}
	\end{equation}
	将$p_x$用对应的算符(\ref{eq:3.23})代替,我们有
	\begin{equation}
		\hat{T} = -\frac{\hbar^2}{2m}\frac{\partial^2}{\partial x^2} = -\frac{\hbar^2}{2m}\frac{\mathrm{d}^2}{\mathrm{d}x^2}
	\end{equation}
	其中用到了式(\ref{eq:3.24}),偏导数就变成了单变量的常导数。经典力学的哈密顿量(\ref{eq:3.20})为
	\begin{equation}
		H = T + V = p_x^2/2m+V\left(x\right)
		\label{eq:3.28}
	\end{equation}
	相应的量子力学哈密顿(或能量)算符为
	\begin{equation}
		\boxed{
			\hat{H} = \hat{T} + \hat{V} = -\frac{\hbar^2}{2m}\frac{\mathrm{d}^2}{\mathrm{d}x^2}+V\left(x\right)
		}
		\label{eq:3.29}
	\end{equation}
	与薛定谔方程 (\ref{eq:3.1}) 中的算符一致。请注意,所有
	这些算符都是线性的。\\
	\indent 量子力学算符如何与系统对应的属性相关联?每个这样的算符都有自己的一组本征函数和本征值。令$\hat{B}$是与物理量$B$相对应的量子力学算符,令$f_i$和$b_i$分别代表算符$\hat{B}$的本征函数和本征值。带入式(\ref{eq:3.14 definition of eigenfunctions and eigenvalues}),我们有
	\begin{equation}
		\hat{B}f_i = b_if_i, \quad i = 1,2,3\cdots
		\label{eq:3.30}
	\end{equation}
	算符$\hat{B}$有许多本征函数和本征值,用下标$i$相区分。$\hat{B}$通常是线性算符,且式(\ref{eq:3.30})是微分方程,对应的解可以给出本征函数和本征值。量子力学假设(无论系统的状态函数是什么)\textit{对属性 $B$ 的测量必须得到算符 $\hat{B}$ 的一个本征值 $b_i$。}例如,一个系统的能量只能通过能量(哈密顿)算符 $\hat{H}$ 的本征值求出。使用$\psi_i$来表示$\hat{H}$的本征函数,我们有本征方程(\ref{eq:3.30})
	\begin{equation}
		\boxed{
			\hat{H}\psi_i = E_i\psi_i
		}
		\label{eq:3.31}
	\end{equation}
	将式(\ref{eq:3.29})带入式(\ref{eq:3.31}),我们可以得到:对于单粒子的一维系统,有
	\begin{equation}
		\left[-\frac{\hbar^2}{2m}\frac{\mathrm{d}^2}{\mathrm{d}x^2}+V\left(x\right)\right]\psi_i = 
		E_i\psi_i
		\label{eq:3.32}
	\end{equation}
	就是式(\ref{eq:3.1})的定态薛定谔方程。因此,我们关于算符的假设与我们之前的工作是一致的。稍后,我们将进一步证明动量算符 (\ref{eq:3.23}) 的选择是正确的,因为在向经典力学的极限过渡中,这一选择会导出$p_x = m\left(\mathrm{d}x / \mathrm{d}t\right)$,这是理所应当的。(见问题 7.59)\\
	\indent 在第一章中,我们假设量子力学系统的状态是由一个状态函数$\Psi$$\left(x,t\right)$指定的,这个状态函数包含了我们可能知道的系统所有的信息。$\Psi$是如何告诉我们关于性质$B$的信息的?我们假设:\textit{若$\Psi$是$\hat{B}$的本征函数,本征值为$b_k$,则对性质$B$的测量可得到值为$b_k$的结果。}例如,考虑系统的能量。能量算符的本征函数为定态薛定谔方程(\ref{eq:3.32})的解$\psi\left(x\right)$。假设系统处于束缚态,有态函数(\ref{eq:1.20})
	\begin{equation}
		\Psi\left(x,t\right) = \mathrm{e}^{-\mathrm{i}Et/\hbar}\psi\left(x\right)
		\label{eq:3.33}
	\end{equation}
	$\Psi\left(x,t\right)$是能量算符$\hat{H}$的本征函数吗?我们有
	\begin{equation*}
		\hat{H}\Psi\left(x,t\right) = \hat{H}\mathrm{e}^{-\mathrm{i}Et/\hbar}\psi\left(x\right)
	\end{equation*}
	$\hat{H}$ 不包含对时间的导数,因此不会影响指数因子$\mathrm{e}^{-\mathrm{i}Et/\hbar}$,我们有
	\begin{equation*}
		\hat{H}\Psi\left(x,t\right) = \mathrm{e}^{-\mathrm{i}Et/\hbar}\hat{H}\psi\left(x\right) = E \mathrm{e}^{-\mathrm{i}Et/\hbar} \psi\left(x\right) = E \Psi\left(x,t\right)
	\end{equation*}
	\begin{equation}
		\hat{H}\Psi = E\Psi
		\label{eq:3.34}
	\end{equation}
	其中,我们用到了式(\ref{eq:3.31})。因此,对于束缚态系统,$\Psi\left(x,t\right)$是$\hat{H}$的本征函数,我们在测量能量时肯定会得到值为$E$的结果。\\
	\indent 考虑另一个动量的例子。算符$\hat{p}_x$的本征函数$g$由以下方程求出:
	\begin{equation*}
		\hat{p}_x g =kg
	\end{equation*}
	\begin{equation}
		\frac{\hbar}{\mathrm{i}} \frac{\mathrm{d}g}{\mathrm{d}x} = kg
		\label{eq:3.35}
	\end{equation}
	我们有(问题3.29)
	\begin{equation}
		g = A\mathrm{e}^{\mathrm{i}kx/\hbar}
		\label{eq:3.36}
	\end{equation}
	其中$A$是任意常数。当$\left|x\right|$充分大时,为了确保$g$不趋于无穷,本征值$k$必须是实数。因此,$\hat{p}_x$算符的本征值一定为实数
	\begin{equation}
		-\infty < k < \infty
		\label{eq:3.37}
	\end{equation}
	这是很合理的。任何对$p_x$的测量都会有符合式(\ref{eq:3.37})的$\hat{p}_x$的本征值。(\ref{eq:3.36}) 中每一个不同的 $k$ 值都给出了不同的本征函数$g$。物理量动量的算符涉及虚数$\mathrm{i}$,这似乎令人惊讶。事实上,$\mathrm{i}$的出现确保了本征值$k$一定为实数。回忆算符$\mathrm{d}/\mathrm{d}x$的本征值为虚数(第3.2节)。比较自由粒子波函数 (\ref{eq:2.30}) 和 $\hat{p}_x$ 的本征函数 (\ref{eq:3.36}),我们注意到以下物理解释: (\ref{eq:2.30}) 中的第一项相当于正动量,代表 $+x$ 方向的运动;(\ref{eq:2.30}) 中的第二项相当于负动量,代表 $-x$ 方向的运动。\\
	\indent 现在来考虑一个盒子中粒子的动量。在一维盒子中处于静止状态的粒子的状态函数为[式 (\ref{eq:3.33}) , (\ref{eq:2.20 energy of one-dimensional box}) , 和 (\ref{eq:2.23 stationary state wave function for the particle in a box}) ]。
	\begin{equation}
		\Psi\left(x,t\right) = \mathrm{e}^{-\mathrm{i}Et/\hbar}\left(\frac{2}{l}\right)^{1/2}\sin\left(\frac{n\pi x}{l}\right)
		\label{eq:3.38}
	\end{equation}
	其中$E = n^2h^2/8ml^2$。粒子的$p_x$值是确定的吗?即$\Psi\left(x,t\right)$是算符$\hat{p}_x$的本征函数吗?注意$\hat{p}_x$的本征函数,我们可以看到:没有任何实常数$k$的数值可以使(\ref{eq:3.36})中的指数函数变成正弦函数,如(\ref{eq:3.38})中的正弦函数。因此$\Psi$不是$\hat{p}_x$的本征函数。我们可以直接验证这一点;我们有
	\begin{equation*}
		\hat{p}_x\Psi = \frac{\hbar}{\mathrm{i}}\frac{\partial}{\partial x}\mathrm{e}^{-\mathrm{i}Et/\hbar}\left(\frac{2}{l}\right)^{1/2}\sin\left(\frac{n\pi x}{l}\right) = \frac{n\pi \hbar}{\mathrm{i}l}\mathrm{e}^{-\mathrm{i}Et/\hbar}\left(\frac{2}{l}\right)^{1/2}\cos\left(\frac{n\pi x}{l}\right)
	\end{equation*}
	由于$\hat{p}_x\Psi \neq \text{常数} \cdot \Psi$,态函数$\Psi$不是算符$\hat{p}_x$的本征函数。\\
	\indent 注意:系统的态函数$\Psi$不一定是与系统物理量$B$相对应的算符$\hat{B}$的本征函数$f_i$。因此,箱中粒子束缚态波函数不是$\hat{p}_x$的本征函数。尽管如此,当我们对箱中粒子束缚态的动量值进行观测时,我们也必须得到$\hat{p}_x$的一个本征值(\ref{eq:3.37})。\\
	\indent 箱中粒子束缚态波函数是$\hat{p}_x$的本征函数吗?由(\ref{eq:3.24}),我们有
	\begin{equation*}
		\hat{p}_x^2\Psi = -\hbar^2\frac{\partial^2}{\partial x^2}\mathrm{e}^{-\mathrm{i}Et/\hbar}\left(\frac{2}{l}\right)^{1/2}\sin\left(\frac{n\pi x}{l}\right) = \frac{n^2\pi^2 \hbar^2}{l^2}\mathrm{e}^{-\mathrm{i}Et/\hbar}\left(\frac{2}{l}\right)^{1/2}\sin\left(\frac{n\pi x}{l}\right)
	\end{equation*}
	\begin{equation}
		\hat{p}_x^2\Psi = \frac{n^2h^2}{4l^2}\Psi
		\label{eq:3.39}
	\end{equation}
	因此,对处于量子数为$n$的束缚态粒子,对$\hat{p}_x^2$的观测始终得到值为$n^2h^2/4l^2$的结果。这一点不足为奇:盒子中的势能为零,哈密顿算符为
	\begin{equation*}
		\hat{H} = \hat{T}+ \hat{V} = \hat{T} = \hat{p}_x^2/2m
	\end{equation*}
	则有式(\ref{eq:3.34}):
	\begin{equation*}
		\hat{H}\Psi = E\Psi = \frac{\hat{p}_x^2}{2m}\Psi
	\end{equation*}
	\begin{equation}
		\hat{p}_x^2\Psi = 2mE\Psi = 2m\frac{n^2h^2}{8ml^2}\Psi = \frac{n^2h^2}{4l^2}\Psi
		\label{eq:3.40}
	\end{equation}
	与 (\ref{eq:3.39}) 一致。$p_x$ 的唯一可能取值是
	\begin{equation}
		p_x^2 = n^2h^2 / 4l^2
		\label{eq:3.41}
	\end{equation}
	\begin{quote}
		\small % 字号小一号(可选 \footnotesize 更小)
		\noindent % 取消首行缩进(可选)
		等式 (\ref{eq:3.41}) 表明,测量 $p_x$ 必然会得到以下两个值中的一个,即$p_x = \pm \frac{1}{2}nh/l$,对应于粒子在盒中向右或向左移动。这种似是而非的说法并不准确。使用第 7 章的方法进行的分析表明:测量值很有可能接近以下两个值之一:$p_x = \pm \frac{1}{2}nh/l$。但任何符合 (\ref{eq:3.37}) 的值都可以通过测量盒子中粒子的 $p_x$ 得出;见问题 7.41。
	\end{quote}
	\indent 我们假设对性质$B$的观测一定会给出$\hat{B}$的其中一个本征值。如果态函数$\Psi$恰好是$B$的本征函数,本征值为$b$,当我们观测$B$时就一定会得到$b$。然而,假设$\Psi$不是$\hat{B}$的本征函数,会发生什么呢?我们仍然断言:当我们测量$B$时,会得到$\hat{B}$ 的一个本征值,但我们无法预测会得到哪个本征值。我们将在第 7 章中看到,得到$\hat{B}$ 的各种本征值的概率是可以预测的。\\
	\begin{examplebox}
		\textbf{例题:}测量长度为 $l$ 的一维盒子中质量为 $m$ 的粒子的能量。\\
		如果开始测量时,粒子的状态函数为\\
		(a)$\Psi = \left(30/l^5\right)^{1/2}x\left(l-x\right), \quad 0 \le x \le l$;\\
		(b)$\Psi = \left(2/l\right)^{1/2}\sin\left(3\pi x/l\right), \quad 0 \le x \le l$;\\
		测量结果可能是哪些值?\\
		\\
		(a)对性质 $E$ 进行测量的可能结果是系统能量(哈密顿)算符$\hat{H}$的本征值。因此,测量值一定是$n^2h^2/8ml^2, \quad n = 1,2,3\cdots$中的一个值。由于$\Psi$不是$\hat{H}$的本征函数[$\left(2/l\right)^{1/2}\sin\left(n\pi x/l\right)$式(\ref{eq:2.23 stationary state wave function for the particle in a box})],我们无法预测这个非稳态会得到其中哪一个本征值。(获得这些本征值的概率可在第 7.6 节的最后一个例子中找到)\\
		(b)由于$\Psi$是$\hat{H}$的本征函数,本征值为$3^2h^2/8ml^2$,测量值为$9^2h^2/8ml^2$。
	\end{examplebox}
	
	
	
	
	
	
	
	
	
	
	
	
	
	
	
	
	
\section{三维多粒子的薛定谔方程}

\section{三维盒子中的粒子}

\section{简并}

\section{平均值}

\section{波函数的约束条件}

\section*{总结}

\section*{习题}