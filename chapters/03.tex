% ===== CHAPTER 3 =====
\chapter{算符}
\section{算符}

\section{本征函数和本征值}
An operator $\hat{A}$ satisfies:
\begin{equation}
	\hat{A} \psi = a \psi
\end{equation}
where $a$ is the eigenvalue.

\section{算符和量子力学}
Key operators in quantum mechanics:
\begin{align}
	\text{Position:} & \quad \hat{x} = x \\
	\text{Momentum:} & \quad \hat{p}_x = -i\hbar \pdv{x} \\
	\text{Hamiltonian:} & \quad \hat{H} = -\frac{\hbar^2}{2m} \nabla^2 + V(\mathbf{r})
\end{align}

\section{三维多粒子的薛定谔方程}

\section{三维盒子中的粒子}

\section{简并}

\section{平均值}

\section{波函数的约束条件}

\section*{总结}

\section*{习题}