% ===== CHAPTER 5 =====
\chapter{角动量}
\label{chap:5}
\section{同时规范多个属性}
\label{sec:5.1 Simultaneous Specification of Several Properties}
    在本章中,我们将讨论角动量,在下一章中,我们将证明对于氢原子的静止态,电子角动量的大小是恒定的。首先,我们要考虑用什么标准来决定一个系统的哪些性质可以同时赋予定值。

    在第\ref{sec:3.3 Operators and Quantum Mechanics}节中,我们假设:若态函数$\Psi$时算符$\hat{A}$的本征函数,本征值为$s$,则对物理量$A$的测量结果一定为为$s$。若$\Psi$同时是$\hat{A}$和$\hat{B}$的本征函数,即$\hat{A}\Psi=s\Psi$,$\hat{B}\Psi=t\Psi$,那么我们就可以同时为物理量 $A$ 和 $B$ 赋以确定的值。什么时候$\Psi$会同时是两个算符的本征函数呢?在第\ref{chap:7}章中,我们将证明如下两个理论。首先,两个算符的同时本征函数的完整集合存在的必要条件是算符之间可对易(这里使用的“\textit{完整}”一词有一定的技术含义,我们将在第 \ref{chap:7} 章再讨论这个问题)。反之亦然,若对应于物理量的两个算符$\hat{A}$ 和 $\hat{B}$ 可对易,则存在一组完整的函数,它们同时是$\hat{A}$和$\hat{B}$的本征函数。因此,若$\left[\hat{A},\hat{B}\right]=0$,那么$\Psi$可以同时是$\hat{A}$和$\hat{B}$的本征函数。

    回忆算符$\hat{A}$和$\hat{B}$的交换子为$\left[\hat{A},\hat{B}\right] \equiv \hat{A}\hat{B} - \hat{B}\hat{A}$[式(\ref{eq:3.7 definition of commutator for two operators})]。下列等式有助于计算交换子。通过详细写出交换子可以证明这些等式(问题5.2):
    \begin{equation}
        \boxed{
            \left[\hat{A},\hat{B}\right] = -\left[\hat{B},\hat{A}\right]
        }
        \label{eq:5.1}
    \end{equation}
    \begin{equation}
        \boxed{
            \left[\hat{A}, \hat{A}^n\right] = 0, \quad n = 1, 2, 3, \ldots
        }
        \label{eq:5.2}
    \end{equation}
    \begin{equation}
        \boxed{
            \left[k\hat{A},\hat{B}\right] = \left[\hat{A}, k\hat{B}\right] = k\left[\hat{A},\hat{B}\right]
        }
        \label{eq:5.3}
    \end{equation}
    \begin{equation}
        \boxed{
            \left[\hat{A}, \hat{B}+\hat{C}\right] = \left[\hat{A},\hat{B}\right] + \left[\hat{A},\hat{C}\right]
        }
        \quad 
        \boxed{
            \left[\hat{A}+\hat{B},\hat{C}\right] = \left[\hat{A},\hat{C}\right] + \left[\hat{B},\hat{C}\right]
        }
        \label{eq:5.4}
    \end{equation}
    \begin{equation}
        \boxed{
            \left[\hat{A}, \hat{B}\hat{C}\right] = \left[\hat{A},\hat{B}\right]\hat{C} + \hat{B}\left[\hat{A},\hat{C}\right]
        }
        \quad
        \boxed{
            \left[\hat{A}\hat{B},\hat{C}\right] = \hat{A}\left[\hat{B},\hat{C}\right] + \left[\hat{A},\hat{C}\right]\hat{B}
        }
        \label{eq:5.5}
    \end{equation}
    其中$k$是常数,所有算符都假设为线性算符。
    \begin{examplebox}
        \textbf{例题:}从$\left[\partial/\partial x,x\right]=1$[式(\ref{eq:3.8})]出发,使用对易子的性质(式(\ref{eq:5.1})至(\ref{eq:5.5}))求以下对易子:(a)$\left[\hat{x},\hat{p_x}\right]$;(b)$\left[\hat{x}, \hat{p_x^2}\right]$;(c)三维单粒子系统的$\left[\hat{x},\hat{H}\right]$。
        \\
        (a)使用式(\ref{eq:5.3})和(\ref{eq:5.1}),因为$\left[\partial/\partial x,x\right]=1$,我们有
        \begin{equation*}
            \left[\hat{x},\hat{p_x}\right] = \left[\hat{x},\frac{\hbar}{\mathrm{i}}\frac{\partial}{\partial x}\right] = \frac{\hbar}{\mathrm{i}}\left[\hat{x},\frac{\partial}{\partial x}\right] = -\frac{\hbar}{\mathrm{i}}\left[\frac{\partial}{\partial x},\hat{x}\right] = -\frac{\hbar}{\mathrm{i}}
        \end{equation*}
        \begin{equation}
            \left[\hat{x},\hat{p_x}\right]=\mathrm{i}\hbar
            \label{eq:5.6}
        \end{equation}
        (b)由式(\ref{eq:5.5})和(\ref{eq:5.6}),有
        \begin{equation*}
            \left[\hat{x},\hat{p_x^2}\right] = \left[\hat{x},\hat{p_x}\right]\hat{p_x} + \hat{p_x}\left[\hat{x},\hat{p_x}\right] = \mathrm{i}\hbar\cdot\frac{\hbar}{\mathrm{i}}\frac{\partial}{\partial x} + \frac{\hbar}{\mathrm{i}}\frac{\partial}{\partial x} \cdot \mathrm{i}\hbar
        \end{equation*}
        \begin{equation}
            \left[\hat{x},\hat{p_x^2}\right] = 2\hbar^2\frac{\partial}{\partial x}
            \label{eq:5.7}
        \end{equation}
        (c)使用式(\ref{eq:5.4})、(\ref{eq:5.3})和(\ref{eq:5.7}),有
        \begin{equation*}
            \begin{aligned}
                \left[\hat{x},\hat{H}\right] &= \left[\hat{x},\hat{T}+\hat{V}\right] = \left[\hat{x},\hat{T}\right] + \left[\hat{x},\hat{V}\left(x,y,z\right)\right] =\left[\hat{x},\hat{T}\right] \\
                &= \left[\hat{x}, \left(1/2m\right)\left(\hat{p_x^2}+\hat{p_y^2}+\hat{p_z^2}\right)\right]\\
                &= \left(1/2m\right)\left[\hat{x},\hat{p_x^2}\right] + \left(1/2m\right)\left[\hat{x},\hat{p_y^2}\right] + \left(1/2m\right)\left[\hat{x},\hat{p_z^2}\right]\\
                &= \frac{1}{2m}2\hbar^2\frac{\partial}{\partial x} + 0 + 0
            \end{aligned}
        \end{equation*}
        \begin{equation}
            \left[\hat{x},\hat{H}\right] = \frac{\hbar^2}{m}\frac{\partial}{\partial x} = \frac{\mathrm{i}\hbar}{m}\hat{p_x}
            \label{eq:5.8}
        \end{equation}
        \\
        \textbf{练习:}求证:对于单粒子三维系统,有
        \begin{equation}
            \left[\hat{p_x^2},\hat{H}\right] = -\mathrm{i}\hbar \:\partial V\left(x,y,z\right)/\partial x
            \label{eq:5.9}
        \end{equation}
    \end{examplebox}

    这些对易子有重要的物理影响。由于$\left[\hat{x},\hat{p_x}\right] \neq 0$,我们不能希望态函数同时是$\hat{x}$和$\hat{p_x}$的本征函数。因此,与不确定性原理相符,我们不能同时测定$x$和$p_x$的精确值。由于$\hat{x}$与$\hat{H}$不可对易,我们不能指望同时为能量和 $x$ 坐标赋予确定的值。定态(具有确定的能量)会显示出$x$的各种可能值,观察到各种$x$值的概率由玻恩公设给出。

    对于不是$\hat{A}$的本征函数的态函数$\Psi$,在完全相同的系统中对$A$的测量将得到各种可能的结果。我们需要对观测值 $A_i$ 的分布或离散性进行某种测量。如果 $\langle A \rangle$ 是这些值的平均值,那么每个测量值与平均值的偏差为$A_i - \langle A \rangle$。如果我们对所有偏差求平均值,就会得到零,因为正负偏差会抵消。因此,为了使所有偏差都为正值,我们要将它们平方。偏差平方的平均值称为 $A$ 的\textbf{方差}(variance),在统计学中用 $\sigma^2\left(A\right)$ 表示,在量子力学中用 $\left(\Delta A\right)^2$ 表示:
    \begin{equation}
        \left(\Delta A\right)^2 \equiv \sigma^2\left(A\right) = \langle \left(A - \langle A \rangle\right)^2 \rangle = \int \Psi^{\ast}\left(\hat{A}-\langle A \rangle\right)^2\Psi\mathrm{d}\tau
        \label{eq:5.10}
    \end{equation}
    其中我们用到了平均值的表达式(\ref{eq:3.88})。定义(\ref{eq:5.10})与以下表达式等价(问题5.7):
    \begin{equation}
        \left(\Delta A\right)^2 = \langle A^2 \rangle - \langle A \rangle^2
        \label{eq:5.11}
    \end{equation}

    方差的正平方根称为\textbf{标准差}(standard deviation),用 $\Delta A$ 或$\sigma\left(A\right)$ 表示。标准差是最常用的价差度量,我们将用它来度量属性 $A$ 的 “不确定性”。

    罗伯逊(Robertson)于 1929 年证明,状态函数为 $\Psi$ 的量子力学系统的两个属性的标准偏差的乘积必须满足不等式
    \begin{equation}
        \sigma\left(A\right)\sigma\left(B\right) = \Delta A \Delta B \ge \frac{1}{2}\left|\int \Psi^{\ast}\left[\hat{A},\hat{B}\right]\Psi\mathrm{d}\tau\right|
        \label{eq:5.12}
    \end{equation}
    (\ref{eq:5.12}) 的证明源于量子力学公设,见问题 7.60。如果$\hat{A}$和$\hat{B}$可对易,那么式(\ref{eq:5.12})中的积分为零,则$\Delta A$和$\Delta B$可能都为零,与上文的讨论相符。

    作为式(\ref{eq:5.12})的一个例子,我们使用式(\ref{eq:5.6})、$\left|z_1z_2\right|=\left|z_1\right|\left|z_2\right|$[式(\ref{eq:1.34 properties of product and quotient absolute values})],以及归一化方法:
    \begin{equation*}
        \Delta x \Delta p_x \ge \frac{1}{2}\left|\int \Psi^{\ast}\left[\hat{x},\hat{p_x}\right]\Psi\mathrm{d}\tau\right| = \frac{1}{2}\left|\int \Psi^{\ast}\mathrm{i}\hbar\Psi\mathrm{d}\tau\right| = \frac{1}{2}\hbar\left|\mathrm{i}\right|\left|\int \Psi^{\ast}\Psi\mathrm{d}\tau\right|
    \end{equation*}
    \begin{equation}
        \sigma\left(x\right)\sigma\left(p_x\right) \equiv \Delta x \Delta p_x \ge \frac{1}{2}\hbar
        \label{eq:5.13}
    \end{equation}

    式(\ref{eq:5.13})通常被认为是\textbf{海森堡不确定性原理}(Heisenberg Uncertainty Principle)(第\ref{sec:1.3 The Uncertainty Principle}节)的定量表述。然而,公式 (\ref{eq:5.12}) 和 (\ref{eq:5.13}) 中标准偏差的含义与第 \ref{sec:1.3 The Uncertainty Principle} 节中不确定度的含义截然不同。为了在式(\ref{eq:5.13})中找到$\Delta x$,我们选取大量的系统,每个系统都有相同的状态函数 $\psi$,在每个系统中对 $x$ 进行一次测量。将测得的数据标记为$x_i$,计算$\langle x \rangle$和偏差的平方$\left(x_1-\langle x \rangle\right)^2$。我们求出偏差平方的平均值,得到方差,再取平方根,得到标准偏差$\sigma\left(x\right)\equiv \Delta x$。然后我们再取大量的系统,它们具有与得到$\Delta x$时一致的状态$\Psi$,对每个系统测量一次$p_x$,由这些数据计算出$\Delta p_x$。因此,(\ref{eq:5.13})中的统计量 $\Delta x$ 和 $\Delta p_x$ 不是单个测量的误差,也不是通过同时测量 $x$ 和 $p_x$ 得出的。

    令$\varepsilon\left(x\right)$是对 $x$ 的单次测量的典型误差,令$\eta \left(p_x\right)$是测量 $x$ 时对 $p_x$ 造成的典型干扰。1927 年,海森堡分析了进行位置测量的具体思想实验,得出结论:乘积 $\varepsilon\left(x\right)\eta\left(x\right)$ 的数量级为 $h$ 或更大。海森堡没有给出这些量的精确定义。小泽正直(Masanao Ozawa)将海森堡的关系改写为
    \begin{equation*}
        \varepsilon\left(x\right)\eta\left(p_x\right) \ge \frac{1}{2}\hbar
    \end{equation*}
    其中$\varepsilon\left(x\right)$定义为$x$ 的测量值与理论值的方均根偏差,$\eta\left(p_x\right)$定义为测量 $x$ 所产生的$p_x$变化的方均根偏差。更一般地说,对于任意两个性质,小泽将海森堡不确定性原理写成
    \begin{equation}
        \varepsilon\left(A\right)\eta\left(B\right) \ge \frac{1}{2}\left|\int \Psi^{\ast}\left[\hat{A},\hat{B}\right]\Psi\mathrm{d}\tau\right|
        \label{eq:5.14}
    \end{equation}
    小泽提出的论点是:在某些情况下,海森堡不等式 (\ref{eq:5.14}) 可能被违反。小泽推导出以下关系来取代 (\ref{eq:5.14}) [M. Ozawa, Phys. Ozawa, Phys. Rev. A, 67, 042105 (2003); available at arxiv.org/abs/quant-ph/0207121]:
    \begin{equation*}
        \varepsilon\left(A\right)\eta\left(B\right) + \varepsilon\left(A\right)\sigma\left(B\right) + \sigma\left(A\right)\eta\left(B\right) \ge \frac{1}{2}\left|\int \Psi^{\ast}\left[\hat{A},\hat{B}\right]\Psi\mathrm{d}\tau\right|
    \end{equation*}
    其中$\sigma\left(A\right)$和$\sigma\left(B\right)$由式(\ref{eq:5.10})给出。2012 年,一项测量中子自旋分量的实验发现:自旋分量不服从海森堡误差-扰动不等式 (\ref{eq:5.14}),但服从小泽不等式[J. Erhart 等,《自然物理学》,8, 185 (2012); arxiv.org/abs/1201.1833]。

    另一个不等式是海森堡不确定关系,即用一台仪器同时测量 $A$ 和 $B$ 的两种性质:
    \begin{equation*}
        \varepsilon\left(A\right)\varepsilon\left(B\right) \ge \frac{1}{2}\left|\int \Psi^{\ast}\left[\hat{A},\hat{B}\right]\Psi\mathrm{d}\tau\right|
    \end{equation*}
    其中$\varepsilon\left(A\right)$和$\varepsilon\left(B\right)$分别是测量$A$与$B$时的实验误差。这一关系已被证明是成立的,前提是某个可信的假设(相信对目前所有可用的测量设备都是成立的)成立(见上文引用的小泽论文参考文献 6-12)。
    \begin{examplebox}
        \textbf{例题:}式 (\ref{eq:3.91}) 、 (\ref{eq:3.92}) 、 (\ref{eq:3.39}) 、 (\ref{eq:3.89}) 之后的等式和问题 3.48 指出:对于单粒子三维系统的基态,有
        \begin{equation*}
            \langle x \rangle = \frac{a}{2}, \quad \langle x^2 \rangle = \left(\frac{1}{3}-\frac{1}{2\pi^2}\right)a^2, \quad \langle p_x \rangle = 0, \quad \langle p_x^2 \rangle = \frac{h^2}{4a^2}
        \end{equation*}
        使用这些结果证明该系统是否遵守不确定性原理(\ref{eq:5.13})。
        \\ \\
        我们有
        \begin{equation*}
            \left(\Delta x\right)^2 = \langle x^2 \rangle - \langle x \rangle^2 = \left(\frac{1}{3}-\frac{1}{2\pi^2}\right)a^2 - \frac{a^2}{4} = \frac{\pi^2-6}{12\pi^2}a^2
        \end{equation*}
        \begin{equation*}
            \Delta x = \left(\frac{\pi^2-6}{12}\right)^{1/2}\frac{a}{\pi}
        \end{equation*}
        \begin{equation*}
            \Delta p_x = \langle p_x^2 \rangle - \langle p_x \rangle^2 = \frac{h^2}{4a^2}, \quad \Delta p_x = \frac{h}{2a}
        \end{equation*}
        \begin{equation*}
            \Delta x \Delta p_x = \left(\frac{\pi^2-6}{12}\right)^{1/2}\frac{h}{2\pi} = 0.568\hbar > \frac{1}{2}\hbar
        \end{equation*}
    \end{examplebox}

    还有一种涉及能量和时间的不确定关系:
    \begin{equation}
        \Delta E \Delta t \ge \frac{1}{2}\hbar
        \label{eq:5.15}
    \end{equation}
    一些文章指出:(\ref{eq:5.15}) 是通过将 $\mathrm{i}\hbar \: \partial /\partial t$ 作为能量算符,并乘以 $t$ 作为时间算符而从 (\ref{eq:5.12}) 推导出来的。然而,能量算符是哈密顿$\hat{H}$,而不是 $\mathrm{i}\hbar \: \partial /\partial t$。此外,时间不是一种可观测的参数,而是量子力学中的一个参数。因此,不存在量子力学的时间算符(量子力学中的名词 \textbf{可观测}(observable) 指的是系统的物理可测量属性)。方程 (\ref{eq:5.15}) 必须通过特殊处理才能得出,我们略去不表(见 \textit{Ballentine},第12.3节)。式 (\ref{eq:5.15}) 的推导表明:$\Delta t$被解释为能量不确定度为$\Delta E$的状态的时间寿命。通常情况下,(\ref{eq:5.15})中的 $\Delta t$ 是能量测量的持续时间。然而,Aharonov和Bohm已经证明:“能量可以在任意短的时间内重复测量”[Y. Aharonov and D. Bohm, \textit{Phys. Rev.}, 122, 1649 (1961); 134, B1417 (1964); see also S. Massar and S. Popescu, \textit{Phys. Rev.} A, 71, 042106 (2005); P. Busch, \textit{The Time–Energy Uncertainty Relation}, arxiv.org/abs/quant-ph/0105049]。

    现在,我们来考虑同时为三个物理量赋予确定值的可能性:假设
    \begin{equation}
        \left[\hat{A},\hat{B}\right] = 0, \quad \left[\hat{A},\hat{C}\right] = 0
        \label{eq:5.16}
    \end{equation}
    这是否足以确保存在三个算符的同时本征函数?由于$\left[\hat{A},\hat{B}\right] = 0$,我们可以为 $\hat{A}$ 和 $\hat{B}$ 构造一组共同的本征函数。由于$\left[\hat{A},\hat{C}\right] = 0$,我们可以为 $\hat{A}$ 和 $\hat{C}$ 构造一组共同的本征函数。如果这两组本征函数是相同的,那么我们就会为所有三个算符找到一组共同的本征函数。因此我们要问:线性算符 $\hat{A}$ 的本征函数集是唯一确定的吗(除了乘以任意常数)?一般来说,答案都是否。如果与 $\hat{A}$ 的本征值相对应的独立本征函数不止一个(即简并),那么简并本征值的本征函数的任何线性组合都是$\hat{A}$ 的本征函数(第 \ref{sec:3.6 Degeneracy} 节)。给出 $\hat{B}$ 本征函数所需的适当线性组合很可能与给出 $\hat{C}$ 本征函数的线性组合不同。事实上,要想拥有三个算符的共同完整的本征函数集,除了满足式(\ref{eq:5.16})外,还需要满足$\left[\hat{B},\hat{C}\right] = 0$。\textit{要有一组完整的函数同时是几个算符的本征函数,每个算符必须与其他每个算符可对易。}

\section{矢量}
\label{sec:5.2 Vectors}
    下一节我们将求解角动量的本征值问题,角动量是一种矢量。因此,我们首先回顾一下矢量。

    完全由大小指定的物理属性(如质量、长度、能量)称为\textbf{标量}(scalar)。需要同时指定大小和方向的物理属性(如力、速度、动量)称为\textbf{矢量}(vector)。矢量由一条有方向的线段表示,其长度和方向表示属性的大小和方向。

    两个矢量 $\mathbf{A}$ 和 $\mathbf{B}$ 的和由如下过程定义:滑动第一个矢量,使其终点与第二个矢量的起点重合,同时保持其方向不变。然后从第二个矢量的尾部到第一个矢量的头部绘制一个新的矢量。参见图 \ref{fig:5.1}。矢量与标量乘积$c\mathbf{A}$定义为:长度为$\left|c\right|$的矢量长度乘以$\mathbf{A}$的长度,若$c$为正,则方向与$\mathbf{A}$相同;若$c$为负,则方向与$\mathbf{A}$相反。
    \begin{figure}[h!]
        \centering
        \includegraphics[width=0.7\textwidth]{Figures/5.1.png}
        \caption{两个矢量的和}
        \label{fig:5.1}
    \end{figure}
    \begin{figure}[h!]
        \centering
        \includegraphics[width=0.7\textwidth]{Figures/5.2.png}
        \caption{单位矢量$\mathrm{i}$、$\mathrm{j}$和$\mathrm{k}$,$\mathrm{A}$的三个分量}
        \label{fig:5.2}
    \end{figure}

    为了获得表示矢量的代数(以及几何)方法,我们在空间中建立了笛卡尔坐标。我们在$x$轴正方向画一个单位矢量,称为$\mathbf{i}$(与虚数单位没有关系)。$y$轴和$z$轴的单位矢量分别称为$\mathbf{j}$和$\mathbf{k}$(图 \ref{fig:5.2})。为了用这三个单位矢量表示空间中的任意矢量$\mathbf{A}$,我们首先将$\mathbf{A}$的起点与原点重合,保持其方向不变。然后,我们找到$\mathbf{A}$在$x$轴、$y$轴和$z$轴上的投影,分别为 $A_x$、$A_y$ 和 $A_z$。由矢量加和的定义,我们有(图 \ref{fig:5.2}):
    \begin{equation}
        \boxed{
            \mathbf{A} = A_x\mathbf{i} + A_y\mathbf{j} + A_z\mathbf{k}
        }
        \label{eq:5.17}
    \end{equation}
    我们可以通过指定$\mathbf{A}$的三个分量($A_x$、$A_y$ 和 $A_z$)来指定矢量 $\mathbf{A}$。因此,三维空间中的矢量可以定义为由三个数组成的有序集合。

    当且仅当两个矢量所有对应的分量都相等时:$A_x = B_x$、$A_y = B_y$ 和 $A_z = B_z$,我们才说这两个向量$\mathbf{A}$和$\mathbf{B}$相等。因此,一个矢量方程等价于三个标量方程。

    为了将两个矢量相加,我们将它们对应的分量相加:
    \begin{equation*}
        \mathbf{A} + \mathbf{B} = A_x\mathbf{i} + A_y\mathbf{j} + A_z\mathbf{k} + B_x\mathbf{i} + B_y\mathbf{j} + B_z\mathbf{k}
    \end{equation*}
    \begin{equation}
        \boxed{
            \mathbf{A} + \mathbf{B} = (A_x + B_x)\mathbf{i} + (A_y + B_y)\mathbf{j} + (A_z + B_z)\mathbf{k}
        }
        \label{eq:5.18}
    \end{equation}
    若$c$是标量,则
    \begin{equation}
        \boxed{
            c\mathbf{A} = cA_x\mathbf{i} + cA_y\mathbf{j} + cA_z\mathbf{k}
        }
        \label{eq:5.19}
    \end{equation}

    矢量$\mathbf{A}$的\textbf{模}(magnitude)为它的长度,由$A$或$\left|\mathbf{A}\right|$表示。模$A$是一个标量。

    两个矢量$\mathbf{A}\cdot\mathbf{B}$的\textbf{点积}(dot product)或\textit{数量积}(scalar product)定义为
    \begin{equation}
        \boxed{
            \mathbf{A}\cdot\mathbf{B} = \left|A\right|\left|B\right|\cos\theta = \mathbf{B}\cdot\mathbf{A}
        }
        \label{eq:5.20}
    \end{equation}
    其中$\theta$是$\mathbf{A}$和$\mathbf{B}$之间的夹角。点积是三个标量的乘积,因此是一个标量。注意$\left|\mathbf{A}\right|\cos\theta$是$\mathbf{A}$在$\mathbf{B}$方向上的投影。根据矢量加法的定义,可以得出矢量 $\mathbf{A}$ + $\mathbf{B}$ 在某个矢量 $\mathbf{C}$ 上的投影是 $\mathbf{A}$ 和 $\mathbf{B}$ 在 $\mathbf{C}$ 上的投影之和。因此,
    \begin{equation}
        \left(\mathbf{A} + \mathbf{B}\right)\cdot\mathbf{C} = \left(\mathbf{A}\cdot\mathbf{C}\right) + \left(\mathbf{B}\cdot\mathbf{C}\right)
        \label{eq:5.21}
    \end{equation}
    由于三个单位矢量 $\mathbf{i}$、$\mathbf{j}$ 和 $\mathbf{k}$ 的长度均为单位长度且互相垂直,我们有
    \begin{equation}
        \mathbf{i}\cdot\mathbf{i} = \mathbf{j}\cdot\mathbf{j} = \mathbf{k}\cdot\mathbf{k} = \cos 0 = 1, \quad \mathbf{i}\cdot\mathbf{j} = \mathbf{i}\cdot\mathbf{k} = \mathbf{j}\cdot\mathbf{k} = \cos \left(\pi/2\right) = 0
        \label{eq:5.22}
    \end{equation}

    使用(\ref{eq:5.22})和分配律(\ref{eq:5.21}),我们有
    \begin{equation*}
        \mathbf{A}\cdot\mathbf{B} = \left(A_x\mathbf{i} + A_y\mathbf{j} + A_z\mathbf{k}\right)\cdot\left(B_x\mathbf{i} + B_y\mathbf{j} + B_z\mathbf{k}\right)
    \end{equation*}
    \begin{equation}
        \boxed{
            \mathbf{A}\cdot\mathbf{B} = A_xB_x + A_yB_y + A_zB_z
        }
        \label{eq:5.23}
    \end{equation}
    其中点积的 9 项中有 6 项为零。

    由式(\ref{eq:5.20})有
    \begin{equation}
        \boxed{
            \mathbf{A}\cdot\mathbf{A} = \left|\mathbf{A}\right|^2
        }
        \label{eq:5.24}
    \end{equation}
    使用式(\ref{eq:5.23}),我们有
    \begin{equation}
        \boxed{
            \left|\mathbf{A}\right| = \left(A_x^2 + A_y^2 + A_z^2\right)^{1/2}
        }
        \label{eq:5.25}
    \end{equation}

    对于三维矢量,还有另一种乘积方式。$\mathbf{A}\times\mathbf{B}$的\textbf{叉积}(cross product)或\textit{向量积}(vector product)是一个新的矢量,其模为
    \begin{equation}
        \mathbf{A}\times\mathbf{B} = \left|\mathbf{A}\right|\left|\mathbf{B}\right|\sin\theta
        \label{eq:5.26}
    \end{equation}
    其线段垂直于$\mathbf{A}$和$\mathbf{B}$所在的平面,其方向与$\mathbf{A}$和$\mathbf{B}$形成一个右手定则的系统(与$x$轴、$y$轴和$z$轴的右手定则相同)。见图 \ref{fig:5.3}。因此,由定义,我们有
    \begin{equation*}
        \mathbf{B}\times\mathbf{A} = -\mathbf{A}\times\mathbf{B}
    \end{equation*}
    同样地(见\textit{Taylor and Mann}, Section 10.2)
    \begin{equation}
        \mathbf{A}\times\left(\mathbf{B}+\mathbf{C}\right) = \mathbf{A}\times\mathbf{B} + \mathbf{A}\times\mathbf{C}
        \label{eq:5.27}
    \end{equation}
    \begin{figure}[h!]
        \centering
        \includegraphics[width=0.4\textwidth]{Figures/5.3.png}
        \caption{两个矢量的叉积}
        \label{fig:5.3}
    \end{figure}
    对于三个单位向量,我们有
    \begin{equation*}
        \mathbf{i}\times\mathbf{i} = \mathbf{j}\times\mathbf{j} = \mathbf{k}\times\mathbf{k} = \sin \: 0 = 0
    \end{equation*}
    \begin{equation*}
        \mathbf{i}\times\mathbf{j} = \mathbf{k}, \quad \mathbf{j}\times\mathbf{i} = -\mathbf{k}, \quad \mathbf{j}\times\mathbf{k} = \mathbf{i}, \quad \mathbf{k}\times\mathbf{j} = -\mathbf{i}, \quad \mathbf{k}\times\mathbf{i} = \mathbf{j}, \quad \mathbf{i}\times\mathbf{k} = -\mathbf{j}
    \end{equation*}
    使用这些等式及分配律(\ref{eq:5.27}),我们有
    \begin{equation*}
        \mathbf{A}\times\mathbf{B} = \left(A_x\mathbf{i} + A_y\mathbf{j} + A_z\mathbf{k}\right)\times\left(B_x\mathbf{i} + B_y\mathbf{j} + B_z\mathbf{k}\right)
    \end{equation*}
    \begin{equation*}
        \mathbf{A}\times\mathbf{B} = \left(A_yB_z - A_zB_y\right)\mathbf{i} + \left(A_zB_x - A_xB_z\right)\mathbf{j} + \left(A_xB_y - A_yB_x\right)\mathbf{k}
    \end{equation*}
    为了便于记忆,我们可以将叉积表示为行列式(见第 8.3 节):
    \begin{equation}
        \mathbf{A}\times\mathbf{B} = \left|
            \begin{array}{ccc}
            \mathbf{i} & \mathbf{j} & \mathbf{k} \\
            A_x & A_y & A_z \\
            B_x & B_y & B_z
            \end{array}
        \right| = \mathbf{i}\left|
            \begin{array}{ccc}
            A_y & A_z \\
            B_y & B_z
            \end{array}
        \right| - \mathbf{j}\left|
            \begin{array}{ccc}
            A_x & A_z \\
            B_x & B_z
            \end{array}
        \right| + \mathbf{k}\left|
            \begin{array}{ccc}
            A_x & A_y \\
            B_x & B_y
            \end{array}
        \right|
        \label{eq:5.28}
    \end{equation}

    我们定义如下的矢量\textbf{微分}(del)算符:
    \begin{equation}
        \boxed{
            \nabla \equiv \mathbf{i}\frac{\partial}{\partial x} + \mathbf{j}\frac{\partial}{\partial y} + \mathbf{k}\frac{\partial}{\partial z}
        }
        \label{eq:5.29}
    \end{equation}
    则由式(\ref{eq:3.23}),线性动量的矢量算符为$\hat{\mathbf{p}}=-\mathrm{i}\hbar\nabla$。

    函数$g\left(x,y,z\right)$的\textbf{梯度}(gradient)定义为将微分算符作用到函数$g$上的结果:
    \begin{equation}
        \boxed{
            \mathbf{grad} \: g\left(x,y,z\right) \equiv \nabla g\left(x,y,z\right) \equiv \mathbf{i}\frac{\partial g}{\partial x} + \mathbf{j}\frac{\partial g}{\partial y} + \mathbf{k}\frac{\partial g}{\partial z}
        }
        \label{eq:5.30}
    \end{equation}
    一个标量函数的梯度为一个矢量函数。矢量$\nabla g\left(x,y,z\right)$代表函数$g$的空间变化率:$\nabla \:g$在$x$方向的分量表示函数$g$在$x$方向的变化率,以此类推。可以证明,矢量$\nabla g$指向 g 变化率最大的方向。由式(\ref{eq:4.24}),势能和力之间的关系为
    \begin{equation}
        \mathbf{F} = -\nabla V\left(x,y,z\right) = -\mathbf{i}\frac{\partial V}{\partial x} - \mathbf{j}\frac{\partial V}{\partial y} - \mathbf{k}\frac{\partial V}{\partial z}
        \label{eq:5.31}
    \end{equation}

    假设某矢量$\mathbf{A}$的三个分量是某参数$t$的函数,即$A_x = A_x\left(t\right)$、$A_y = A_y\left(t\right)$和$A_z = A_z\left(t\right)$。则我们将$\mathbf{A}$对$t$的\textbf{导数}(derivative)定义为
    \begin{equation}
        \frac{\mathrm{d}\mathbf{A}}{\mathrm{d}t} = \mathbf{i}\frac{\mathrm{d}A_x}{\mathrm{d}t} + \mathbf{j}\frac{\mathrm{d}A_y}{\mathrm{d}t} + \mathbf{k}\frac{\mathrm{d}A_z}{\mathrm{d}t}
        \label{eq:5.32}
    \end{equation}

    矢量符号是表示函数变量的一种便捷方法。双粒子系统的波函数可以写作$\psi\left(x_1,y_1,z_1,x_2,y_2,z_2\right)$,如果矢量$\mathbf{r}_1$从原点指向粒子1,那么$\mathbf{r}_1$就具有$x,y,z$三个方向的分量,指定$\mathbf{r}_1$等同于指定三个坐标$x,y,z$。$\mathbf{r}_2$也是如此。因此,我们可以将波函数写成$\psi\left(\mathbf{r}_1,\mathbf{r}_2\right)$。矢量符号也可以用在积分中,例如式(\ref{eq:3.57})对全空间的积分经常写成$\int \cdots \int\left|\Psi\left(\mathbf{r}_1,\cdots,\mathbf{r}_n\right)\right|^2\mathrm{d}\mathbf{r}_1\cdots\mathrm{d}\mathbf{r}_n$。
    \\
    
    \noindent\textbf{$n$维空间的矢量}

    矢量的定义可以推广到三维以上的空间。三维空间中的矢量$\mathbf{A}$可以由其模$\left|\mathbf{A}\right|$和方向来定义;若在笛卡尔坐标系中,还可以用其三个分量$\left(A_x,A_y,A_z\right)$来定义。因此,我们可以把三维矢量定义为按特定顺序排列的三个实数$\left(A_x,A_y,A_z\right)$的集合。$n$ 维实数矢量 “空间”(有时称为超空间)中的矢量 $\mathbf{B}$ 定义为 $n$ 个实数的有序集合$\left(B_1,B_2,\cdots,B_n\right)$,其中$B_1$,$B_2$,$\cdots$,$B_n$是$\mathbf{B}$的\textbf{分量}(components)。不要担心无法将 $n$ 维空间中的矢量可视化。

    函数的变量通常使用 $n$ 维矢量符号表示。例如,在双粒子系统中,与$\psi\left(x_1,y_1,z_1,x_2,y_2,z_2\right)$不同,我们可以定义一个六维矢量$\mathbf{q}$,其分量满足$q_1=x_1, q_2=y_1, q_3=z_1, q_4=x_2, q_5=y_2, q_6=z_2$,且将波函数写成 $\psi\left(\mathbf{q}\right)$。对于有$n$个粒子的系统,我们可以定义一个有$3n$个分量的矢量$\mathbf{q}$,将波函数写成$\psi\left(\mathbf{q}\right)$,使用$\int\left|\psi\left(\mathbf{q}\right)\right|^2\mathrm{d}\mathbf{q}$来进行归一化处理。

    寻找分子平衡几何的理论使用 $n$ 维矢量(第 \ref{sec:15.10 Molecular Geometry} 节)。第 \ref{sec:5.2 Vectors} 节的其余部分与第 \ref{sec:15.10 Molecular Geometry} 节相关,在学习第 \ref{sec:15.10 Molecular Geometry} 节之前不必阅读。

    当且仅当$n$维矢量$\mathbf{B}$和$\mathbf{C}$的所有分量都相等时:$B_1 = C_1$、$B_2 = C_2$、$\cdots$、$B_n = C_n$,我们才说这两个向量\textbf{相等}(equal)。因此,在$n$维空间中,一个矢量方程等价于 $n$ 个标量方程。两个$n$维矢量$\mathbf{B}$与$\mathbf{D}$\textbf{和}(sum)的定义为矢量$\left(B_1+D_1,B_2+D_2,\cdots,B_n+D_n\right)$。矢量的差定义类似。矢量$k\mathbf{B}$定义为矢量$\left(kB_1,kB_2,\cdots,kB_n\right)$,其中 $k$ 是标量。在三维空间中,若$k>0$,则矢量$k\mathbf{A}$的方向一致。在$n$维空间中亦然。与实数对$\left(A_x,A_y,A_z\right)$定义三维空间中的一个点类似,$\left(B_1,B_2,\cdots,B_n\right)$定义$n$维空间中的一个\textbf{点}(point)。

    $n$ 维实数矢量的\textbf{长度}(length)(或\textbf{大小}(magnitude)或\textbf{欧几里得范数}(Euclidean norm))$\left|\mathbf{B}\right|$(有时表示为$\left|\left|\mathbf{B}\right|\right|$ )定义为
    \begin{equation*}
        \left|\mathbf{B}\right| \equiv \left(\mathbf{B}\cdot\mathbf{B}\right)^{1/2} = \left(B_1^2 + B_2^2 + \cdots + B_n^2\right)^{1/2}
    \end{equation*}
    长度为1的矢量称为\textbf{归一化的}(normalized)。

    两个$n$维矢量$\mathbf{B}$和$\mathbf{G}$的\textbf{内积}(inner product)或\textbf{数量积}(scalar product)定义为标量
    \begin{equation*}
        \mathbf{B}\cdot\mathbf{G} \equiv B_1G_1 + B_2G_2 + \cdots + B_nG_n
    \end{equation*}
    若$\mathbf{B}\cdot\mathbf{G}=0$,则称$\mathbf{B}$和$\mathbf{G}$\textbf{正交}(orthogonal)。由式(\ref{eq:5.20}),两个$n$维矢量夹角$\theta$的余弦值定义为$\cos\theta \equiv \left(\mathbf{B}\cdot\mathbf{C}\right)/\left|\mathbf{B}\right|\left|\mathbf{C}\right|$。我们可以证明,这个定义使得 $\theta$ 位于-1 到 1 的范围内。

    在三维空间内,单位矢量$\mathbf{i} = \left(1,0,0\right), \mathbf{j} = \left(0,1,0\right), \mathbf{k} = \left(0,0,1\right)$互相垂直。因此,任意矢量都可以表示为这三个矢量的线性组合[式(\ref{eq:5.17})]。在$n$维空间中,单位矢量$\mathbf{e}_1 \equiv \left(1,0,\cdots,0\right), \mathbf{e}_2 \equiv \left(0,1,\cdots,0\right), \cdots, \mathbf{e}_n \equiv \left(0,0,\cdots,1\right)$互相正交。由于任意$n$维矢量$\mathbf{B}$等于$B_1\mathbf{e}_1 + B_2\mathbf{e}_2 + \cdots + B_n\mathbf{e}_n$,任意$n$维矢量都可以表示为这$n$个单位矢量的线性组合。因此,这 $n$ 个矢量可以说是 $n$ 维实数矢量空间的\textbf{基}(basis)。由于矢量$\mathbf{e_1}, \mathbf{e}_2, \cdots, \mathbf{e}_n$是正交且归一的,它们是这个实矢量空间的\textbf{正交}基(orthonormal basis)。点积$\mathbf{B}\cdot\mathbf{e_i}$给出了矢量 $\mathbf{B}$ 在 $\mathbf{e}_i$ 方向上的分量。矢量空间有许多可能的基集。任何由 $n$ 个线性无关的实矢量组成的集合都可以作为 $n$ 维实矢量空间的基。

    一个三维矢量可以用它的三个分量来指定,也可以用它的长度和方向来指定。可以通过给出矢量与 $x$、$y$ 和 $z$ 轴正半轴所成的三个角度来指定方向。这些角度称为矢量的\textbf{方向角}(direction angles),其值在$0$到$\pi$之间。但是,一旦给出了其他两个方向角,$z$ 轴的方向角就会固定下来,因此只有两个方向角是独立的。因此,一个三维矢量可以用长度和两个方向角来表示。同样,在$n$维空间中,一个矢量和每个单位矢量$\mathbf{e_1},\mathbf{e_2},\cdots,\mathbf{e_n}$之间的方向角也可以根据上述两个矢量之间夹角的余弦公式求出。因此,一个 $n$ 维矢量可以用它的长度和 $n-1$ 个方向角来指定。

    三变量函数的梯度由 (\ref{eq:5.30}) 定义。$n$变量函数$f\left(q_1,q_2,\cdots,q_n\right)$的\textbf{梯度}(gradient)定义为$n$维矢量,其分量分别为$f$对每个变量的一阶偏导数:
    \begin{equation*}
        \nabla f \equiv \left(\partial f/\partial q_1\right)\mathbf{e_1} + \left(\partial f/\partial q_2\right)\mathbf{e}_2 + \cdots + \left(\partial f/\partial q_n\right)\mathbf{e}_n
    \end{equation*}

    我们考虑的是实数 $n$ 维矢量空间。狄拉克(Dirac)对量子力学的表述使用的是复数、无穷维的矢量空间,对它的讨论从略。
















\section{单粒子系统的角动量}
\label{sec:5.3 Angular Momentum of a One-Particle System}

\section{角动量梯度算符法}
\label{sec:5.4 The Ladder-Operator Method for Angular Momentum}

\section*{总结}

\section*{习题}