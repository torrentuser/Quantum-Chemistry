% ===== CHAPTER 4 =====
\chapter{谐振子}
\label{chap:4}
    我们将在第 13.2 节中看到,气相分子的能量可以近似为平动、转动、振动和电子能量之和。电子能量的计算将在第 13 章至第 17 章中讨论。平动能是分子作为一个整体在容纳物质的容器空间中的运动动能。平动能级可视为三维空间中粒子的能级(第 3.5 节)。二原子(双原子)分子的转动能级可以用刚性转子的转动能级来近似表示,第 6.4 节将对此进行讨论。双原子分子的最低几个振动能级可以用谐振子的能级来近似表示(第 4.3 节),多原子分子的振动能可以用线性谐振子振动能之和来近似表示(第 15.12 节)。本章第 4.2 节将求解谐振子的薛定谔方程。作为初步介绍,第 4.1 节讨论了求解微分方程的幂级数解法,该方法用于求解谐振子的薛定谔方程。第 4.3 节讨论分子振动。第 4.4 节介绍了求一维薛定谔方程特征值和特征函数的数值方法。
\section{微分方程的幂级数解}
\label{sec:4.1 Power-Series Solutions of Differential Equations}
    到目前为止,我们只考虑了势能$V\left(x\right)$是常数的情况。这使得薛定谔方程成为一个具有常数系数的二阶线性齐次微分方程,我们知道如何求解这个方程。对于 $V$ 随 $x$ 变化的情况,一种有用的方法是尝试薛定谔方程的幂级数解法。\\
    \indent 为了说明这种方法,考虑微分方程
    \begin{equation}
        y^{\prime\prime}\left(x\right) + c^2 y\left(x\right) = 0
        \label{eq:4.1}
    \end{equation}
    其中$c^2>0$。当然,这个微分方程有常系数,但我们可以用幂级数法求解。让我们先用特征方程$s^2+c^2=0$来求解这个方程。特征方程的根是$s=\pm \mathrm{i}c$。回忆第 2.2 节中的工作[公式 (\ref{eq:2.10 schrödinger equation for particle between 0 and l}) 和 (\ref{eq:4.1}) 是相同的],当特征方程的根是纯虚数时,我们会得到三角函数形式的通解:
    \begin{equation}
        y\left(x\right) = A \cos\: cx + B \sin\: cx
        \label{eq:4.2}
    \end{equation}
    其中 $A$ 和 $B$ 是积分常量。式(\ref{eq:4.2})的另一种形式是
    \begin{equation}
        y = D \sin\left(cx+e\right)
        \label{eq:4.3}
    \end{equation}
    其中 $D$ 和 $e$ 是任意常数。利用两角之和的正弦公式,我们可以证明 (\ref{eq:4.3}) 等价于 (\ref{eq:4.2})。\\
    \indent 现在让我们用幂级数法来求解 (\ref{eq:4.1})。我们假设方程的解可以在$x=0$处展开成Taylor级数的形式(见问题:4.1),也就是说,我们假设
    \begin{equation}
        y\left(x\right) = \sum_{n=0}^{\infty}a_nx^n = a_0 + a_1 x + a_2 x^2 + a_3 x^3 + \cdots
        \label{eq:4.4}
    \end{equation}
    其中 $a$ 为常数系数,有待确定,以使式 (\ref{eq:4.1}) 成立。对式(\ref{eq:4.4})求导,有
    \begin{equation}
        y^{\prime}\left(x\right) = a_1 + 2a_2 x + 3a_3 x^2 + \cdots = \sum_{n=1}^{\infty}na_nx^{n-1}
        \label{eq:4.5}
    \end{equation}
    其中,我们假设该数列可以逐项求导(对于无穷级数来说,这并不总是正确的)。对于$y^{\prime\prime}$,我们有
    \begin{equation}
        y^{\prime\prime}\left(x\right) = 2a_2 + 3\left(2\right)a_3 x + \cdots = \sum_{n=2}^{\infty}n(n-1)a_nx^{n-2}
        \label{eq:4.6}
    \end{equation}
    将 (\ref{eq:4.6}) 和 (\ref{eq:4.4}) 代入 (\ref{eq:4.1}),得到
    \begin{equation}
        \sum_{n=2}^{\infty}n(n-1)a_nx^{n-2} + c^2\sum_{n=0}^{\infty}a_nx^n = 0
        \label{eq:4.7}
    \end{equation}
    我们想把 (\ref{eq:4.7}) 中的两个和合并起来。只要满足一定条件,我们就可以将两个无穷级数逐项相加,得到它们的和:
    \begin{equation}
        \sum_{j=0}^{\infty}b_jx^j + \sum_{j=0}^{max}c_jx^j = \sum_{j=0}^{\infty}(b_j+c_j)x^j
        \label{eq:4.8}
    \end{equation}
    要将 (\ref{eq:4.8}) 应用于 (\ref{eq:4.7}) 中的两个和,我们希望每个和的极限相同,$x$ 的幂相同。因此,我们改变了 (\ref{eq:4.7}) 中第一个和的下标,定义$k \equiv n-2$。极限$n = 2$到无穷大与$k = 0$到无穷大相对应,将$n=k+2$代入第一个和,有
    \begin{equation}
        \sum_{n=2}^{\infty}n\left(n-1\right)a_nx^{n-2} = \sum_{k=0}^{\infty}\left(k+2\right)\left(k+1\right)a_{k+2}x^k = \sum_{n=0}^{\infty}\left(n+2\right)\left(n+1\right)a_{n+2}x^n
        \label{eq:4.9}
    \end{equation}
    (\ref{eq:4.9}) 中的最后一个等式是有效的,因为求和指数是一个\textbf{虚拟变量}(dummy variable);我们用什么字母来表示这个变量没有区别。例如,求和$\sum_{i=1}^{3}c_ix^i$和$\sum_{m=1}^{3}c_mx^m$是相等的,因为它们只有求和指数不同。如果我们将这个求和展开就会更好理解:
    \begin{equation*}
        \sum_{i=1}^{3}c_ix_6 = c_1x^1 + c_2x^2 + c_3x^3, \quad \sum_{m=1}^{3}c_mx^m = c_1x^1 + c_2x^2 + c_3x^3
    \end{equation*}
    在 (\ref{eq:4.9}) 的最后一个等式中,我们只是将表示求和指数的符号从 $k$ 改为 $n$。定积分中的积分变量也是一个虚拟变量,因为定积分的值不受变量字母的影响:
    \begin{equation}
        \int_{a}^{b}f\left(x\right)dx = \int_{a}^{b}f\left(t\right)dt
        \label{eq:4.10}
    \end{equation}
    将式(\ref{eq:4.9}) 代入 (\ref{eq:4.7}),我们得到
    \begin{equation}
        \sum_{n=0}^{\infty}\left[\left(n+2\right)\left(n+1\right)a_{n+2}+c^2a_n\right]x^n = 0
        \label{eq:4.11}
    \end{equation}
    如果 (\ref{eq:4.11}) 对所有 $x$ 值都成立,那么 $x$ 的每个幂的系数都必须为零。为了说明这一点,请看方程
    \begin{equation}
        \sum_{j=0}^{\infty}b_jx^j = 0
        \label{eq:4.12}
    \end{equation}
    在式(\ref{eq:4.12}),令$x=0$,我们有$b_0=0$。将式(\ref{eq:4.12})左右两侧对$x$求导,则$b_1=0$。求$n$阶导数,令$x=0$,我们得到$b_n=0$。因此,从式(\ref{eq:4.11})中,我们有:
    \begin{equation}
        \left(n+2\right)\left(n+1\right)a_{n+2} + c^2 a_n = 0
        \label{eq:4.13}
    \end{equation}
    \begin{equation}
        a_{n+2} = -\frac{c^2}{\left(n+1\right)\left(n+2\right)}a_n
        \label{eq:4.14}
    \end{equation}
    方程(\ref{eq:4.14}) 是一个\textbf{递推关系式}(recursion relation),如果我们知道$a_0$的值,就能用式(\ref{eq:4.14})计算出$a_2$、$a_4$、$a_6$等的值。如果我们知道$a_1$的值,就能用式(\ref{eq:4.14})计算出$a_3$、$a_5$、$a_7$等的值。由于对$a_0$和$a_1$的选择没有限制,它们是任意常数,我们分别用$A$和$Bc$来表示:
    \begin{equation}
        a_0 = A, \quad a_1 = Bc
        \label{eq:4.15}
    \end{equation}
    则利用式(\ref{eq:4.14}),我们可以计算系数:
    \begin{equation*}
        a_0 = A, \quad a_2 = -\frac{c^2}{1 \cdot 2}A, \quad a_4 = \frac{c^4}{1 \cdot 2\cdot 3}A, \quad a_6 = -\frac{c^6}{6!}A, \quad \ldots
    \end{equation*}
    \begin{equation}
        a_{2k} = (-1)^k \frac{c^{2k}}{(2k)!}A, \quad k = 0, 1, 2, \ldots
        \label{eq:4.16}
    \end{equation}
    \begin{equation*}
        a_1 = Bc, \quad a_3 = -\frac{c^3}{2 \cdot 3}Bc, \quad a_5 = \frac{c^5}{2 \cdot 3 \cdot 4 \cdot 5}Bc, \quad a_7 = -\frac{c^7}{7!}Bc, \quad \ldots
    \end{equation*}
    \begin{equation}
        a_{2k+1} = (-1)^k \frac{c^{2k+1}}{(2k+1)!}B, \quad k = 0, 1, 2, \ldots
        \label{eq:4.17}
    \end{equation}
    根据式(\ref{eq:4.4})、 (\ref{eq:4.16}) 和 (\ref{eq:4.17}) 我们有
    \begin{equation}
        y = \sum_{n=0}^{\infty}a_nx^n = \sum_{n=0,2,4, \cdots}^{\infty}a_nx^n + \sum_{n=1,3,5, \cdots}^{\infty}a_nx^n
        \label{eq:4.18}
    \end{equation}
    \begin{equation}
        y = A\sum_{k=0}^{\infty}(-1)^k\frac{c^{2k}x^{2k}}{(2k)!} + B\sum_{k=0}^{\infty}(-1)^k\frac{c^{2k+1}x^{2k+1}}{(2k+1)!}
        \label{eq:4.19}
    \end{equation}
    (\ref{eq:4.19}) 中的两个级数是 $\cos\: cx$ 和 $\sin\: cx$ 的Taylor级数(问题:4.2)。因此,与式(\ref{eq:4.2}) 相同,我们有$y = \cos\: cx + \sin\: cx$。

\section{一维谐振子}
\label{sec:4.2 The}

\section{双原子分子的振动}
\label{sec:4.3 Vibrations of Diatomic Molecules}

\section{一维定态薛定谔方程的数值解法}
\label{sec:4.4 Numerical Solutions of the One-Dimensional Time-Independent Schrödinger Equation}

\section*{总结}

\section*{习题}
