% ===== CHAPTER 2 =====
\chapter{箱中粒子}
	单粒子一维系统的定态波函数和能级是通过求解定态薛定谔方程 (\ref{eq:1.19 time-independent Schrödinger Equation}) 得到的。在本章中,我们将求解一个简单系统的定态薛定谔方程,即一维盒子中的粒子(第 2.2 节)。由于薛定谔方程是微分方程,我们首先讨论微分方程。
\section{微分方程}
	本节只讨论常微分方程,即只有一个自变量的方程。[偏微分方程有不止一个自变量。含时薛定谔方程 (\ref{eq:1.16 Time-dependent Schödinger equation}) 就是一个例子,其中 $t$ 和 $x$ 都是自变量。] 
	常微分方程是自变量 $x$、因变量$y\left(x\right)$和$y$的1、2、$\cdots$、$n$阶导数$y\left(y^{\prime},y^{\prime\prime}, \cdots, y^{\left(n\right)}\right)$之间的关系式。例如
	\begin{equation}
		y^{\prime\prime\prime}+2x\left(y^{\prime}\right)^2+y^2\sin x = 3\mathrm{e}^x
		\label{eq:2.1 an example of differential equation}
	\end{equation}
	微分方程的阶数与方程中最高阶导数的阶数相同。因此,方程 (\ref{eq:2.1 an example of differential equation}) 为三阶微分方程。\\
	\indent \textbf{线性微分方程}是一种特殊的微分方程,其形式为
	\begin{equation}
		A_n\left(x\right)y^{\left(n\right)}+A_{n-1}\left(x\right)y^{\left(n-1\right)}+A_{n-2}\left(x\right)y^{\left(n-2\right)}+\cdots+A_1\left(x\right)y^{\prime}+A_0\left(x\right)y=g\left(x\right)
		\label{eq:2.2 linear differential equation}
	\end{equation}
	其中所有的$A_i$和$g$ (有些可能为零)只是$x$的函数。在$n$阶线性微分方程(\ref{eq:2.2 linear differential equation})中,$y$及其各阶导数的次幂均为一次。不满足式(\ref{eq:2.2 linear differential equation})的微分方程为\textbf{非线性}微分方程。如果(\ref{eq:2.2 linear differential equation})中的$g\left(x\right)=0$,则线性微分方程是\textbf{齐次}的,否则是\textbf{非齐次}的。一维定态薛定谔方程(\ref{eq:1.19 time-independent Schrödinger Equation})就是二阶线性齐次微分方程。\\
	\indent 通过除以 $y^{\prime \prime}$ 的系数,我们可以将每一个线性齐次二阶微分方程化为以下形式
	\begin{equation}
		y^{\prime\prime}+P\left(x\right)y^{\prime}+Q\left(x\right)y=0
		\label{eq:2.3 linear homogeneous differential equation}
	\end{equation}
	设$y_1$和$y_2$是两个满足方程(\ref{eq:2.3 linear homogeneous differential equation})的独立函数。“独立”的意思是$y_2$并不是$y_1$的简单倍数。因此,线性齐次微分方程(\ref{eq:2.3 linear homogeneous differential equation})的通解为
	\begin{equation}
		y = c_1y_1+c_2y_2
		\label{eq:2.4 general solution of the li_{}near homogeneous differential equation}
	\end{equation}
	其中$c_1$和$c_2$是任意常数。将式(\ref{eq:2.4 general solution of the li_{}near homogeneous differential equation})带入(\ref{eq:2.3 linear homogeneous differential equation})的左边即可轻松证明:
	\begin{equation}
		\begin{aligned}
			c_1 & y_1^{\prime\prime}+c_2y_2^{\prime\prime}+P\left(x\right)c_1y_1^{\prime}+P\left(x\right)c_2y_2^{\prime}+Q\left(x\right)c_1y_1+Q\left(x\right)c_2y_2\\
			& = c_1\left[	y_1^{\prime\prime}+P\left(x\right)y_1^{\prime}+Q\left(x\right)y_1\right]+c_2\left[	y_2^{\prime\prime}+P\left(x\right)y_2^{\prime}+Q\left(x\right)y_2\right]\\
			& = c_1 \cdot 0 + c_2 \cdot 0 = 0
		\end{aligned}
		\label{eq:2.5}
	\end{equation}
	其中我们用到了$y_1$和$y_2$是方程(\ref{eq:2.3 linear homogeneous differential equation})的解。\\
	\indent $n$ 次微分方程的通解通常有 $n$ 个任意常数。为了确定这些常数,我们可能需要\textbf{定解条件},即规定 $y$ 或其各阶导数在某一点或多点的值的条件。例如:如果 $y$ 是固定在两点上的振动弦的位移,我们就知道 $y$ 在这两点上必须为零。\\
	\indent 一个重要的特例是\textbf{常系数}线性齐次二阶微分方程:
	\begin{equation}
		y^{\prime\prime}+py^{\prime}+qy=0
		\label{eq:2.6 constant coefficients in linear homogeneous second-order differential equation}
	\end{equation}
	其中$p$与$q$是常数。为了解这个方程(\ref{eq:2.6 constant coefficients in linear homogeneous second-order differential equation}),我们暂且假定解的形式是$y = \mathrm{e}^{sx}$。我们正在寻找一个导数与常数相乘会与原函数相互抵消的函数。指数函数在微分时会重复,因此是正确的选择。将其代入 (\ref{eq:2.6 constant coefficients in linear homogeneous second-order differential equation}) 即可得出
	\begin{equation*}
		s^2\mathrm{e}^{sx}+ps\mathrm{e}^{sx}+q\mathrm{e}^{sx}=0
	\end{equation*}
	\begin{equation}
		\boxed{
			s^2+ps+q=0
		}
		\label{eq:2.7 auxiliary equation}
	\end{equation}
	方程 (\ref{eq:2.7 auxiliary equation}) 称为\textbf{特征方程}。它是一个有两个根 $s_1$ 和 $s_2$ 的一元二次方程,只要 $s_1$ 和 $s_2$ 不相等,就能给出 (\ref{eq:2.6 constant coefficients in linear homogeneous second-order differential equation}) 的两个独立解。因此,(\ref{eq:2.6 constant coefficients in linear homogeneous second-order differential equation})的通解为
	\begin{equation}
		\boxed{
			y = c_1\mathrm{e}^{s_1x}+c_2\mathrm{e}^{s_2x}
		}
		\label{eq:2.8 general solution for 2.6}
	\end{equation}
	例如,对于方程$y^{\prime\prime}+6y^{\prime}-7y=0$,其特征方程为$s^2+6s-7=0$。解这个特征方程,有$s_1=1$,$s_2=-7$,所以通解为$y = c_1\mathrm{e}^{x}+c_2\mathrm{e}^{-7x}$。
	

\section{一维盒子中的粒子}
	

\section{一维自由粒子}

\section{矩形势井中的粒子}

\section{隧穿}

\section*{总结}

\section*{习题}
