% ===== CHAPTER 8 =====
\chapter{变分法}
\label{chap:8}
\section{变分理论}
\label{sec:8.1 The Variation Theorem}
    为了处理包含相互作用粒子系统(如原子和分子)的定态薛定谔方程,我们必须使用近似方法。本章讨论变分法,它允许我们在不求解薛定谔方程的情况下近似得到系统的基态能量。变分法基于如下理论:
    \begin{center}
        \parbox{0.8\textwidth}{
            \textbf{变分理论:}
            
            给定一个系统,其哈密顿算符$\hat{H}$与时间无关,其最低能量本征值为$E_1$,如果 $\phi$ 是系统粒子坐标的任何满足问题定解条件的已归一化品优函数,那么
            \begin{equation}
                \boxed{
                    \int \phi^{\ast}\hat{H}\:\mathrm{d}\tau \geq E_1, \quad \phi \text{已归一化}
                }
                \label{eq:8.1}
            \end{equation}
        }
    \end{center}

    变分理论允许我们计算系统基态能量的最大值。

    为了证明(\ref{eq:8.1}),我们将$\phi$用$\hat{H}$的本征函数完全正交集,即定态本征函数$\psi_k$展开:
    \begin{equation}
        \phi = \sum_k a_k \psi_k
        \label{eq:8.2}
    \end{equation}
    其中
    \begin{equation}
        \hat{H}\psi_k = E_k \psi_k
        \label{eq:8.3}
    \end{equation}
    注意:展开式(\ref{eq:8.2})要求$\phi$满足与$\psi_k$相同的定解条件。将(\ref{eq:8.2})代入(\ref{eq:8.1})的左侧,我们得到
    \begin{equation*}
        \int \phi^{\ast}\hat{H}\phi\:\mathrm{d}\tau = \int \sum_{k}a_k^{\ast} \psi_k^{\ast} \hat{H} \sum_{j} a_j \psi_j \:\mathrm{d}\tau = \int \sum_{k}a_k^{\ast} \psi_k^{\ast} \sum_{j} a_j \hat{H} \psi_j \:\mathrm{d}\tau
    \end{equation*}
    利用本征方程 (\ref{eq:8.3}),并假设交换积分和无限求和是成立的,我们可以得到
    \begin{equation*}
        \begin{aligned}
            \int \phi^{\ast}\hat{H}\phi \:\mathrm{d}\tau &= \int \sum_{k} a_k^{\ast} \psi_k^{\ast} \hat{H} \sum_{j} a_j \psi_j \:\mathrm{d}\tau = \int \sum_{k} a_k^{\ast} \psi_k^{\ast} \sum_{j} a_j \hat{H} \psi_j \:\mathrm{d}\tau \\
            &= \sum_k\sum_{j} a_k^{\ast} a_j E_j \delta_{kj}
        \end{aligned}
    \end{equation*}
    其中用到了本征函数$\psi_k$的正交性。我们对 $j$ 求和,像往常一样,除了 $j=k$ 的项之外,Kronecker $\delta$使所有项都为零,从而得到
    \begin{equation}
        \int \phi^{\ast} \hat{H}\phi \:\mathrm{d}\tau = \sum_{k}a_k^{\ast} a_k E_k = \sum_k |a_k|^2 E_k
        \label{eq:8.4}
    \end{equation}
    
    由于$E_1$是$\hat{H}$的最低能量本征值,我们有$E_k \geq E_1$,又由于$\left|a_k\right|^2$为非负数,我们可以将不等式$E_k \geq E_1$的左右两侧乘以$|a_k|^2$而不改变不等号的方向,从而得到$|a_k|^2 E_k \geq |a_k|^2 E_1$。因此,$\sum_k |a_k|^2 E_k \geq \sum_k |a_k|^2 E_1$,使用(\ref{eq:8.4}),我们得到
    \begin{equation}
        \int \phi^{\ast} \hat{H}\phi \:\mathrm{d}\tau = \sum_k |a_k|^2 E_k \geq \sum_k |a_k|^2 E_1 = E_1 \sum_k |a_k|^2
        \label{eq:8.5}
    \end{equation}

    因为$\phi$已归一化,我们有$\int \phi^{\ast} \phi \: \mathrm{d}\tau = 1$。将展开式(\ref{eq:8.2})代入归一化条件,得到
    \begin{equation*}
        1 = \int \phi^{\ast} \phi \: \mathrm{d}\tau = \int \sum_{k} a_k^{\ast} \psi_k^{\ast} \sum_{j} a_j \psi_j \: \mathrm{d}\tau = \sum_{k} \sum_{j} a_k^{\ast} a_j \int \psi_k^{\ast} \psi_j \: \mathrm{d}\tau = \sum_{k} \sum_{j} a_k^{\ast} a_j \delta_{kj}
    \end{equation*}
    \begin{equation}
        1 = \sum_{k} |a_k|^2
        \label{eq:8.6}
    \end{equation}
    [注意,在推导公式 (\ref{eq:8.4}) 和 (\ref{eq:8.6}) 时,我们基本上分别重复了公式 (\ref{eq:7.70}) 和 (\ref{eq:7.69}) 的推导。]

    将(\ref{eq:8.6}) 代入 (\ref{eq:8.5})就得到了变分理论(\ref{eq:8.1})的证明:
    \begin{equation}
        \int \phi^{\ast} \hat{H}\phi \:\mathrm{d}\tau \geq E_1, \quad \phi \text{已归一化}
        \label{eq:8.7}
    \end{equation}

    假设我们有一个未归一化的函数$\phi$。为了应用变分理论,我们用一个归一化常数$N$乘以$\phi$使得$N\phi$满足归一化条件。在(\ref{eq:8.7})中,我们将$\phi$替换为$N\phi$,得到
    \begin{equation}
        \left|N\right|^2 \int \phi^{\ast} \hat{H}\phi \:\mathrm{d}\tau \geq E_1
        \label{eq:8.8}
    \end{equation}
    $N$由下式导出:$\int \left(N\phi\right)^{\ast} \left(N\phi\right) \:\mathrm{d}\tau = \left|N\right|^2 \int \phi^{\ast} \phi \:\mathrm{d}\tau = 1$,因此$\left|N\right|^2 = 1/\int \phi^{\ast} \phi \:\mathrm{d}\tau$。将其代入(\ref{eq:8.8}),我们得到
    \begin{equation}
        \boxed{
            \frac{\int \phi^{\ast} \hat{H}\phi \:\mathrm{d}\tau}{\int \phi^{\ast} \phi \:\mathrm{d}\tau} \geq E_1
        }
        \label{eq:8.9}
    \end{equation}
    其中$\phi$是满足问题定解条件的任意品优函数(不必归一化)。

    函数$\phi$称为\textbf{试探变分函数}(trial variational function),(\ref{eq:8.1})中的积分[或 (\ref{eq:8.9}) 中的积分之比]称为\textbf{变分积分}(variational integral)。为了获取基态能量$E_1$的良好近似值,我们会尝试许多试探变分函数,并寻找能给出最小变分积分值的函数。由 (\ref{eq:8.1}) 可知,变分积分的值越小,我们对 $E_1$ 的近似就越好。推翻量子力学的一种方法是找到一个试探变分函数,使已知 $E_1$ 的某个系统的变分积分小于 $E_1$。

    令$\psi_1$为系统基态能量的真实波函数:
    \begin{equation}
        \hat{H}\psi_1 = E_1 \psi_1
        \label{eq:8.10}
    \end{equation}
    如果我们碰巧幸运地发现了一个等于 $\psi_1$ 的变分函数,那么在 (\ref{eq:8.1}) 中使用 (\ref{eq:8.10}) ,我们可以看到变分积分将等于 $E_1$。因此,基态波函数给出了变分积分的最小值。因此,我们预期变分积分值越小,试探变分函数就越接近真实的基态波函数。然而,事实证明,变分积分接近 $E_1$ 的速度要比试探变分函数接近 $\psi_1$ 的速度快得多,而且有可能用一个相当差的 $\phi$ 得到一个相当好的近似值 $E_1$。

    在实践中,人们通常会在试探函数 $\phi$ 中引入几个参数,然后对这些参数进行变分,从而使变分积分最小化。能否成功使用变分法,取决于能否对试探函数做出正确的选择。

    让我们来看一些变分法的例子。虽然这种方法的真正用途是针对我们不知道确切解的问题,但我们还是要考虑那些完全可解的问题,以便判断我们结果的准确性。

    \begin{examplebox}
        \textbf{例题:}为长度为 $l$ 的一维势箱中的粒子设计一个试探变分函数。
        \\

        波函数在势箱外为零,定解条件要求在$x=0$和$x=l$处$\psi=0$。变分函数$\phi$必须满足这些在边界点为零的条件。如公式 (\ref{eq:4.57}) 后所述,基态 $\psi$ 在边界点内部没有节点,因此希望 $\phi$ 也没有内部节点。具有这些性质的一个简单函数为抛物线函数
        \begin{equation}
            \phi = x\left(l-x\right), \quad 0 \leq x \leq l
            \label{eq:8.11}
        \end{equation}
        以及在势箱外,有$\phi=0$。由于我们尚未归一化$\phi$,我们使用(\ref{eq:8.9})来计算。势箱内部的哈密顿算符为$-\left(\hbar^2/2m\right)\mathrm{d}^2/\mathrm{d}x^2$。对于 (\ref{eq:8.9}) 的分子和分母,我们可以得出
        \begin{equation}
            \int \phi^{\ast} \hat{H}\phi \:\mathrm{d}\tau = -\frac{\hbar^2}{2m} \int_{0}^{l} \left(lx-x^2\right)\frac{\mathrm{d}^2}{\mathrm{d}x^2} \left(lx-x^2\right) \mathrm{d}x = \frac{\hbar^2}{m} \int_0^l \left(lx-x^2\right) \:\mathrm{d}x = \frac{\hbar^2 l^3}{6m}
            \label{eq:8.12}
        \end{equation}
        \begin{equation}
            \int \phi^{\ast} \phi \:\mathrm{d}\tau = \int_0^l x^2\left(l-x\right)^2 \:\mathrm{d}x = \frac{l^5}{30}
            \label{eq:8.13}
        \end{equation}
        将它们带入变分理论(\ref{eq:8.9}),我们得到
        \begin{equation*}
            E_1 \leq \frac{5h^2}{4\pi^2 ml^2} = 0.1266515\frac{\hbar^2}{ml^2}
        \end{equation*}
        根据(\ref{eq:2.20 energy of one-dimensional box}),$E_1$的准确值为$h^2/8ml^2=0.125h^2/8ml^2$,能量的误差约为$1.3\%$。

        由于$\int \left|\phi\right|^2 \: \mathrm{d}\tau = l^5/30$,将(\ref{eq:8.11})归一化为$\left(30/l^5\right)^{1/2} x\left(l-x\right)$。图\ref{fig:7.3}显示,该函数与真正的箱中粒子基态波函数非常相似。
        \\

        \textbf{练习:}一个单粒子一维系统的势能函数满足
        \begin{equation*}
            V = \begin{cases}
                V_0, & 0 \leq x \leq l \\
                \infty, & elsewhere
            \end{cases}
        \end{equation*}
        (其中$V_0$是常数)。

        (a)使用试探函数
        \begin{equation*}
            \phi = \begin{cases}
                \sin\left(\pi x/l\right), & 0 \leq x \leq l \\
                0, & elsewhere
            \end{cases}
        \end{equation*}
        来估算该系统的基态能量;

        (b)解释为什么(a)的结果是该系统的准确基态能量。\textit{提示:见第\ref{chap:4}章的问题之一。}(答案:(a)$V_0 + h^2/8ml^2$。)
    \end{examplebox}

    上一个例题的试探函数中没有参数。下一个例题有。

    \begin{examplebox}
        \textbf{例题:}对于一维谐振子,设计一个带有参数的变分函数,并找出参数的最佳值。估计基态能量。
        \\

        变分函数$\phi$必须平方可积,即当$x \to \pm \infty$时,$\phi \to 0$。函数$\mathrm{e}^{-x}$在$+ \infty$时满足条件,但在$-\infty$时变为无穷大。函数$\mathrm{e}^{-x^2}$在$\pm \infty$时都满足条件。然而,从量纲上看,这种做法并不令人满意,因为$\mathrm{e}$的指数必须是无量纲的。这可以从Taylor级数中看出$\mathrm{e}^z = 1 + z + z^2/2! + z^3/3! + \ldots$[式(\ref{eq:4.44})]。由于该级数中的所有项都必须有相同的量纲,那么$z$的量纲就必须为1,也就是说$\mathrm{e}^z$中的$z$必须是无量纲的。因此,我们将$\mathrm{e}^{-x^2}$替换为$\mathrm{e}^{-cx^2}$,其中$c$的量纲为长度$^{-2}$。我们将$c$作为变分参数。真实的基态$\psi$没有节点。此外,由于$V = \frac{1}{2}kx^2$是偶函数,由于奇函数在原点有节点,那么基态的$\psi$一定具有确定的宇称,也一定是偶函数。试探函数$\mathrm{e}^{-cx^2}$具有我们需要的性质:无节点,且为偶函数。

        对$\hat{H}$使用(\ref{eq:4.30}),再使用附录中的积分,有(问题8.3)
        \begin{equation*}
            \begin{aligned}
                \int \phi^{\ast} \hat{H}\phi \:\mathrm{d}\tau &= -\frac{\hbar^2}{2m} \int_{-\infty}^{\infty} \mathrm{e}^{-2cx^2} \frac{\mathrm{d}^2\mathrm{e}^{-cx^2}}{\mathrm{d}x^2} \:\mathrm{d}x + 2\pi^2\nu^2m\int_{-\infty}^{\infty} x^2 \mathrm{e}^{-2cx^2} \:\mathrm{d}x \\
                &= \frac{\hbar^2}{m} \left(\frac{\pi c}{m}\right)^{1/2} + \nu^2m\left(\frac{\pi^5}{8}\right)^{1/2} c^{-3/2}
            \end{aligned}
        \end{equation*}
        \begin{equation*}
            \int \phi^{\ast} \phi \:\mathrm{d}\tau = \int_{-\infty}^{\infty} \mathrm{e}^{-2cx^2} \:\mathrm{d}x = 2\int_{0}^{\infty} \mathrm{e}^{-2cx^2} \:\mathrm{d}x = \left(\frac{\pi}{2c}\right)^{1/2}
        \end{equation*}
        试探积分$W$为
        \begin{equation}
            W \equiv \frac{\int \phi^{\ast} \hat{H}\phi \:\mathrm{d}\tau}{\int \phi^{\ast} \phi \:\mathrm{d}\tau} = \frac{\hbar^2c}{2m} + \frac{\pi^2\nu^2m}{2c}
            \label{eq:8.14}
        \end{equation}
        现在,我们通过改变$c$来最小化$W$(\ref{eq:8.14})。$W$最小的必要条件为
        \begin{equation*}
            \frac{\mathrm{d} W}{\mathrm{d} c} = 0 = \frac{\hbar^2}{2m} - \frac{\pi^2\nu^2m}{2c^2}
        \end{equation*}
        \begin{equation}
            c = \pm \pi \nu m/\hbar
            \label{eq:8.15}
        \end{equation}
        其中,负根$c = -\pi\nu m/\hbar$应舍去,因其会导致$\phi = \mathrm{e}^{-cx^2}$不满足平方可积的要求。将$c = \pi\nu m/\hbar$代入(\ref{eq:8.14}),我们得到$W = \frac{1}{2}h\nu$。这是一维谐振子基态能量的准确值。当$c = \pi\nu m/\hbar$时,变分函数$\phi$除了未归一化以外,与谐振子基态波函数(\ref{eq:4.53})和(\ref{eq:4.31})完全一致。

        对于归一化的谐振子变分函数$\phi = \left(2c/\pi\right)^{1/4} \mathrm{e}^{-cx^2}$,若$c$的值很大,会导致$\phi$在$x=0$附近从其最大值快速下降。那么概率密度就只在$x=0$附近较大。势函数$V = \frac{1}{2}kx^2$在$x=0$附近也很小,所以较大的$c$值意味着较小的$\left\langle V \right\rangle = \left\langle \phi \middle| V \middle| \phi \right\rangle$。[注意:$\left\langle V \right\rangle$等于(\ref{eq:8.14})右侧的第二项。]然而,由于较大的$c$值使得$\phi$从其最大值快速减小,那么在$x=0$附近$\left|\mathrm{d}\phi/\mathrm{d}x\right|$的值变得很大,根据问题7.7b,大$\left|\mathrm{d}\phi/\mathrm{d}x\right|$意味着大$\left\langle T \right\rangle$[等于(\ref{eq:8.14})右侧的第一项。]$c$的最佳值可以使总和$\left\langle T \right\rangle + \left\langle V \right\rangle = W$最小化。在原子和分子中,真正的波函数是通过将电子限制在低$V$区域(靠近原子核)来最小化$\left\langle V \right\rangle$的趋势与通过允许电子概率密度在大区域内扩散来最小化$\left\langle T \right\rangle$的趋势之间的折衷。
        \\

        \textbf{练习:}考虑一个单粒子一维系统,其势函数满足
        \begin{equation*}
            V = \begin{cases}
                V = 0, & -\frac{1}{2}l \leq x \leq \frac{1}{2}l \\
                V = b\hbar^2/ml^2, & elsewhere
            \end{cases}
        \end{equation*}
        其中$b$是正常数(图 \ref{fig:2.5} 中的 $V_0 = b\hbar^2/ml^2$ 和原点发生了偏移)。

        (a)对于变分函数
        \begin{equation*}
            \phi = \begin{cases}
                \left(x-c\right)\left(x+c\right) = x^2 - c^2, & -c \leq x \leq c \\
                0, & elsewhere
            \end{cases}
        \end{equation*}
        其中变分参数$c > \frac{1}{2}l$。我们发现变分积分$W$由下式给出:
        \begin{equation*}
            W = \frac{\hbar^2}{ml^2}\left[\frac{5l^2}{4c^2} + b\left(1-\frac{15l}{16c} + \frac{5l^3}{32c^3} - \frac{3l^5}{256c^5}\right)\right]
        \end{equation*}
        在同一幅图上绘制$\phi$和$V$的图象。求使 $W$ 最小的 $c$ 值所满足的方程。

        (b)当$V_0 = 20\hbar^2/ml^2$时,求出$c$的最佳值和$W$的值,将其与基态能量真实值$2.814\hbar^2/ml^2$进行比较(问题4.31c)。(\textit{提示:}可以使用Excel表格求解器或可编程计算器来计算$c/l$的值。)

        (\textit{答案:}(a)$48t^4-24t^2-128t^3/b+3=0$,其中$t \equiv c/l$;(b)$c = 0.6715l$,$W = 3.454\hbar^2/ml^2$。)
    \end{examplebox}

\section{变分法的推广}
\label{sec:8.2 Extension of the Variational Method}
    上一节介绍的变分法只提供了基态能量和波函数的信息。现在我们讨论将变分法推广到激发态。(另请参见第 \ref{sec:8.5 Linear Variational Functions} 节)。

    考虑一下我们可以如何推广变分法,来估算第一激发态的能量。根据能量从低到高:
    \begin{equation*}
        E_1 \leq E_2 \leq E_3 \leq \ldots
    \end{equation*}
    我们将系统的状态标号为$1,2,3,\ldots$。我们已经证明:对于一个已归一化的变分函数$\phi$,有[式(\ref{eq:8.4})和(\ref{eq:8.6})]
    \begin{equation*}
        \int \phi^{\ast} \hat{H}\phi \:\mathrm{d}\tau = \sum_{k=1}^{\infty} |a_k|^2 E_k \quad \text{和} \quad \int \phi^{\ast} \phi \:\mathrm{d}\tau = \sum_{k=1}^{\infty} |a_k|^2 = 1
    \end{equation*}
    其中的$a_k$是展开式$\phi = \sum_{k} a_k \psi_k$中的系数[式(\ref{eq:8.2})]。我们有$a_k = \left\langle \psi_k \middle| \phi \right\rangle$[式(\ref{eq:7.40})]。让我们只讨论与真实基态波函数$\psi$正交的归一化函数$\phi$。那么$a_1 = \left\langle \psi_1 \middle| \phi \right\rangle = 0$,因此
    \begin{equation}
        \int \phi^{\ast} \hat{H}\phi \:\mathrm{d}\tau = \sum_{k=2}^{\infty} |a_k|^2 E_k \quad \text{和} \quad \int \phi^{\ast} \phi \:\mathrm{d}\tau = \sum_{k=2}^{\infty} |a_k|^2 = 1
        \label{eq:8.16}
    \end{equation}
    对于$k \geq 2$,我们有$E_k \geq E_2$和$\left|a_k\right|^2E_k \geq \left|a_k\right|^2E_2$。因此,
    \begin{equation}
        \sum_{k=2}^{\infty} \left|a_k\right|^2 E_k \geq \sum_{k=2}^{\infty} \left|a_k\right|^2 E_2 = E_2 \sum_{k=2}^{\infty} \left|a_k\right|^2 = E_2
        \label{eq:8.17}
    \end{equation}
    将(\ref{eq:8.16})和(\ref{eq:8.17})相结合,我们得到了所需的结果:
    \begin{equation}
        \int \phi^{\ast} \hat{H}\phi \:\mathrm{d}\tau \geq E_2, \quad \text{若} \int \psi^{\ast} \phi \:\mathrm{d}\tau = 0 \quad \text{且} \int \phi^{\ast} \phi \:\mathrm{d}\tau = 1
        \label{eq:8.18}
    \end{equation}
    通过不等式 (\ref{eq:8.18}) 我们可以得到第一激发态的能量 $E_2$ 的上限。然而,限制条件$\left\langle \psi_1 \middle| \phi \right\rangle = 0$使得这种方法在应用上会很麻烦。

    对于特定的系统,尽管我们不知道真实基态波函数,但确定$\left\langle \psi_1 \middle| \phi \right\rangle = 0$是可能的。$V$ 是 $x$ 的偶函数的一维问题就是一个例子。在这种情况下,基态波函数总是偶函数,而第一激发态波函数是奇函数。(所有波函数必须具有确定的宇称。基态波函数是无节点的,由于奇函数在原点消失,所以基态波函数必须是偶函数。第一个激发态波函数有一个节点,一定是奇函数。)因此,对于奇函数,$\left\langle \psi_1 \middle| \phi \right\rangle = 0$一定成立,偶函数$\psi_1$乘以一个奇函数$\phi$得到一个奇被积函数,从$-\infty$到$+\infty$的积分为零。

    另一个例子是在中心场中运动的粒子(第 \ref{sec:6.1 The One-Particle Central Force Problem} 节)。势能的形式可能使我们无法求解本征函数中的径向因子 $R(r)$。然而,其中的角度因子 $\psi$ 是球谐函数[式 (\ref{eq:6.16}) ],而不同 $l$ 值的球谐函数是正交的。因此,我们可以利用试探函数中的因子$Y_l^m$,求出任意给定角动量$l$的最低态能量上限。这一结果取决于将 (\ref{eq:8.18}) 推广到更高的激发态:
    \begin{equation}
        \frac{\int \phi^{\ast} \hat{H}\phi \:\mathrm{d}\tau}{\int \phi^{\ast} \phi \:\mathrm{d}\tau} \geq E_{k+1}, \quad \text{若} \int \psi_1^{\ast} \phi \: \mathrm{d}\tau = \int \psi_2^{\ast} \phi \:\mathrm{d}\tau = \cdots = \int \psi_k^{\ast} \phi \:\mathrm{d}\tau = 0
        \label{eq:8.19}
    \end{equation}












\section{行列式}
\label{sec:8.3 Determinants}

\section{联立线性方程组}
\label{sec:8.4 Simultaneous Linear Equations}

\section{线性变分函数}
\label{sec:8.5 Linear Variational Functions}

\section{矩阵、本征值和本征向量}
\label{sec:8.6 Matrices, Eigenvalues, and Eigenvectors}

\section*{总结}

\section*{习题}