% ===== CHAPTER 8 =====
\chapter{变分法}
\label{chap:8}
\section{变分理论}
\label{sec:8.1 The Variation Theorem}
    为了处理包含相互作用粒子系统(如原子和分子)的定态薛定谔方程,我们必须使用近似方法。本章讨论变分法,它允许我们在不求解薛定谔方程的情况下近似得到系统的基态能量。变分法基于如下理论:
    \begin{center}
        \parbox{0.8\textwidth}{
            \textbf{变分理论:}
            
            给定一个系统,其哈密顿算符$\hat{H}$与时间无关,其最低能量本征值为$E_1$,如果 $\phi$ 是系统粒子坐标的任何满足问题定解条件的已归一化品优函数,那么
            \begin{equation}
                \boxed{
                    \int \phi^{\ast}\hat{H}\:\mathrm{d}\tau \geq E_1, \quad \phi \text{已归一化}
                }
                \label{eq:8.1}
            \end{equation}
        }
    \end{center}

    变分理论允许我们计算系统基态能量的最大值。

    为了证明(\ref{eq:8.1}),我们将$\phi$用$\hat{H}$的本征函数完全正交集,即定态本征函数$\psi_k$展开:
    \begin{equation}
        \phi = \sum_k a_k \psi_k
        \label{eq:8.2}
    \end{equation}
    其中
    \begin{equation}
        \hat{H}\psi_k = E_k \psi_k
        \label{eq:8.3}
    \end{equation}
    注意:展开式(\ref{eq:8.2})要求$\phi$满足与$\psi_k$相同的定解条件。将(\ref{eq:8.2})代入(\ref{eq:8.1})的左侧,我们得到
    \begin{equation*}
        \int \phi^{\ast}\hat{H}\phi\:\mathrm{d}\tau = \int \sum_{k}a_k^{\ast} \psi_k^{\ast} \hat{H} \sum_{j} a_j \psi_j \:\mathrm{d}\tau = \int \sum_{k}a_k^{\ast} \psi_k^{\ast} \sum_{j} a_j \hat{H} \psi_j \:\mathrm{d}\tau
    \end{equation*}
    利用本征方程 (\ref{eq:8.3}),并假设交换积分和无限求和是成立的,我们可以得到
    \begin{equation*}
        \begin{aligned}
            \int \phi^{\ast}\hat{H}\phi \:\mathrm{d}\tau &= \int \sum_{k} a_k^{\ast} \psi_k^{\ast} \hat{H} \sum_{j} a_j \psi_j \:\mathrm{d}\tau = \int \sum_{k} a_k^{\ast} \psi_k^{\ast} \sum_{j} a_j \hat{H} \psi_j \:\mathrm{d}\tau \\
            &= \sum_k\sum_{j} a_k^{\ast} a_j E_j \delta_{kj}
        \end{aligned}
    \end{equation*}
    其中用到了本征函数$\psi_k$的正交性。我们对 $j$ 求和,像往常一样,除了 $j=k$ 的项之外,Kronecker $\delta$使所有项都为零,从而得到
    \begin{equation}
        \int \phi^{\ast} \hat{H}\phi \:\mathrm{d}\tau = \sum_{k}a_k^{\ast} a_k E_k = \sum_k |a_k|^2 E_k
        \label{eq:8.4}
    \end{equation}
    
    由于$E_1$是$\hat{H}$的最低能量本征值,我们有$E_k \geq E_1$,又由于$\left|a_k\right|^2$为非负数,我们可以将不等式$E_k \geq E_1$的左右两侧乘以$|a_k|^2$而不改变不等号的方向,从而得到$|a_k|^2 E_k \geq |a_k|^2 E_1$。因此,$\sum_k |a_k|^2 E_k \geq \sum_k |a_k|^2 E_1$,使用(\ref{eq:8.4}),我们得到
    \begin{equation}
        \int \phi^{\ast} \hat{H}\phi \:\mathrm{d}\tau = \sum_k |a_k|^2 E_k \geq \sum_k |a_k|^2 E_1 = E_1 \sum_k |a_k|^2
        \label{eq:8.5}
    \end{equation}

    因为$\phi$已归一化,我们有$\int \phi^{\ast} \phi \: \mathrm{d}\tau = 1$。将展开式(\ref{eq:8.2})代入归一化条件,得到
    \begin{equation*}
        1 = \int \phi^{\ast} \phi \: \mathrm{d}\tau = \int \sum_{k} a_k^{\ast} \psi_k^{\ast} \sum_{j} a_j \psi_j \: \mathrm{d}\tau = \sum_{k} \sum_{j} a_k^{\ast} a_j \int \psi_k^{\ast} \psi_j \: \mathrm{d}\tau = \sum_{k} \sum_{j} a_k^{\ast} a_j \delta_{kj}
    \end{equation*}
    \begin{equation}
        1 = \sum_{k} |a_k|^2
        \label{eq:8.6}
    \end{equation}
    [注意,在推导公式 (\ref{eq:8.4}) 和 (\ref{eq:8.6}) 时,我们基本上分别重复了公式 (\ref{eq:7.70}) 和 (\ref{eq:7.69}) 的推导。]

    将(\ref{eq:8.6}) 代入 (\ref{eq:8.5})就得到了变分理论(\ref{eq:8.1})的证明:
    \begin{equation}
        \int \phi^{\ast} \hat{H}\phi \:\mathrm{d}\tau \geq E_1, \quad \phi \text{已归一化}
        \label{eq:8.7}
    \end{equation}

    假设我们有一个未归一化的函数$\phi$。为了应用变分理论,我们用一个归一化常数$N$乘以$\phi$使得$N\phi$满足归一化条件。在(\ref{eq:8.7})中,我们将$\phi$替换为$N\phi$,得到
    \begin{equation}
        \left|N\right|^2 \int \phi^{\ast} \hat{H}\phi \:\mathrm{d}\tau \geq E_1
        \label{eq:8.8}
    \end{equation}
    $N$由下式导出:$\int \left(N\phi\right)^{\ast} \left(N\phi\right) \:\mathrm{d}\tau = \left|N\right|^2 \int \phi^{\ast} \phi \:\mathrm{d}\tau = 1$,因此$\left|N\right|^2 = 1/\int \phi^{\ast} \phi \:\mathrm{d}\tau$。将其代入(\ref{eq:8.8}),我们得到
    \begin{equation}
        \boxed{
            \frac{\int \phi^{\ast} \hat{H}\phi \:\mathrm{d}\tau}{\int \phi^{\ast} \phi \:\mathrm{d}\tau} \geq E_1
        }
        \label{eq:8.9}
    \end{equation}
    其中$\phi$是满足问题定解条件的任意品优函数(不必归一化)。

    函数$\phi$称为\textbf{试探变分函数}(trial variational function),(\ref{eq:8.1})中的积分[或 (\ref{eq:8.9}) 中的积分之比]称为\textbf{变分积分}(variational integral)。为了获取基态能量$E_1$的良好近似值,我们会尝试许多试探变分函数,并寻找能给出最小变分积分值的函数。由 (\ref{eq:8.1}) 可知,变分积分的值越小,我们对 $E_1$ 的近似就越好。推翻量子力学的一种方法是找到一个试探变分函数,使已知 $E_1$ 的某个系统的变分积分小于 $E_1$。

    令$\psi_1$为系统基态能量的真实波函数:
    \begin{equation}
        \hat{H}\psi_1 = E_1 \psi_1
        \label{eq:8.10}
    \end{equation}
    如果我们碰巧幸运地发现了一个等于 $\psi_1$ 的变分函数,那么在 (\ref{eq:8.1}) 中使用 (\ref{eq:8.10}) ,我们可以看到变分积分将等于 $E_1$。因此,基态波函数给出了变分积分的最小值。因此,我们预期变分积分值越小,试探变分函数就越接近真实的基态波函数。然而,事实证明,变分积分接近 $E_1$ 的速度要比试探变分函数接近 $\psi_1$ 的速度快得多,而且有可能用一个相当差的 $\phi$ 得到一个相当好的近似值 $E_1$。

    在实践中,人们通常会在试探函数 $\phi$ 中引入几个参数,然后对这些参数进行变分,从而使变分积分最小化。能否成功使用变分法,取决于能否对试探函数做出正确的选择。

    让我们来看一些变分法的例子。虽然这种方法的真正用途是针对我们不知道确切解的问题,但我们还是要考虑那些完全可解的问题,以便判断我们结果的准确性。

    \begin{examplebox}
        \textbf{例题:}为长度为 $l$ 的一维势箱中的粒子设计一个试探变分函数。
        \\

        波函数在势箱外为零,定解条件要求在$x=0$和$x=l$处$\psi=0$。变分函数$\phi$必须满足这些在边界点为零的条件。如公式 (\ref{eq:4.57}) 后所述,基态 $\psi$ 在边界点内部没有节点,因此希望 $\phi$ 也没有内部节点。具有这些性质的一个简单函数为抛物线函数
        \begin{equation}
            \phi = x\left(l-x\right), \quad 0 \leq x \leq l
            \label{eq:8.11}
        \end{equation}
        以及在势箱外,有$\phi=0$。由于我们尚未归一化$\phi$,我们使用(\ref{eq:8.9})来计算。势箱内部的哈密顿算符为$-\left(\hbar^2/2m\right)\mathrm{d}^2/\mathrm{d}x^2$。对于 (\ref{eq:8.9}) 的分子和分母,我们可以得出
        \begin{equation}
            \int \phi^{\ast} \hat{H}\phi \:\mathrm{d}\tau = -\frac{\hbar^2}{2m} \int_{0}^{l} \left(lx-x^2\right)\frac{\mathrm{d}^2}{\mathrm{d}x^2} \left(lx-x^2\right) \mathrm{d}x = \frac{\hbar^2}{m} \int_0^l \left(lx-x^2\right) \:\mathrm{d}x = \frac{\hbar^2 l^3}{6m}
            \label{eq:8.12}
        \end{equation}
        \begin{equation}
            \int \phi^{\ast} \phi \:\mathrm{d}\tau = \int_0^l x^2\left(l-x\right)^2 \:\mathrm{d}x = \frac{l^5}{30}
            \label{eq:8.13}
        \end{equation}
        将它们带入变分理论(\ref{eq:8.9}),我们得到
        \begin{equation*}
            E_1 \leq \frac{5h^2}{4\pi^2 ml^2} = 0.1266515\frac{\hbar^2}{ml^2}
        \end{equation*}
        根据(\ref{eq:2.20 energy of one-dimensional box}),$E_1$的准确值为$h^2/8ml^2=0.125h^2/8ml^2$,能量的误差约为$1.3\%$。

        由于$\int \left|\phi\right|^2 \: \mathrm{d}\tau = l^5/30$,将(\ref{eq:8.11})归一化为$\left(30/l^5\right)^{1/2} x\left(l-x\right)$。图\ref{fig:7.3}显示,该函数与真正的箱中粒子基态波函数非常相似。
        \\

        \textbf{练习:}一个单粒子一维系统的势能函数满足
        \begin{equation*}
            V = \begin{cases}
                V_0, & 0 \leq x \leq l \\
                \infty, & elsewhere
            \end{cases}
        \end{equation*}
        (其中$V_0$是常数)。

        (a)使用试探函数
        \begin{equation*}
            \phi = \begin{cases}
                \sin\left(\pi x/l\right), & 0 \leq x \leq l \\
                0, & elsewhere
            \end{cases}
        \end{equation*}
        来估算该系统的基态能量;

        (b)解释为什么(a)的结果是该系统的准确基态能量。\textit{提示:见第\ref{chap:4}章的问题之一。}(答案:(a)$V_0 + h^2/8ml^2$。)
    \end{examplebox}

    上一个例题的试探函数中没有参数。下一个例题有。

    \begin{examplebox}
        \textbf{例题:}对于一维谐振子,设计一个带有参数的变分函数,并找出参数的最佳值。估计基态能量。
        \\

        变分函数$\phi$必须平方可积,即当$x \to \pm \infty$时,$\phi \to 0$。函数$\mathrm{e}^{-x}$在$+ \infty$时满足条件,但在$-\infty$时变为无穷大。函数$\mathrm{e}^{-x^2}$在$\pm \infty$时都满足条件。然而,从量纲上看,这种做法并不令人满意,因为$\mathrm{e}$的指数必须是无量纲的。这可以从Taylor级数中看出$\mathrm{e}^z = 1 + z + z^2/2! + z^3/3! + \ldots$[式(\ref{eq:4.44})]。由于该级数中的所有项都必须有相同的量纲,那么$z$的量纲就必须为1,也就是说$\mathrm{e}^z$中的$z$必须是无量纲的。因此,我们将$\mathrm{e}^{-x^2}$替换为$\mathrm{e}^{-cx^2}$,其中$c$的量纲为长度$^{-2}$。我们将$c$作为变分参数。真实的基态$\psi$没有节点。此外,由于$V = \frac{1}{2}kx^2$是偶函数,由于奇函数在原点有节点,那么基态的$\psi$一定具有确定的宇称,也一定是偶函数。试探函数$\mathrm{e}^{-cx^2}$具有我们需要的性质:无节点,且为偶函数。

        对$\hat{H}$使用(\ref{eq:4.30}),再使用附录中的积分,有(问题8.3)
        \begin{equation*}
            \begin{aligned}
                \int \phi^{\ast} \hat{H}\phi \:\mathrm{d}\tau &= -\frac{\hbar^2}{2m} \int_{-\infty}^{\infty} \mathrm{e}^{-2cx^2} \frac{\mathrm{d}^2\mathrm{e}^{-cx^2}}{\mathrm{d}x^2} \:\mathrm{d}x + 2\pi^2\nu^2m\int_{-\infty}^{\infty} x^2 \mathrm{e}^{-2cx^2} \:\mathrm{d}x \\
                &= \frac{\hbar^2}{m} \left(\frac{\pi c}{m}\right)^{1/2} + \nu^2m\left(\frac{\pi^5}{8}\right)^{1/2} c^{-3/2}
            \end{aligned}
        \end{equation*}
        \begin{equation*}
            \int \phi^{\ast} \phi \:\mathrm{d}\tau = \int_{-\infty}^{\infty} \mathrm{e}^{-2cx^2} \:\mathrm{d}x = 2\int_{0}^{\infty} \mathrm{e}^{-2cx^2} \:\mathrm{d}x = \left(\frac{\pi}{2c}\right)^{1/2}
        \end{equation*}
        试探积分$W$为
        \begin{equation}
            W \equiv \frac{\int \phi^{\ast} \hat{H}\phi \:\mathrm{d}\tau}{\int \phi^{\ast} \phi \:\mathrm{d}\tau} = \frac{\hbar^2c}{2m} + \frac{\pi^2\nu^2m}{2c}
            \label{eq:8.14}
        \end{equation}
        现在,我们通过改变$c$来最小化$W$(\ref{eq:8.14})。$W$最小的必要条件为
        \begin{equation*}
            \frac{\mathrm{d} W}{\mathrm{d} c} = 0 = \frac{\hbar^2}{2m} - \frac{\pi^2\nu^2m}{2c^2}
        \end{equation*}
        \begin{equation}
            c = \pm \pi \nu m/\hbar
            \label{eq:8.15}
        \end{equation}
        其中,负根$c = -\pi\nu m/\hbar$应舍去,因其会导致$\phi = \mathrm{e}^{-cx^2}$不满足平方可积的要求。将$c = \pi\nu m/\hbar$代入(\ref{eq:8.14}),我们得到$W = \frac{1}{2}h\nu$。这是一维谐振子基态能量的准确值。当$c = \pi\nu m/\hbar$时,变分函数$\phi$除了未归一化以外,与谐振子基态波函数(\ref{eq:4.53})和(\ref{eq:4.31})完全一致。

        对于归一化的谐振子变分函数$\phi = \left(2c/\pi\right)^{1/4} \mathrm{e}^{-cx^2}$,若$c$的值很大,会导致$\phi$在$x=0$附近从其最大值快速下降。那么概率密度就只在$x=0$附近较大。势函数$V = \frac{1}{2}kx^2$在$x=0$附近也很小,所以较大的$c$值意味着较小的$\left\langle V \right\rangle = \left\langle \phi \middle| V \middle| \phi \right\rangle$。[注意:$\left\langle V \right\rangle$等于(\ref{eq:8.14})右侧的第二项。]然而,由于较大的$c$值使得$\phi$从其最大值快速减小,那么在$x=0$附近$\left|\mathrm{d}\phi/\mathrm{d}x\right|$的值变得很大,根据问题7.7b,大$\left|\mathrm{d}\phi/\mathrm{d}x\right|$意味着大$\left\langle T \right\rangle$[等于(\ref{eq:8.14})右侧的第一项。]$c$的最佳值可以使总和$\left\langle T \right\rangle + \left\langle V \right\rangle = W$最小化。在原子和分子中,真正的波函数是通过将电子限制在低$V$区域(靠近原子核)来最小化$\left\langle V \right\rangle$的趋势与通过允许电子概率密度在大区域内扩散来最小化$\left\langle T \right\rangle$的趋势之间的折衷。
        \\

        \textbf{练习:}考虑一个单粒子一维系统,其势函数满足
        \begin{equation*}
            V = \begin{cases}
                V = 0, & -\frac{1}{2}l \leq x \leq \frac{1}{2}l \\
                V = b\hbar^2/ml^2, & elsewhere
            \end{cases}
        \end{equation*}
        其中$b$是正常数(图 \ref{fig:2.5} 中的 $V_0 = b\hbar^2/ml^2$ 和原点发生了偏移)。

        (a)对于变分函数
        \begin{equation*}
            \phi = \begin{cases}
                \left(x-c\right)\left(x+c\right) = x^2 - c^2, & -c \leq x \leq c \\
                0, & elsewhere
            \end{cases}
        \end{equation*}
        其中变分参数$c > \frac{1}{2}l$。我们发现变分积分$W$由下式给出:
        \begin{equation*}
            W = \frac{\hbar^2}{ml^2}\left[\frac{5l^2}{4c^2} + b\left(1-\frac{15l}{16c} + \frac{5l^3}{32c^3} - \frac{3l^5}{256c^5}\right)\right]
        \end{equation*}
        在同一幅图上绘制$\phi$和$V$的图象。求使 $W$ 最小的 $c$ 值所满足的方程。

        (b)当$V_0 = 20\hbar^2/ml^2$时,求出$c$的最佳值和$W$的值,将其与基态能量真实值$2.814\hbar^2/ml^2$进行比较(问题4.31c)。(\textit{提示:}可以使用Excel表格求解器或可编程计算器来计算$c/l$的值。)

        (\textit{答案:}(a)$48t^4-24t^2-128t^3/b+3=0$,其中$t \equiv c/l$;(b)$c = 0.6715l$,$W = 3.454\hbar^2/ml^2$。)
    \end{examplebox}

\section{变分法的推广}
\label{sec:8.2 Extension of the Variational Method}
    上一节介绍的变分法只提供了基态能量和波函数的信息。现在我们讨论将变分法推广到激发态。(另请参见第 \ref{sec:8.5 Linear Variational Functions} 节)。

    考虑一下我们可以如何推广变分法,来估算第一激发态的能量。根据能量从低到高:
    \begin{equation*}
        E_1 \leq E_2 \leq E_3 \leq \ldots
    \end{equation*}
    我们将系统的状态标号为$1,2,3,\ldots$。我们已经证明:对于一个已归一化的变分函数$\phi$,有[式(\ref{eq:8.4})和(\ref{eq:8.6})]
    \begin{equation*}
        \int \phi^{\ast} \hat{H}\phi \:\mathrm{d}\tau = \sum_{k=1}^{\infty} |a_k|^2 E_k \quad \text{和} \quad \int \phi^{\ast} \phi \:\mathrm{d}\tau = \sum_{k=1}^{\infty} |a_k|^2 = 1
    \end{equation*}
    其中的$a_k$是展开式$\phi = \sum_{k} a_k \psi_k$中的系数[式(\ref{eq:8.2})]。我们有$a_k = \left\langle \psi_k \middle| \phi \right\rangle$[式(\ref{eq:7.40})]。让我们只讨论与真实基态波函数$\psi$正交的归一化函数$\phi$。那么$a_1 = \left\langle \psi_1 \middle| \phi \right\rangle = 0$,因此
    \begin{equation}
        \int \phi^{\ast} \hat{H}\phi \:\mathrm{d}\tau = \sum_{k=2}^{\infty} |a_k|^2 E_k \quad \text{和} \quad \int \phi^{\ast} \phi \:\mathrm{d}\tau = \sum_{k=2}^{\infty} |a_k|^2 = 1
        \label{eq:8.16}
    \end{equation}
    对于$k \geq 2$,我们有$E_k \geq E_2$和$\left|a_k\right|^2E_k \geq \left|a_k\right|^2E_2$。因此,
    \begin{equation}
        \sum_{k=2}^{\infty} \left|a_k\right|^2 E_k \geq \sum_{k=2}^{\infty} \left|a_k\right|^2 E_2 = E_2 \sum_{k=2}^{\infty} \left|a_k\right|^2 = E_2
        \label{eq:8.17}
    \end{equation}
    将(\ref{eq:8.16})和(\ref{eq:8.17})相结合,我们得到了所需的结果:
    \begin{equation}
        \int \phi^{\ast} \hat{H}\phi \:\mathrm{d}\tau \geq E_2, \quad \text{若} \int \psi^{\ast} \phi \:\mathrm{d}\tau = 0 \quad \text{且} \int \phi^{\ast} \phi \:\mathrm{d}\tau = 1
        \label{eq:8.18}
    \end{equation}
    通过不等式 (\ref{eq:8.18}) 我们可以得到第一激发态的能量 $E_2$ 的上限。然而,限制条件$\left\langle \psi_1 \middle| \phi \right\rangle = 0$使得这种方法在应用上会很麻烦。

    对于特定的系统,尽管我们不知道真实基态波函数,但确定$\left\langle \psi_1 \middle| \phi \right\rangle = 0$是可能的。$V$ 是 $x$ 的偶函数的一维问题就是一个例子。在这种情况下,基态波函数总是偶函数,而第一激发态波函数是奇函数。(所有波函数必须具有确定的宇称。基态波函数是无节点的,由于奇函数在原点消失,所以基态波函数必须是偶函数。第一个激发态波函数有一个节点,一定是奇函数。)因此,对于奇函数,$\left\langle \psi_1 \middle| \phi \right\rangle = 0$一定成立,偶函数$\psi_1$乘以一个奇函数$\phi$得到一个奇被积函数,从$-\infty$到$+\infty$的积分为零。

    另一个例子是在中心场中运动的粒子(第 \ref{sec:6.1 The One-Particle Central Force Problem} 节)。势能的形式可能使我们无法求解本征函数中的径向因子 $R(r)$。然而,其中的角度因子 $\psi$ 是球谐函数[式 (\ref{eq:6.16}) ],而不同 $l$ 值的球谐函数是正交的。因此,我们可以利用试探函数中的因子$Y_l^m$,求出任意给定角动量$l$的最低态能量上限。这一结果取决于将 (\ref{eq:8.18}) 推广到更高的激发态:
    \begin{equation}
        \frac{\int \phi^{\ast} \hat{H}\phi \:\mathrm{d}\tau}{\int \phi^{\ast} \phi \:\mathrm{d}\tau} \geq E_{k+1}, \quad \text{若} \int \psi_1^{\ast} \phi \: \mathrm{d}\tau = \int \psi_2^{\ast} \phi \:\mathrm{d}\tau = \cdots = \int \psi_k^{\ast} \phi \:\mathrm{d}\tau = 0
        \label{eq:8.19}
    \end{equation}

\section{行列式}
\label{sec:8.3 Determinants}
    第 \ref{sec:8.5 Linear Variational Functions} 节讨论了一种变分函数,它产生了一个涉及行列式的方程。因此,我们现在讨论行列式。
     
    \textbf{行列式}(Determinant)是一个包含 $n^2$ 个量(称为\textbf{元素}(elements))的正方形数组;行列式的值是根据其元素计算出来的,计算方法稍后给出。$n$称为行列式的\textbf{阶}(order)。使用$a_{ij}$来代表一个典型元素,我们有$n$阶行列式
    \begin{equation}
            \mathrm{det}\left(a_{ij}\right) = \begin{vmatrix}
                a_{11} & a_{12} & a_{13} & \cdots & a_{1n} \\
                a_{21} & a_{22} & a_{23} & \cdots & a_{2n} \\
                \vdots & \vdots & \vdots & \ddots & \vdots \\
                a_{n1} & a_{n2} & a_{n3} & \cdots & a_{nn}
            \end{vmatrix}
            \label{eq:8.20}
    \end{equation}
    (\ref{eq:8.20})中的竖线与绝对值无关。在考虑$n$阶行列式值的定义方法之前,我们先考虑一阶、二阶和三阶行列式。

    一阶行列式有一个元素,其值就是该元素本身。因此,
    \begin{equation}
        \left| a_{11} \right| = a_{11}
        \label{eq:8.21}
    \end{equation}
    其中的竖线表示行列式而不是绝对值。

    二阶行列式有四个元素,其值定义为
    \begin{equation}
        \boxed{
            \begin{vmatrix}
                a_{11} & a_{12} \\
                a_{21} & a_{22}
            \end{vmatrix} = a_{11}a_{22} - a_{12}a_{21}
        }
        \label{eq:8.22}
    \end{equation}
    三阶行列式的值定义为
    \begin{equation}
            \boxed{
            \begin{vmatrix}
                a_{11} & a_{12} & a_{13} \\
                a_{21} & a_{22} & a_{23} \\
                a_{31} & a_{32} & a_{33}
            \end{vmatrix} = a_{11}\begin{vmatrix}
                a_{22} & a_{23} \\
                a_{32} & a_{33}
            \end{vmatrix} - a_{12}\begin{vmatrix}
                a_{21} & a_{23} \\
                a_{31} & a_{33}
            \end{vmatrix} + a_{13}\begin{vmatrix}
                a_{21} & a_{22} \\
                a_{31} & a_{32}
            \end{vmatrix}
        }
        \label{eq:8.23}
    \end{equation}
    \begin{equation}
        \begin{aligned}
            = & a_{11}a_{22}a_{33} - a_{11}a_{23}a_{32} - a_{12}a_{21}a_{33} + a_{12}a_{23}a_{31} \\ 
            & + a_{13}a_{21}a_{32} - a_{13}a_{22}a_{31}
        \end{aligned}
        \label{eq:8.24}
    \end{equation}
    三阶行列式的计算方法是:用正负号交替写下第一行的元素,然后将每个元素乘以某个二阶行列式;在三阶行列式中划掉某个元素出现的行和列,就能找到与该元素相乘的二阶行列式。将$n$阶行列式的第$i$行和第$j$列删去后得到的$\left(n-1\right)$阶行列式称为元素$a_{ij}$的\textbf{余子式}(minor)。我们再定义$a_{ij}$的\textbf{代数余子式}(cofactor)为$a_{ij}$的余子式乘以$(-1)^{i+j}$。因此,(\ref{eq:8.23}) 说明三阶行列式的计算方法是:将顶行的每个元素乘以它的代数余子式,然后将三个乘积相加。[注意:式(\ref{eq:8.22})通过代数余子式符合该方法。因为$a_{11}$的代数余子式为$a_{22}$,$a_{12}$的代数余子式为$-a_{21}$。]一个数值的例子如下
    \begin{equation*}
        \begin{aligned}
            \begin{vmatrix}
                5 & 10 & 2 \\
                0.1 & 3 & 1 \\
                0 & 4 & 4
            \end{vmatrix} &= 5\begin{vmatrix}
                3 & 1 \\
                4 & 4
            \end{vmatrix} - 10\begin{vmatrix}
                0.1 & 1 \\
                0 & 4
            \end{vmatrix} + 2\begin{vmatrix}
                0.1 & 3 \\
                0 & 4
            \end{vmatrix} \\
            &= 5\left(8\right) - 10\left(0.4\right) + 2\left(0.4\right) = 36.8
            \end{aligned}
    \end{equation*}

    我们将$a_{ij}$的余子式记为$M_{ij}$,代数余子式记为$C_{ij}$,则
    \begin{equation}
        c_{ij} = \left(-1\right)^{i+j}M_{ij}
        \label{eq:8.25}
    \end{equation}
    则三阶行列式的展开式(\ref{eq:8.23})可以写为
    \begin{equation}
        \mathrm{det}\left(a_{ij}\right) = \begin{vmatrix}
            a_{11} & a_{12} & a_{13} \\
            a_{21} & a_{22} & a_{23} \\
            a_{31} & a_{32} & a_{33}
        \end{vmatrix} = a_{11}C_{11} + a_{12}C_{12} + a_{13}C_{13}
        \label{eq:8.26}
    \end{equation}
    三阶行列式可以用任意一行的元素和相应的代数余子式展开。例如,利用第二行展开三阶行列式,我们可以得出
    \begin{equation}
        \mathrm{det}\left(a_{ij}\right) = a_{21}C_{21} + a_{22}C_{22} + a_{23}C_{23}
        \label{eq:8.27}
    \end{equation}
    \begin{equation}
        \mathrm{det}\left(a_{ij}\right) = -a_{21}\begin{vmatrix}
            a_{11} & a_{13} \\
            a_{31} & a_{33}
        \end{vmatrix} + a_{22}\begin{vmatrix}
            a_{11} & a_{13} \\
            a_{31} & a_{33}
        \end{vmatrix} - a_{23}\begin{vmatrix}
            a_{11} & a_{12} \\
            a_{31} & a_{32}
        \end{vmatrix}
        \label{eq:8.28}
    \end{equation}
    二阶行列式的展开式表明 (\ref{eq:8.28}) 等于 (\ref{eq:8.24})。我们也可以使用任意一列的元素和相应的代数余子式来展开行列式,这一点很容易得到验证。因此,对于三阶行列式,我们可以写成
    \begin{equation*}
        \mathrm{det}\left(a_{ij}\right) = a_{k1}C_{k1} + a_{k2}C_{k2} + a_{k3}C_{k3} = \sum_{l=1}^{3} a_{kl}C_{kl}, \quad k = 1,2,3
    \end{equation*}
    \begin{equation*}
        \mathrm{det}\left(a_{ij}\right) = a_{1k}C_{1k} + a_{2k}C_{2k} + a_{3k}C_{3k} = \sum_{l=1}^{3} a_{lk}C_{lk}, \quad k = 1,2,3
    \end{equation*}
    第一个展开式按某一行展开,第二个则按某一列展开。

    我们通过类似的行(或列)展开来定义高阶行列式。对于 $n$ 阶行列式,
    \begin{equation}
        \mathrm{det}\left(a_{ij}\right) = \sum_{l=1}^{n} a_{kl}C_{kl} = \sum_{l=1}^{n} a_{lk}C_{lk}, \quad k = 1,2,\ldots,n
        \label{eq:8.29}
    \end{equation}

    一些有关行列式的性质如下所述(证明见\textit{Sokolnikoff and Redheffer}, pp. 702–707):
        \begin{enumerate}
            \item 如果行列式某行(或列)的每个元素都为零,则行列式的值为零。
            \item 交换行列式的两行(或两列)会使行列式的值改变符号。
            \item 如果行列式的两行(或两列)相同,则该行列式的值为零。
            \item 如果将行列式的一行(或一列)乘以一个常数 $k$,则该行列式的值也乘以 $k$。
            \item 如在一行的每个元素上加上另一行相应元素的相同常数倍数,行列式的值不变。这个定理也适用于将一列的倍数加到另一列。
            \item 互换所有相应的行和列,行列式的值保持不变。(这种互换意味着第一列变为第一行,第二列变为第二行,等等)
        \end{enumerate}
    
    \begin{examplebox}
        \textbf{例题:}使用性质5计算
        \begin{equation}
            B = \begin{vmatrix}
                1 & 2 & 3 & 4 \\
                4 & 1 & 2 & 3 \\
                3 & 4 & 1 & 2 \\
                2 & 3 & 4 & 1
            \end{vmatrix}
            \label{eq:8.30}
        \end{equation}
        \\

        将第一行的元素乘以$-2$并加到第四行,将其变为$2+\left(-2\right)1=0$,$3+\left(-2\right)2= -1$,$4+\left(-2\right)3 = -2$,$1+\left(-2\right)4 = -7$。同理,将第一行乘以$-3$并加到第三行,乘以$-4$并加到第二行。这样,(\ref{eq:8.30})变为
        \begin{equation}
            B = \begin{vmatrix}
                1 & 2 & 3 & 4 \\
                0 & -7 & -10 & -13 \\
                0 & -2 & -8 & -10 \\
                0 & -1 & -2 & -7
            \end{vmatrix} = 1\begin{vmatrix}
                -7 & -10 & -13 \\
                -2 & -8 & -10 \\
                -1 & -2 & -7
            \end{vmatrix}
            \label{eq:8.31}
        \end{equation}
        其中,我们将$B$按第一列展开。第二行减去第三行的两倍,第一行再减去第三行的七倍,得出
        \begin{equation}
            B = \begin{vmatrix}
                0 & 4 & 36 \\
                0 & -4 & 4 \\
                -1 & -2 & -7
            \end{vmatrix} = \left(-1\right)\begin{vmatrix}
                4 & 36 \\
                -4 & 4
            \end{vmatrix} = -\left(16 + 144\right) = -160
            \label{eq:8.32}
        \end{equation}
    \end{examplebox}

    行列式中从左上方到右下方的对角线就是\textbf{主对角线}(principal diagonal)。\textbf{对角行列式}(diagonal determinant)是指除主对角线上的元素外,其他元素均为零的行列式。对于一个对角行列式,
    \begin{equation}
        \begin{aligned}
            \begin{vmatrix}
                a_{11} & 0 & 0 & \cdots & 0 \\
                0 & a_{22} & 0 & \cdots & 0 \\
                0 & 0 & a_{33} & \cdots & 0 \\
                \vdots & \vdots & \vdots & \ddots & \vdots \\
                0 & 0 & 0 & \cdots & a_{nn}
            \end{vmatrix} &= a_{11}\begin{vmatrix}
                a_{22} & 0 & \cdots & 0 \\
                0 & a_{33} & \cdots & 0 \\
                \vdots & \vdots & \ddots & \vdots \\
                0 & 0 & \cdots & a_{nn}
            \end{vmatrix} = a_{11}a_{22}\begin{vmatrix}
                a_{33} & 0 & \cdots & 0 \\
                0 & a_{44} & \cdots & 0 \\
                \vdots & \vdots & \ddots & \vdots \\
                0 & 0 & \cdots & a_{nn}
            \end{vmatrix} \\
            &= \cdots = a_{11}a_{22}a_{33}\cdots a_{nn}
        \end{aligned}
    \label{eq:8.33}
    \end{equation}
    \textit{对角行列式的值等于其对角元素的乘积。}

    一个行列式的唯一非零元素出现在以主对角线为中心的方形块中,这个行列式就是\textbf{块对角形式}(block diagonal form)。 如果我们把每个正方形块看作行列式,那么块对角行列式就等于块的乘积。例如
    \begin{equation}
        \begin{vmatrix}
            a & b & 0 & 0 & 0 & 0 \\
            c & d & 0 & 0 & 0 & 0 \\
            0 & 0 & e & 0 & 0 & 0 \\
            0 & 0 & 0 & f & g & h \\
            0 & 0 & 0 & i & j & k \\
            0 & 0 & 0 & l & m & n
        \end{vmatrix} = \begin{vmatrix}
            a & b \\
            c & d
        \end{vmatrix} \left(e\right)\begin{vmatrix}
            f & g & h \\
            i & j & k \\
            l & m & n
        \end{vmatrix}
        \label{eq:8.34}
    \end{equation}
    虚线(图中未画出)勾勒出区块的轮廓。用顶行元素展开左边的行列式,再用它们的顶行展开后面的几个行列式,就可以很容易地证明方程 (\ref{eq:8.34}) (问题 8.21)。

\section{联立线性方程组}
\label{sec:8.4 Simultaneous Linear Equations}
    要处理下一节所要讨论的变分函数,我们需要知道联立线性方程组。

    考虑以下关于$n$个未知数的$n$个线性方程系统:
    \begin{equation}
        \begin{aligned}
            a_{11}x_1 + a_{12}x_2 + \cdots + a_{1n}x_n &= b_1 \\
            a_{21}x_1 + a_{22}x_2 + \cdots + a_{2n}x_n &= b_2 \\
            &\vdots \\
            a_{n1}x_1 + a_{n2}x_2 + \cdots + a_{nn}x_n &= b_n
        \end{aligned}
        \label{eq:8.35}
    \end{equation}
    其中的$a$和$b$是已知常数,$x_1,x_2,\ldots,x_n$是未知数。如果至少有一个$b$值为非零,我们得到了一个\textbf{非齐次}(inhomogeneous)方程组。这样的系统可以用Cramer法则(Cramer's rule)求解。(Cramer法则的证明见\textit{Sokolnikoff and Redheffer}, pp. 708。)令$\mathrm{det}\left(a_{ij}\right)$表示(\ref{eq:8.35})中未知数的系数行列式,由Cramer法则,$x_k \: \left(k = 1,2,\ldots,n\right)$的值由下式给出:
    \begin{equation}
        x_k = \frac{
            \begin{vmatrix}
                a_{11} & a_{12} & \cdots & a_{1,k-1} & b_1 & a_{1,k+1} & \cdots & a_{1n} \\
                a_{21} & a_{22} & \cdots & a_{2,k-1} & b_2 & a_{2,k+1} & \cdots & a_{2n} \\
                \vdots & \vdots & \ddots & \vdots & \vdots & \vdots & \ddots & \vdots \\
                a_{n1} & a_{n2} & \cdots & a_{n,k-1} & b_n & a_{n,k+1} & \cdots & a_{nn}
            \end{vmatrix}
        }{\mathrm{det}\left(a_{ij}\right)}, \quad k = 1,2,\ldots,n
        \label{eq:8.36}
    \end{equation}
    其中的$\mathrm{det}\left(a_{ij}\right)$由(\ref{eq:8.20})给出,分子中的行列式通过将$\mathrm{det}\left(a_{ij}\right)$的第$k$列替换为$b_1,b_2,\ldots,b_n$来获得。虽然Cramer法则具有理论意义,但不应在数值计算中使用,因为连续消除未知数的效率更高。

    一种广泛使用的连续消元法称为\textbf{高斯消元法}(Gaussian elimination),其过程如下:将第一个方程的两边同除以$x_1$的系数$a_{11}$,使得$x_1$的系数变为1。然后用第二个方程减去第一个方程的$a_{21}$倍,用第三个方程减去第一个方程的$a_{31}$倍,以此类推,直到用第$n$个方程减去第一个方程的$a_{n1}$倍。这就在所有方程中消去了$x_1$(除了第一个方程)。接下来,将第二个方程的两边同除以$x_2$的系数$a_{22}$,使得$x_2$的系数变为1。然后用第三,第四$\cdots$,第$n$个方程减去第二个方程的适当倍数,这就消去了$x_2$(除了第一和第二个方程)。重复这个过程,直到第$n$个方程只包含$x_n$,第$n-1$个方程只包含$x_{n-1}$和$x_n$,等等。通过第$n$个方程解得$x_n$,然后将其代入第$n-1$个方程中解得$x_{n-1}$,将$x_{n-1}$和$x_n$代入第$n-2$个方程中解得$x_{n-2}$,以此类推,直到解出所有未知数。如果在任何阶段,我们想要除以的系数恰好为零,那么零系数的等式就会与后来在所需位置上具有非零系数的等式交换。(高斯消元法也是一种有效的行列式求值方法,见问题 8.25。)

    一种相关的方法是\textbf{高斯-若尔当消元法}(Gauss-Jordan elimination),其过程与高斯消元法类似,但与高斯消元法将$x_2$从方程$3,4,\ldots,n$中消去不同,通过减去第二个方程的适当倍数,我们将$x_2$从方程$1,3,4,\ldots,n$中消去;与将$x_3$从方程$4,5,\ldots,n$中消去不同,我们将$x_3$从方程$1,2,4,5,\ldots,n$中消去,以此类推。最后,方程1只包含$x_1$,方程2只包含$x_2$,方程$n$只包含$x_n$。高斯-若尔当消元法比高斯消元法需要更多的计算量。

    若(\ref{eq:8.35})中所有的$b$都为零,我们得到了一个\textbf{齐次线性方程组}(homogeneous linear equations):
    \begin{equation}
        \begin{aligned}
            a_{11}x_1 + a_{12}x_2 + \cdots + a_{1n}x_n &= 0 \\
            a_{21}x_1 + a_{22}x_2 + \cdots + a_{2n}x_n &= 0 \\
            &\vdots \\
            a_{n1}x_1 + a_{n2}x_2 + \cdots + a_{nn}x_n &= 0
        \end{aligned}
        \label{eq:8.37}
    \end{equation}
    (\ref{eq:8.37})的一组显而易见的解为$x_1=x_2 = \ldots = x_n=0$,称为\textbf{平凡解}(trivial solution)。若(\ref{eq:8.37})的系数行列式$\mathrm{det}\left(a_{ij}\right)$不为零,我们可以使用Cramer法则(\ref{eq:8.36})来求解未知数,并发现$x_k=0, \: k=1,2,\ldots,n$,因为 (\ref{eq:8.36}) 分子中的行列式有一列所有元素都为零。因此,当$\mathrm{det}\left(a_{ij}\right) \neq 0$时,唯一的解就是平凡解,这没有任何意义。要使关于 $n$ 个未知数的,由$n$个齐次方程组成的线性方程组有一组非平凡解,系数的行列式必须为零。而且,这个条件足以确保非平凡解的存在。因此,我们得到了极为重要的定理:
    \begin{center}
        \parbox{0.8\textwidth}{
            \textit{当且仅当系数行列式为零时,关于 $n$ 个未知数的,由$n$个齐次方程组成的线性方程组有一组非平凡解。}
        }
    \end{center}

    假设$\mathrm{det}\left(a_{ij}\right) = 0$,则(\ref{eq:8.37})有一组非平凡解。我们如何找到它呢?由于$\mathrm{det}\left(a_{ij}\right) = 0$,Cramer法则(\ref{eq:8.36})给出$x_k = 0/0, \: k = 1,2,\ldots,n$,这没有任何意义。因此,Cramer法则并不能立即起到帮助作用。注意到,若$x_1=d_1,x_2=d_2,\ldots,x_n=d_n$是(\ref{eq:8.37})的一组解,那么$x_1=cd_1,x_2=cd_2,\ldots,x_n=cd_n$也是(\ref{eq:8.37})的一组解,其中$c$是任意常数。这一点很容易看出,因为
    \begin{equation*}
        a_{11}cd_1 + a_{12}cd_2 + \cdots + a_{1n}cd_n = c\left(a_{11}d_1 + a_{12}d_2 + \cdots + a_{1n}d_n\right) = c\cdot 0 = 0
    \end{equation*}
    以此类推。因此,线性齐次方程组的解将包含一个任意常数,我们无法为每一个未知数找到一个唯一的值。为了求解(\ref{eq:8.37}),我们给任意一个未知数赋以任意的值,如$x_n=c$,其中$c$是任意常数。确定了$x_n$的值之后,我们将(\ref{eq:8.37})中每个方程的最后一项移到右边,得到
    \begin{equation}
        \begin{aligned}
            a_{11}x_1 + a_{12}x_2 + \cdots + a_{1,n-1}x_{n-1} &= -a_{1n}c \\
            a_{21}x_1 + a_{22}x_2 + \cdots + a_{2,n-1}x_{n-1} &= -a_{2n}c \\
            &\vdots \\
            a_{n-1,1}x_1 + a_{n-1,2}x_2 + \cdots + a_{n-1,n-1}x_{n-1} &= -a_{n-1,n}c \\
            a_{n1}x_1 + a_{n2}x_2 + \cdots + a_{n,n-1}x_{n-1} &= -a_{nn}c
        \end{aligned}
        \label{eq:8.38}
    \end{equation}
    我们现在有$n-1$个未知数和$n$个方程,这比我们需要的多了一个。因此,我们删去(\ref{eq:8.38})中的任意一个方程,以最后一个为例。这就得到了关于$n-1$个未知数的,由$n-1$个\textit{非}齐次方程组成的线性方程组。然后就可以用Cramer法则(\ref{eq:8.36})来求解未知数$x_1,x_2,\ldots,x_{n-1}$。因为(\ref{eq:8.38})中,右侧的常数都包含因子$c$,所以根据\ref{sec:8.3 Determinants}节的定理4,$x_1,x_2,\ldots,x_{n-1}$的值也包含因子$c$。因此,(\ref{eq:8.37})的解可以写成
    \begin{equation}
        x_1 = ce_1, \quad x_2 = ce_2, \quad \ldots, \quad x_{n-1} = ce_{n-1}, \quad x_n = c
        \label{eq:8.39}
    \end{equation}
    其中$c$是任意常数,$e_1,e_2,\ldots,e_{n-1}$是数字。

    \begin{examplebox}
        \textbf{例题:}求解:
        \begin{equation*}
            \begin{aligned}
                3x_1 + 4x_2 + x_3 &= 0 \\
                x_1 + 3x_2 - 2x_3 &= 0 \\
                x_1 - 2x_2 + 5x_3 &= 0
            \end{aligned}
        \end{equation*}

        这是线性齐次方程组,我们先计算系数行列式。得出(见练习)
        \begin{equation*}
            \begin{vmatrix}
                3 & 4 & 1 \\
                1 & 3 & -2 \\
                1 & -2 & 5
            \end{vmatrix} = 0
        \end{equation*}
        因此,存在一组非平凡解。我们给$x_3$赋值$c$,删去第三个方程,得到
        \begin{equation*}
            \begin{aligned}
                3x_1 + 4x_2 &= -c \\
                x_1 + 3x_2 &= 2c
            \end{aligned}
        \end{equation*}
        第一个方程减去第二个方程的三倍,得到$-5x_2 = -7c$,因此$x_2 = \frac{7}{5}c$。带入$x_1+3x_2 = 2c$,得到$x_1 +\frac{21}{5}c = 2c$,因此$x_1 = -\frac{11}{5}c$。因此,通解为$x_1 = -\frac{11}{5}c, \: x_2 = \frac{7}{5}c, \: x_3 = c$。对于那些对分式过敏的人(无意冒犯,译者注),我们定义一个新的任意常数$s \equiv \frac{1}{5}c$,则$x_1 = -11s, \: x_2 = 7s, \: x_3 = 5s$。
        \\

        \textbf{练习:}

        (a)证明本例中系数行列式的值为零;

        (b)证明:将第二个方程乘以某个常数并与第一个方程相加,就可以得到本例中的第三个方程。
    \end{examplebox}

    如果 $n-1$ 个未知数中的 $n-1$ 个非均质方程组的行列式 [(\ref{eq:8.38}),最后一个方程省略] 恰好为零,则上述步骤将失效。这样Cramer法则的分母中就有了一个零,也就失去了作用。为了解决这个问题,我们可以先将任意值赋值给另一个未知数,而不是 $x_n$。我们还可以尝试删去 (\ref{eq:8.38}) 的其他方程,而不是最后一个方程。我们要寻找的是一个不会消失的 $n-1$ 阶行列式,它由系统 (\ref{eq:8.37}) 的系数行列式通过删除一行和一列而形成。如果存在这样的行列式,那么按照给出的步骤,正确选择要舍弃的方程,正确选择要赋值的未知数,我们就可以求解方程组,并得到形式为 (\ref{eq:8.39}) 的解。如果不存在这样的行列式,我们就必须给两个未知数任意赋值,并尝试从这里开始求解。因此,(\ref{eq:8.37}) 的解可能包含两个(甚至更多)任意常数。

    求解线性齐次方程组的有效方法是对方程进行高斯-若尔当消元。如果只存在三元解,最终得到的方程组将是 $x_1=0,x_2=0,\ldots,x_n=0$。如果存在非三元解,则至少有一个方程会简化为 $0=0$ 的形式;如果得到 $m$ 个 $0=0$ 形式的方程,我们就为其中的 $m$ 个未知数指定任意常数,并用这 $m$ 个未知数来表示其余的未知数。

    \begin{examplebox}
        \textbf{例题:}使用高斯-若尔当消元法求解上例中的方程组。

        在使用高斯消元法或高斯-若尔当消元法求解包含$n$个齐次或非齐次方程的方程组时,通过略去变量$x_1,\ldots,x_n$,只写出包含第$n$行和第$\left(n+1\right)$列系数和常数(包含任意零系数)组成的阵列,我们可以省去不必要的过程;然后,我们对每一行的数字进行运算,就好像这一行就是它所代表的等式一样,从而产生下一个阵列。

        为了开始除法消元,我们交换第一个方程和第二个方程,从而以 $a_{11} = 1$ 开始。分离系数并进行高斯-若尔当消元,我们得到
        \begin{equation*}
            \begin{matrix}
                1 & 3 & -2 & 0 \\
                3 & 4 & 1 & 0 \\
                1 & -2 & 5 & 0
            \end{matrix} \quad \rightarrow \quad 
            \begin{matrix}
                1 & 3 & -2 & 0 \\
                0 & -5 & 7 & 0 \\
                0 & -5 & 7 & 0
            \end{matrix} \quad \rightarrow \quad 
            \begin{matrix}
                1 & 3 & -2 & 0 \\
                0 & -1 & -\frac{7}{5} & 0 \\
                0 & 0 & 0 & 0
            \end{matrix} \quad \rightarrow \quad
            \begin{matrix}
                1 & 0 & \frac{11}{5} & 0 \\
                0 & 1 & -\frac{7}{5} & 0 \\
                0 & 0 & 0 & 0
            \end{matrix}
        \end{equation*}
        第一个阵列是原方程组,但第一个和第二个方程互换。为了将$x_1$从第二和第三个方程中消去,我们将第二个方程乘以$-3$并加到第二个方程上,将第一个方程乘以$-1$并加到第三个方程上,这就得到了第二个阵列。对第二行除以5得到第三个阵列。为了将$x_2$从第一和第三个方程中消去,我们将第二个方程乘以$-3$并加到第一个方程上,将第二个方程乘以$-7/5$并加到第三个方程上,得到第四个阵列。由于第四个阵列中$x_3$的系数为零,我们不能使用第三个方程从第一和第二个方程中消去$x_3$(这原本是高斯-若尔当消元法的下一步)。因此,我们将$0=0$的第三个方程删去,设$x_3=k$,其中$k$是任意常数。最后一个阵列中的第一和第二个方程有$x_1 + \frac{11}{5}x_3 = 0$和$x_2 - \frac{7}{5}x_3 = 0$,或$x_1 = -\frac{11}{5}x_3$和$x_2 = \frac{7}{5}x_3$。因此,通解为$x_1 = -\frac{11}{5}k, \: x_2 = \frac{7}{5}k, \: x_3 = k$。
    \end{examplebox}

\section{线性变分函数}
\label{sec:8.5 Linear Variational Functions}

















\section{矩阵、本征值和本征向量}
\label{sec:8.6 Matrices, Eigenvalues, and Eigenvectors}

\section*{总结}

\section*{习题}