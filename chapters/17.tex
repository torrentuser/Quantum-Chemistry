% ===== CHAPTER 17 =====
\chapter{半经验和分子力学方法求解分子}
\label{chap:17}
\section{半经验分子轨道求解平面共轭分子}
\label{sec:17.1 Semiempirical MO Treatments of Planar Conjugated Molecules}

\section{Hückel分子轨道理论}
\label{sec:17.2 The Hückel MO Method}

\section{PPP方法}
\label{sec:17.3 The PPP Method}

\section{一般的半经验分子轨道和密度泛函方法}
\label{sec:17.4 General Semiempirical MO and DFT Methods}

\section{分子力学方法}
\label{sec:17.5 The Molecular-Mechanics Method}

\section{经验和半经验方法处理溶剂效应}
\label{sec:17.6 Empirical and Semiempirical Treatments of Treating Solvent Effects}

\section{化学反应}
\label{sec:17.7 Chemical Reactions}

\section{量子化学的未来}
\label{sec:17.8 The Future of Quantum Chemistry}
	20 世纪 50 年代,人们普遍认为,除了非常小的分子之外,对所有分子性质进行有意义的从头计算法是不可能的。在这一时期撰写的量子化学书籍中,有这样的表述:“我们永远无法指望(对有机化合物)进行令人满意的从头计算法”,“对于比氢分子离子更复杂的系统,从一开始就放弃获得薛定谔方程精确解的尝试是明智的”。1959 年,Mulliken 和 Roothaan 发现了多原子分子量子力学精确计算的 “瓶颈”,即评估多中心积分的困难。现在,这一瓶颈已被消除。\\
	\indent 对中等尺寸分子进行 Hartree-Fork 从头计算法和几何优化已成为常规工作,并可采用计算效率较高的方法(如 DFT、MP2 和 SCS-MP2)将电子相关性包括在内。各种量子力学方法和基集的可靠程度已通过大量计算得到证实。可以进行精确从头计算法或密度泛函计算的分子大小受到现有电子计算机的速度和存储容量的限制。随着更大、更快计算机的开发,处理更大的分子将变得可行。\\
	\indent 近年来,量子化学取得了长足进步,量子力学计算已成为帮助解决各种实际化学问 题的重要工具。多年前,有关分子的量子力学计算主要局限于理论化学家阅读的期刊,而现在,这类计算经常出现在各种化学家阅读的期刊上,如《\textit{美国化学会志}》。量子化学正被应用于溶液中离子的水合、表面催化、反应中间产物的结构和能量、生物分子的构象以及酶催化反应的研究等问题。在许多情况下,理论计算可能无法给出明确的答案,但它们往往足以让理论与实验进行富有成效的互动。此外,伍德沃德-霍夫曼规则等定性概念也为化学反应过程和化学键的研究提供了重要启示。\\
	\indent 1998 年诺贝尔化学奖由沃尔特·科恩(Walter Kohn)(密度函数理论的开发者之一)和约翰·波普尔(John A. Pople)(高斯系列程序和广泛使用的高斯基集、PPP方法、CNDO 和 INDO 方法的开发者之一,以及最早将 MP 和 CC 方法应用于分子计算的人之一)分享。诺贝尔奖委员会指出:计算量子化学正在 “彻底改变整个化学”。

\section*{习题}

